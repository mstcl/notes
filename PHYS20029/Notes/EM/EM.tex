%%%%%%%%%%%%%%%%%%%%%%%%%%%%%%%%%%%%%%%%%%%%%%%%%%%%%%%%%%%%%%%%%%%%%
\documentclass[12pt,chapterprefix=false,dvipsnames]{scrbook}
\KOMAoptions{twoside=false}
%%%%%%%%%%%%%%%%%%%%%%%%%%%%%%%%%%%%%%%%%%%%%%%%%%%%%%%%%%%%%%%%%%%%%

%%%%%%%%%%%%%%%%%%%%%%%%%%%%%%%%%%%%%%%%%%%%%%%%%%%%%%%%%%%%%%%%%%%%%
% Useful packages
\usepackage{mathtools}
\usepackage{booktabs}
\usepackage{svg}
\usepackage{amssymb,amsmath,physics,bm,bbold}
\usepackage{enumerate}

%--------------------------------------------------------------------
% Hyper ref
\usepackage{hyperref}
\usepackage{cleveref}

\colorlet{mylinkcolor}{NavyBlue}
\colorlet{mycitecolor}{Aquamarine}
\colorlet{myurlcolor}{Aquamarine}

\hypersetup{
	linkcolor  = mylinkcolor!, citecolor  = mycitecolor!, urlcolor = myurlcolor!, colorlinks = true, }
%--------------------------------------------------------------------
\usepackage[]{natbib}% Bibliography
\bibliographystyle{chicago}

%=================================
% pre-defined theorem environments
\usepackage{amsthm}
\usepackage{framed}
\newtheoremstyle{dotless}{}{}{\itshape}{}{\bfseries}{}{ }{}
\theoremstyle{dotless}
\newtheorem{prototheorem}{Theorem}[section]

\newenvironment{theorem}
{\colorlet{shadecolor}{orange!15}\begin{shaded}\begin{prototheorem}}
			{\end{prototheorem}\end{shaded}}

\newtheorem{protolemma}[prototheorem]{Lemma}
\newenvironment{lemma}
{\colorlet{shadecolor}{blue!15}\begin{shaded}\begin{protolemma}}
			{\end{protolemma}\end{shaded}}

\newtheorem{protocorollary}[prototheorem]{Corollary}
\newenvironment{corollary}
{\colorlet{shadecolor}{pink!15}\begin{shaded}\begin{protocorollary}}
			{\end{protocorollary}\end{shaded}}

\theoremstyle{definition}
\newtheorem{protonotation}{Notation}[section]
\newenvironment{notation}
{\colorlet{shadecolor}{green!15}\begin{shaded}\begin{protonotation}}
			{\end{protonotation}\end{shaded}}

\newtheorem{protoexample}{Example}[section]
\newenvironment{example}
{\colorlet{shadecolor}{red!15}\begin{shaded}\begin{protoexample}}
			{\end{protoexample}\end{shaded}}

\newtheorem{protodefinition}{Definition}[section]
\newenvironment{definition}
{\colorlet{shadecolor}{black!15}\begin{shaded}\begin{protodefinition}}
			{\end{protodefinition}\end{shaded}}

\newtheorem{protoderivation}{Derivation}[section]
\newenvironment{derivation}
{\colorlet{shadecolor}{purple!15}\begin{shaded}\begin{protoderivation}}
			{\end{protoderivation}\end{shaded}}
%=================================
% useful commands
\DeclareMathOperator*{\argmin}{arg\,min}
\DeclareMathOperator*{\argmax}{arg\,max}
\DeclareMathOperator*{\supp}{supp}

\def\vec#1{{\ensuremath{\bm{{#1}}}}}
\def\mat#1{\vec{#1}}

%=================================
% convenient notations
\newcommand{\XX}{\mathbb{X}}
\newcommand{\RR}{\mathbb{R}}
\newcommand{\EE}{\mathbb{E}}
\newcommand{\PP}{\mathbb{P}}
\newcommand{\CC}{\mathbb{C}}

\newcommand{\sL}{\mathcal{L}}
\newcommand{\sX}{\mathcal{X}}
\newcommand{\sY}{\mathcal{Y}}

\newcommand{\ind}{\mathbb{1}}

%%%%%%%%%%%%%%%%%%%%%%%%%%%%%%%%%%%%%%%%%%%%%%%%%%%%%%%%%%%%%%%%%%%%%
% Typography, change document font
\usepackage[tt=false, type1=true]{libertine}
\usepackage[varqu]{zi4}
\usepackage[libertine]{newtxmath}
\usepackage[T1]{fontenc}

\usepackage[protrusion=true,expansion=true]{microtype}

% Disable paragraph indentation, and increase gap
\usepackage{parskip}

\title{Electromagnetic Notes}
\author{Compiled by Nhat Pham\\ based on lectures from PHYS20029 \\and
	Griffiths' \textit{Introduction to Electrodynamics} } \date{Last update: \today}

\begin{document}

\maketitle

\tableofcontents

\chapter{Introduction}%
\label{cha:introduction}

\section{Recap of first year Electromagnetism}%
\label{sec:recap_of_first_year_electromagnetism}%

What we covered in first year:

\begin{itemize}
	\item \textit{Gauss' Law}: Integral over enclosed
	      surface containing an electric field gives the total charge over
	      that surface.
	      \begin{equation}
		      \iint\limits_{S}\bm{E}\,\dd{\bm{S}}
		      = \frac{Q}{\epsilon_{0}}
	      \end{equation}
	\item \textit{Ampère's law:} Path of a magnetic
	      field around a line integral is proportional to the current.
	      \begin{equation}
		      \oint\limits_{P}\bm{B}\,\dd{\bm{l}}
		      = \mu_{0}I
	      \end{equation}
	\item \textit{Biot-Savart law:}
	      Magnetic field arising from a small current containing element
	      in the wire. Equivalent in magnetism to \textit{Coulomb's law}
	      in electrostatic.
	      \begin{equation}
		      \bm{B}\left(\bm{r}\right) =
		      \frac{\mu_{0}I}{4\pi}\int\frac{\dd{\bm{l}}\times\left(\bm{r}-\bm{r^\prime}\right)}{\left|\bm{r}-\bm{r^\prime}\right|^3}
	      \end{equation}
\end{itemize}

This course will be concerned with deriving and using the
\textit{differential forms} of these integral equations. We will
eventually arrive at Maxwell's equations. We will also consider
two new fields $\bm{D}$ and
$\bm{H}$.

\textbf{Note:} Be aware of c.g.s system that changes
the formulae as well as the units

\chapter{Electrostatics}%
\label{cha:electrostatics}

\section{Electrostatics---what you know so far}%
\label{sec:electrostatics_what_you_know_so_far}%

\begin{definition}
	Electronic charge is a property that is associated with the
	fundamental particles, protons (quarks), electrons etc.\ that
	occur in nature.
\end{definition}

The Coulomb charge is the smallest free charge observed
(fractional charges of quarks are smaller but isolated quarks do
not appear in nature).

To properly consider the electromagnetic behaviour, we need
quantum theory in atomic length scales. E.g., the quantum
description of the hydrogen atom is the application of the
coulomb potential in Schrödinger's equation. We are only
learning classical electrodynamics.

\section{Coulomb's Law}%
\label{sec:coulomb_s_law}

\begin{definition}
	The force between two charged particles in S.I.\@ units is
	\begin{equation}
		\bm{F} =
		\frac{q_{1}q_{2}}{4\pi\epsilon_{0}r^{2}}\hat{\bm{r}}
	\end{equation}
\end{definition}

The total force on a test charge $Q$ is the
sum of the forces from the other charges in the system. This is
the \textit{superposition principle}. I.e., the field from one particle
does not change the effect from any other charges in the system.
\begin{definition}
	Superposition of electric forces applies, such that
	\begin{equation}
		\bm{F} = \bm{F_{1}} +
		\bm{F_{2}} + \hdots
	\end{equation}
	The electric field $\bm{E}$ at point
	$r$ is the force per unit charge exerted on a
	test charge, such that
	\begin{equation}
		\label{eq:force_charge_field}
		\bm{F} = Q_{test}
		\bm{E}
	\end{equation}
	From this, we can deduce that the total electric field is the
	superposition of electric fields of all charges in the system
	\begin{equation}
		\bm{E} = \bm{E_{1}} +
		\bm{E_{2}} + \hdots
	\end{equation}
\end{definition}

Note that superposition in this case is not a logical necessity
but an experimental fact; if the force is proportional to the
square of the charges, then this would not work.

We might ask---what is an electric field? We come to it through
an intermediary step in calculating forces, thus we can define
it as that. Otherwise, we can treat it as abstract or physical,
it does not affect how these particles behave.

\section{Total charge}%
\label{sec:total_charge}

\begin{definition}
	The total charge in a system of discrete (point charges) is
	\begin{equation}
		Q = \sum_i q_{i}
	\end{equation}
\end{definition}

For continuous charge distributions, the sum becomes an
integral, and we consider instead the \textit{charge densities}.
For each dimension, the charge densities are:

\begin{center}
	\begin{tabular}{c  c}
		\textbf{System} & \textbf{Unit charge relation}
		\\
		\toprule
		Line charge     & $\lambda\,\dd{l} = \dd{q}$        \\ Surface
		charge          & $\sigma\,\dd{a}^\prime = \dd{q}$  \\ Volume
		charge          & $\rho\,\dd{\tau^\prime} = \dd{q}$
		\\
	\end{tabular}
\end{center}

\begin{definition}
	The total charge in system of continuous charge with a charge
	density is
	\begin{equation}
		\label{eq:continuous_charge}
		Q = \int\limits_{\textit{body}}\dd{q}
	\end{equation}
\end{definition}

Greek symbols have been used because $V$ is
used for potentials.

\section{Charge densities and fields}%
\label{sec:charge_densities_and_fields}%

Knowing the charge densities and the total charge, we can write
Coulomb's law for electric fields. As an example, with 3D charge
density, we start with the unit electric field, in differential
form,
\begin{equation}
	\dd{\bm{E}}(\bm{r}) =
	\frac{1}{4\pi\epsilon_{0}\bm{r}^{2}}\rho\,\dd{\tau}
\end{equation}

Integrating both sides will give us the resulting electric
field.

\begin{definition}
	Coulomb's law for a continuous charge distribution is
	\begin{equation}
		\label{eq:coulombs_law_field_continuous}
		\bm{E}(\bm{r}) =
		\frac{1}{4\pi\epsilon_{0}}\int\frac{\rho(\bm{r}^\prime)}{r^2}\bm{\hat{r}}\,\dd{\tau}
	\end{equation}
\end{definition}

\begin{example}
	The examples asked us to
	\begin{enumerate}
		\item Derive electric field at a vertical distance above a line of
		      charge.
		\item Derive electric field at a vertical distance above a circular
		      loop and thus electric field from a flat circular disk.
	\end{enumerate}

	Some takeaways:

	If we are asked to find the electric field of a surface or
	volume:
	\begin{enumerate}
		\item Split into smaller dimension, \textit{unit length} for a
		      surface, \textit{unit surface} for a volume.
		\item Find the electric field of the smaller dimension shape. \item Express the unit charge in terms of charge density.
		\item Integrate over the original shape's limits.
	\end{enumerate}
\end{example}

\section{Gauss' Law}%
\label{sec:gauss_law}

\begin{definition}
	For any volume or surface that encloses a charge
	$Q$ then
	\begin{equation}
		\oint \bm{E} \vdot \dd{\bm{a}} =
		\frac{1}{\epsilon_{0}}Q
	\end{equation}
\end{definition}

If we have high symmetry in the charge distribution, we can
integrate over a symmetrical surface to find the electric field.
For more complicated situations, Gauss' law
\textit{and} the superposition principle if there
are still underlying symmetries to be exploited.

Charge distributions that are a superposition of any Gaussian
distributions asks us to use the superposition principle to
evaluate the integral.

\begin{example}
	The examples asked us to
	\begin{enumerate}
		\item Find the flux through faces of a cube of a charge at the corner
		      of a cube. We can resize the cube to centre the charge so that
		      we can use Gauss' law from symmetry
		\item Find the field of a uniformly charged solid sphere \item Find electric field well inside a long cylinder with charge
		      density that varies by perpendicular distance from principle
		      axis.
		\item Find the electric field from an infinite sheet with surface
		      charge.
		\item Find the field of two infinite sheets with surface charge
		      opposite each other (field on either sides and in the middle)
	\end{enumerate}

	The common strategy seems to be finding symmetry or finding a
	smaller part of the original geometry and integrate. Draw some
	field lines to get the feel for the geometry of the problem.

\end{example}

\section{Drawing fields}%
\label{sec:drawing_fields}

The field from a positive charge always point outwards and the
magnitude decreases as $1/r^{2}$. Field lines are
represented as arrows that give its direction and whose lengths
give its magnitude. Alternative, we can connect the neighbouring
arrows and the density of the field lines can represent the
strength of the field instead of the length.

\begin{figure}
	\centering
	\includesvg[width=0.9\columnwidth]{field_lines.svg}
	\caption{(a) shows vector field plot while (b) shows field lines method.}
\end{figure}

\textbf{Aside:} Plotting field lines via Python
utilises
\verb|matplotlib|.

\section{Differential form of Gauss' Law}%
\label{sec:differential_form_of_gauss_law}%

We can re-write Gauss' law by applying the divergence theorem.

\begin{theorem}\label{theorem:divergence}
	The divergence theorem for a vector field $\bm{X}$
	is
	\begin{equation}
		\oint\limits_{S}\bm{X}\vdot
		\dd{\bm{S}} =
		\int\limits_{V}\divergence{\bm{X}}\dd{V}
	\end{equation}
\end{theorem}

For Gauss' Law we can derive from the enclosed charge density in
a volume $V$ from
equation~\ref{eq:continuous_charge} a new identity. Thus we
discover that for an enclosed charge,

\begin{equation}
	\frac{Q}{\epsilon_{0}} =
	\frac{1}{\epsilon_{0}}\int\limits_{\tau}
	\rho(\bm{r})\dd{\tau}
\end{equation}

Combine with the divergence theorem
from~\ref{theorem:divergence}, we find that

\begin{equation}
	\int\limits_{\tau} \divergence{\bm{E}}
	\dd{\tau} = \int\limits_{\tau}
	\frac{\rho(\bm{r})}{\epsilon_{0}} \dd{\tau}
\end{equation}

\begin{definition}
	We arrive at the differential form by differentiating both sides
	with respect to $\tau$
	\begin{equation}
		\label{eq:maxwell_gauss_law_differential}
		\curl{\bm{E}} = \frac{\rho(\bm{r})}{\epsilon_{0}}
	\end{equation}
	This is the first of Maxwell's equation.
\end{definition}

\section{The curl of the electric field}%
\label{sec:the_curl_of_e_}%

Consider the electric field for a static charge
$q$, in spherical coordinates, the length
differential is

\begin{equation}
	\dd{\bm{l}} =
	\dd{r}\hat{\bm{r}} +
	r\dd{\theta}\hat{\bm{\theta}} + r
	\sin{\theta}\dd{\phi}\hat{\bm{\phi}}
\end{equation}

Since our system is a singular charge with spherical symmetry,
the angular differential terms disappear. Thus, we are left with

\begin{equation}
	\bm{E} \vdot \dd{\bm{l}} =
	\frac{q}{4\pi\epsilon_{0}r^2}\dd{r}
\end{equation}

Integrating both sides leaves us with

\begin{equation}
	\int_a^b \bm{E} \vdot \dd{\bm{l}}
	=
	\frac{q}{4\pi\epsilon_{0}}\int_a^b \frac{1}{r^2}
	\dd{r}
	=
	\frac{q}{4\pi\epsilon_{0}}\left(\frac{1}{r_{a}} -
	\frac{1}{r_{b}}\right)
\end{equation}

For a closed loop, the integral evaluates to 0.

\begin{corollary}
	\begin{equation}
		\oint \bm{E}\vdot\dd{\bm{l}}
		=
		0
	\end{equation}
	In other words, $\bm{E}$ is a conservative field.
	Using Stoke's theorem, we can conclude that
	\begin{equation}
		\label{eq:curl_of_e}
		\curl{\bm{E}} = 0
	\end{equation}
\end{corollary}

\textbf{Note:} This result only applies for
electrostatic fields and not when there are time varying
magnetic fields.

\section{Electrostatic potential}%
\label{sec:electrostatic_potential}%

From~\ref{sec:the_curl_of_e_} we find that for an electrostatic
field, it is conservative. The following applies to any
conservative field

\begin{definition}
	The electrostatic potential as a function of a position vector
	is the negative of the integral of the electrostatic field along
	some path from a starting point to another point. Since it is
	conservative, it is path-independent, hence the difference in
	potential at two points is the integral evaluated from one point
	to the other.
	\begin{equation}
		\label{eq:potential_vector_field_int} V(\bm{b}) -
		V(\bm{a}) = -
		\int_a^b\bm{E}\vdot\dd{\bm{l}}
	\end{equation}
	As a result, we have the following identity for the electric
	potential:
	\begin{equation}
		\label{eq:potential_vector_field}
		\bm{E} = -\gradient{V}
	\end{equation}
\end{definition}

\begin{example}
	Find the potential inside and outside a uniformly charged
	spherical shell of radius $R$. Very similar
	steps to~\ref{dv:charge_from_average}.

	Inside, it is uniform and depends on the radius of the sphere,
	while outside, it depends on the position of our test charge.

	We get to
	\begin{equation}
		V(z) = \frac{2\pi R\sigma}{2\epsilon_{0}z}\left[\sqrt{{(R+z)}^2} - \sqrt{{(R-z)}^2} \right]
	\end{equation}
	and must consider that the second square root term is
	$z - R$ for test charge outside and
	$R - z$ for test charge inside.

	Thus, we have
	\begin{equation}
		V(r) = \begin{cases}
			\frac{1}{4\pi\epsilon_{0}}\frac{q}{r}, & r \geq R \\
			\frac{1}{4\pi\epsilon_{0}}\frac{q}{R}, & r \leq R \\
		\end{cases}
	\end{equation}
\end{example}

\section{Notes on the Scalar potential}%
\label{sec:notes_on_the_scalar_potential}%

\begin{itemize}
	\item Potential is different from potential energy.
	\item Finding potentials is easier than vector fields.
	\item Finding vector fields using the
	      relation~\ref{eq:potential_vector_field}. \item There is not an absolute definition of potential---we only
	      observe potential \textit{differences}. Thus, `zero'
	      potential is arbitrarily defined, usually at infinity. This is
	      for convenience rather than mathematical necessity. \item Potentials also follow the superposition principle. \end{itemize}

\section{Poisson's equation}%
\label{sec:poisson_s_equation}

Substitute~\ref{eq:potential_vector_field} into the curl of electric
field, and we get Poisson's equation.

\begin{definition}
	Poisson's equation is
	\begin{equation}
		\label{eq:poissons}
		\laplacian{V} = -\frac{\rho(\bm{r})}{\epsilon_{0}}
	\end{equation}
\end{definition}

\section{Laplace's equation I}%
\label{sec:laplace_s_equation}

\begin{definition}
	In the absence of any charges, $\rho(\bm{r}) = 0$. Thus,
	Laplace's equation is
	\begin{equation}
		\laplacian{V}=0
	\end{equation}
\end{definition}

\section{Point charge}%
\label{sec:point_charge}

A point charge has spherical symmetry. If we choose our zero
potential at infinity and evaluate the integral then we can
evaluate the integral to find the electrostatic potential.

\begin{definition}
	The electrostatic potential of a point charge is
	\begin{equation}
		V(\bm{r}) =
		\frac{1}{4\pi\epsilon_{0}}\frac{q}{r}
	\end{equation}
\end{definition}

Note that point charges do not actually exist, they are there
for easier calculations.

\section{Electrostatic potential for a continuous distribution}%
\label{sec:electrostatic_potential_for_a_continuous_distribution}%

For a continuous charge distribution, we can use the
superposition principle and sum over all point charges.

\begin{definition}
	The electrostatic potential for a continuous distribution is
	\begin{equation}
		\label{eq:potential_continuous}
		V(\bm{r}) =
		\frac{1}{4\pi\epsilon_{0}}\iiint\limits_V\frac{\rho(\bm{r}^\prime)}{r^{\prime\prime}}\dd{\tau^\prime}
	\end{equation}
	For surface and line charge densities, it is the same form,
	except we are integrating over a surface or a path.
\end{definition}

\section{Summary of equations}%
\label{sec:summary_of_equations}

Relating $V$ and $\rho$:
Equation~\ref{eq:poissons} and
equation~\ref{eq:potential_continuous}.

Relating $\rho$ and $\bm{E}$:
Equation~\ref{eq:coulombs_law_field_continuous},
equation~\ref{eq:curl_of_e} and
equation~\ref{eq:maxwell_gauss_law_differential}.

Relating $\bm{E}$ and $V$:
Equation~\ref{eq:potential_vector_field} and
equation~\ref{eq:potential_vector_field_int}.

\section{Boundary conditions for the electric field}%
\label{sec:boundary_conditions_for_the_electric_field}

For a thin box passing through a surface charge, as we pass from
below to above the surface, there is a discontinuous change in
the electric field if the box is infinitesimally thin. This
result is from first year. Since we only care about normal
components, because for tangential components to the surface
(i.e.\ sides of the box), this forms a closed loop and the
integral is thus 0. We find that the normal component has a
discontinuity, so that

\begin{equation}
	\label{eq:thin_box_sheet}
	\bm{E}^\perp_{\mathrm{above}} -
	\bm{E}^\perp_{\mathrm{below}} =
	\frac{\sigma}{\epsilon_{0}}\hat{\bm{n}}
\end{equation}

We can integrate this to find the potential above and below. As
the integral path length tends to zero, the integral tends to
zero, so that the potential above is equal to the potential
below.

\begin{equation}
	V_{\mathrm{above}} = V_{\mathrm{below}}
\end{equation}

We can also arrange~\ref{eq:thin_box_sheet} to get

\begin{definition}
	The boundary conditions we have for the electrostatic field due
	to surface charges
	\begin{equation}
		\label{eq:boundary_conditions}
		\pdv{V_{\mathrm{above}}}{n} - \pdv{V_{\mathrm{below}}}{n} =
		-\frac{\sigma}{\epsilon_{0}}
	\end{equation}
\end{definition}

Where the partial derivatives represent the normal derivative
$
	\gradient{\hat{n}}$.

\section{Work done to move a charge}%
\label{sec:work_done_to_move_a_charge}%

\begin{definition}
	The work needed to move a charge in an electric field is
	\begin{equation}
		\label{eq:work_potential_charge}
		W = -Q \int_a^b \bm{E} \vdot
		\dd{ \bm{l}  } = Q\left[V( \bm{b}  ) - V(
			\bm{a})\right] = Q \Delta V
	\end{equation}
\end{definition}

\section{The energy of a distribution of point charges}%
\label{sec:the_energy_of_a_distribution_of_point_charges}%

To move one charge to the other, separated by a distance of
infinity, to a finite distance $r$, the work
done is

\begin{equation}
	W_2 = \frac{1}{4\pi\epsilon_{0}}\frac{q_2q_1}{r_{12}}
\end{equation}

The superposition principle is used to find the total work done
in moving two or more charges together. It is the sum of the
work done to bring each individual pair together.

\begin{equation}
	W = \sum_{i=2}^N W_i
\end{equation}

To avoid double counting the pairs, we halve the final sum,
which yields
\begin{equation}
	W = \frac{1}{8\pi\epsilon_{0}} \sum_{i=1}^N
	\sum_{j \neq i}^N \frac{q_i q_j}{r_{ij}}
\end{equation}

We can rearrange this equation further
using~\ref{eq:work_potential_charge}.

\begin{definition}
	The work done in moving all these charges from infinity to a
	finite distance is

	\begin{equation}
		W = \frac{1}{2} \sum_{i=1}^N q_i V(
		\bm{r}_i  )
	\end{equation}
\end{definition}

\section{The energy of a continuous charge distribution}%
\label{sec:the_energy_of_a_continuous_charge_distribution}%

Once again, we can integrate to convert from discrete charges to
continuous charges. We can substitute
in~\ref{eq:maxwell_gauss_law_differential} then integrate by parts to get a
nicer expression.

\begin{derivation}
	\begin{equation}
		\begin{aligned}
			W & = \frac{\epsilon_{0}}{2}\int\left(\divergence{\bm{E}}\right)V \dd{\tau} \\
			  & = \frac{\epsilon_{0}}{2}\left[-\int\bm{E}\vdot\gradient{V}
				\dd{\tau} + \oint V \bm{E}
			\dd{ \bm{a} }\right]                                                        \\
			  & = \frac{\epsilon_{0}}{2}\left[-\int\bm{E}^2\dd{\tau}+\oint V
				\bm{E}\dd{\bm{a}}\right]
		\end{aligned}
	\end{equation}
\end{derivation}

This is the \textit{correct} equation, but in theory if
you integrate over bigger volumes (as long as it encloses the
charge), the contribution from the volume will overtake the
contribution from the surface. The first term is the
contribution from the volume, while the second is the surface.
Since outside our standard sphere (let's say we integrate over
all space), $\rho = 0$, our result for
$W$ must be the same. But the volume
integral grows as $E^2$ is positive, so the
surface integral must decrease. In fact, it decreases by
$1/r$. Thus, if we integrate over all space,
we are left with the first term only.

\begin{definition}
	The work done on continuous charge distribution is
	\begin{equation}
		\label{eq:work_continuous}
		W = \frac{\epsilon_{0}}{2}\int\limits_{\mathrm{all\ space}} E^2
		\dd{\tau}
	\end{equation}
\end{definition}

This gives us a different idea to think about energy in
electrostatics. If we have a series of charges (discrete or
continuous). Rather than thinking in terms of the work done by
bringing these charges together, we can think of the work as the
integral of the electric field over all space. One could say the
electric field `stores' energy.

It is \textbf{important} to understand that the energy of
the electric field does not obey the superposition principle. If
we have two charges and bring them together, the total energy is
not the energy in one field plus the energy in the other. This
is because it is not a linear relationship.

\begin{example}
	Find the electrostatic energy of a uniformly charged spherical
	shell of radius $R$ and total charge
	$q$.

	Note that we can use either use
	\begin{equation}
		W = \frac{1}{2} \int \sigma V
		\dd{a}
	\end{equation}
	or
	\begin{equation}
		W = \frac{1}{2} \int E^2 \dd{\tau}
	\end{equation}
	to get to the same result.
\end{example}

\section{The properties of conductors}%
\label{sec:the_properties_of_conductors}%

\begin{definition}
	Inside a conductor, there are several important properties to
	remember
	\begin{itemize}
		\item $\bm{E} = 0$ inside a conductor. If there is a deficit
		      of charge, then it will move until equilibrium is reached.
		\item $\rho = 0$ inside a conductor. We can have charges
		      inside a conductor, but they must cancel.
		\item Charges in a conductor are \textbf{mobile}. This means
		      they can move freely.
		\item Any net charge resides on the \textbf{surface}.
		\item A surface of a conductor is an \textbf{equipotential},
		      otherwise charges would move freely until potential equilibrium
		      is reached.
		\item $\bm{E}$ is perpendicular to the surface at the
		      surface. Tangential components would mean charges can move along
		      the surface, this is not the case for an equipotential.
	\end{itemize}
\end{definition}

\section{Induced charges}%
\label{sec:induced_charges}

If a positively charged particle is brought closed to a
conductor, its field will attract negative charges towards it.
These negative charges are closer to the positive particle and
thus creates a net attraction. The charges in the conductor move
around the surface to cancel the field in the conductor so that
at equilibrium there is no $\bm{E}$ field in the
conductor.

The result is a redistribution of charge on the surface of the
conductor.

\begin{example}
	An uncharged spherical conductor has a cavity of arbitrary shape
	with a charge $+q$ placed somewhere in it.
	What is the electric field outside of the conductor?

	Outside the conductor, using Gauss' Law, for a spherical
	conductor, it is
	\begin{equation}
		\frac{1}{4\pi\epsilon_{0}}\frac{q}{r^2}
		\hat{ \bm{r}  }
	\end{equation}

	Does a strangely shaped cavity affect the distribution of the
	charge distributed on the surface? The arrangement of negative
	charges in the inside surface of the cvity that is attracted to
	the cavity cancels out the asymmetry that is produced by the
	shape of the cavity. The net effect is that the shape of the
	cavity does not matter. We can simply argue that we have a
	uniform surface charge. See~\ref{sec:uniqueness_theorems} for
	relevant theorems.
\end{example}
\section{Surface charge and the force on a conductor}%
\label{sec:surface_charge_and_the_force_on_a_conductor}%

If we have a \textit{static} field on a conductor, then
the boundary conditions for the charge at the surface follow the
result from~\ref{sec:boundary_conditions_for_the_electric_field}. We have a discontinuity in
the field, and immediately outside the conductor, it is given
by~\ref{eq:thin_box_sheet}.

We can rearrange for the charge density and express it in terms
of the potential
\begin{equation}
	\sigma = -\epsilon_{0} \pdv{V}{n}
\end{equation}

Using the expression for force in~\ref{eq:force_charge_field}, we
can express the force \textit{per unit area} as

\begin{equation}
	\label{eq:force_per_area_conductor}
	\bm{f} = \sigma \bm{E_{\mathrm{avg}}}
\end{equation}

Because there is a discontinuity, we take the
\textbf{average} of the electric field above and below
(or inside and outside).

For a conductor, this is half of~\ref{eq:force_per_area_conductor} since
the field is 0 inside and outside, it is given
by~\ref{eq:thin_box_sheet}.

\begin{definition}
	The force per unit area for a static field on a conductor is
	\begin{equation}
		\bm{f} = \frac{1}{2\epsilon_{0}}\sigma^2
		\hat{ \bm{n} }
	\end{equation}
	or as a pressure, it is
	\begin{equation}
		P = \frac{\epsilon_{0}}{2}E^2
	\end{equation}
\end{definition}

\section{Capacitors}%
\label{sec:capacitors}

A capacitor is a device with two conductors separated in space.

Suppose there is a charge $+Q$ and
$-Q$ on each conductor respectively. Each
conductor is an equipotential, thus we can consider the
\textit{potential difference} between them as the integral of the
electric field from the negative to the positive conductor.

Since electric field is proportional to the
charge~\ref{eq:force_charge_field}, or more general
as~\ref{eq:coulombs_law_field_continuous} we can substitute this into the
integral, so the potential difference is proportional to the
charge

We call this constant of proportionality the
\textit{capacitance}.

\begin{definition}
	The relation between the potential difference and charge is
	\begin{equation}
		\label{eq:capacitance_charge_pd}
		C = \frac{Q}{V}
	\end{equation}
\end{definition}

\begin{example}
	Find the capacitance of a parallel plate capacitor made up of
	two metal sheets of area $A$ separated by a
	distance $d$.

	Recall that for a linear system like this,
	\begin{equation}
		E = \frac{V}{d}
	\end{equation}
	After some algebra, we are left with
	\begin{equation}
		C = \frac{\epsilon_{0}A}{d}
	\end{equation}
\end{example}

\begin{example}
	Find the capacitance of two concentric spherical metallic shells
	with radii $a$ and $b$,
	where $a < b$.

	We can use Gauss' Law to determine the electric field for
	various radii. It is non-zero for $a \leq r \leq b$. As
	usual, we can determine the potential through integration.

	Then we go one step further and establish the capacitance using
	the non-zero potential between the two shells. This turns out to
	be
	\begin{equation}
		C = \frac{4\pi\epsilon_{0}ab}{(b-a)}
	\end{equation}
\end{example}
\section{Energy stored in a capacitor}%
\label{sec:energy_stored_in_a_capacitor}%

We can ask: how much work is needed to ``charge'' a capacitor?
This is equivalent to moving charges from the positive plate to
the negative plate.

We need to work \textit{harder} as we transfer more
charge from one plate to the other.

If a plate already as charge $q$ on it, with
potential difference between plates $V = q/C$,
then from~\ref{eq:work_potential_charge}, the work needed to add a
charge of $\dd{q}$ is

\begin{equation}
	\dd{W} = V \dd{q} =
	\frac{q}{C}\dd{q}
\end{equation}

Integrate this equation to get the \textit{total work done} or
\textit{total energy stored}. Further, we can use our derived
expressions to rearrange.

\begin{definition}
	The energy stored in a capacitor or work done to charge a
	capacitor is
	\begin{equation}
		\label{eq:energy_in_capacitor} W =
		\frac{1}{2} CV^2
	\end{equation}
\end{definition}

If we now connect our capacitors to some wires then the
capacitor will discharge because of the potential difference
between the two plates. Thus, the capacitor is said to be
storing energy. In fact, if the space between the plates is air
then it would require really big capacitors for a substantial
amount of energy. Dielectrics would be a better material.

\chapter{Potentials}%
\label{cha:potentials}

\section{Introduction to potentials}%
\label{sec:introduction_to_potentials}

To find the electric field or potential from charge densities
(as shown in~\ref{sec:summary_of_equations}) is quite difficult as not
all of them have simple analytic solutions.

For conductors, we also do not know where the charges are in
advance, since the electric field inside is 0.

In most cases, it is better to start from Poisson's equation
in~\ref{sec:poisson_s_equation} in regions with charge, and
Laplace's equation in~\ref{sec:laplace_s_equation} in regions with
no charge.

\section{Laplace's equation II}%
\label{sec:laplace_s_equation_ii}

\subsection{3D Laplacian}%
\label{sub:3d_laplacian}

If we look at the 3D Laplace's equation in full:
\begin{equation}
	\label{eq:3D_laplace_s_equation_in_full}
	\pdv[2]{V}{x} + \pdv[2]{V}{y} +
	\pdv[2]{V}{z} = 0
\end{equation}

We see that we have some expressions of requirements. Normally,
the second derivative will tell us whether there's a maximum or
minimum (positive or negative value). Because this is equal to
0, we cannot have a minimum for all three directions, otherwise,
it would not be equal to 0.

Relevantly, we have Earnshaw's theorem, which maintains that a
collection of point charges cannot be maintained in a stable
stationary equilibrium (impossible to trap a particle in a 3D
electric field). If we can trap a particle, then it would mean
all points must be a minimum. To follow Laplace's equation, we
can only have unstable points (no local maxima or minima)

\subsection{1D Laplacian}%
\label{sub:1d_laplacian}

In 1D, it is a relatively straightforward, the first derivate is
just a constant, and the function itself is a linear function.

If we take any point on that straight line, and look at two
neighbouring points, then the potential is just the average
potential of the two surround points.

\subsection{2D Laplacian}%
\label{sub:2d_laplacian}

In 2D, like in 3D, it's not possible to write a general
solution, since it is not an ODE.\@ However, a useful numerical
method yields one result such that the value of
$V$ at some point $(x,y)$ is
the average of values around $(x,y)$. If it is a
circle, then the average value is at the centre.

The average value of a function, in any dimension is given by
\begin{equation}
	\label{eq:average_value_of_a_function}
	\bar{f} = {\int_A 1}^{-1}\int_A f
\end{equation}

Thus, for a circle, the average potential is
\begin{equation}
	\label{eq:2d_laplacian_numerical}
	V_{\mathrm{avg}} = V(x,y) =
	\frac{1}{2\pi R}\oint\limits_{\mathrm{circle}} V
	\dd{l}
\end{equation}

By making this circle infinitesimally small, we get a valid
result for the potential. We can setup a spreadsheet to perform
this calculation.

\begin{figure}[htpb]
	\centering
	\includesvg[width=0.9\linewidth]{./2d_laplacian.svg}
	\caption{Suppose we want to find the potential for the box shaded
		blue, if we know the potential of its non-diagonal
		neighbours, then we can average those to find the blue box. We can do this
		via a relaxation method. If we iteratively find the potential of
		its neighbours given some boundary conditions, then we can determine the potential in the centre.}%
	\label{fig:2d_laplacian}%
\end{figure}

\section{Laplace's equation III}%
\label{sec:laplace_s_equation_iii}%

We now focus on the 3D Laplace's equation. We found in the
previous Section~\ref{sec:laplace_s_equation_ii} that the potential at
some point is just the average of the potential surrounding
points.

We can justify this by considering the surrounding points as a
sphere with a point charge $q$ outside, and
find the average potential over the whole sphere. Let's place
our external point charge along the
$z$-axis.

From our previous derivations, we can simply write the down the
result. From the origin, our potential due to
$q$ is
\begin{equation}
	V = \frac{1}{4\pi\epsilon_{0}}\frac{q}{z}
\end{equation}

Let's go along with the `average' method and try to get to the
same answer.

\begin{figure}[htpb]
	\centering
	\includesvg[width=0.9\linewidth]{./3d_laplacian.svg}
	\caption{A 3D Laplace's equation set up with a point charge outside a
		sphere we are considering the potential at the surface shaded
		orange on the sphere.}%
	\label{fig:3d_laplacian}%
\end{figure}

\begin{derivation}%
	\label{dv:charge_from_average}%
	Immediately, we expect the potential at a point
	$r_0$ on the surface to be (imagine
	$q$ is at the centre of some other sphere).
	\begin{equation}
		V = \frac{1}{4\pi\epsilon_{0}}\frac{q}{r_0}
	\end{equation}

	We can determine $r_0$ from the cosine rule.
	\begin{equation}
		r_0^2 = z^2 + R^2 - 2zR\cos{\theta}
	\end{equation}
	From~\ref{eq:average_value_of_a_function}, we can add the third dimension to
	this to represent our surface integral as
	\begin{equation}
		V_{\mathrm{avg}} =
		\frac{1}{4\pi R^2}\oiint\limits_{\mathrm{sphere}} V
		\dd{a}
	\end{equation}

	Remember, this is the average potential of the surface of a
	sphere of radius $R$ away from the origin.

	Do some substitution in spherical coordinates (varying
	$(\theta,
		\phi)$ while keeping $r = r_0$),
	bearing in mind the Jacobian and we will get that the average
	potential is
	\begin{equation}
		V_{\mathrm{avg}} =
		\frac{1}{4\pi R^2}\frac{q}{4\pi\epsilon_{0}}
		\int_0^{2\pi}\int_0^{\pi} {\left[ z^2 + R^2 - 2zR\cos{\theta} \right]}^{-\frac{1}{2}} R^2
		\sin{\theta}\dd{\theta}
		\dd{\phi}
	\end{equation}

	We can integrate with respect to $\phi$ first
	as there are no $\phi$-dependent terms
	\begin{equation}
		V_{\mathrm{avg}} =
		\frac{2\pi}{4\pi}\frac{q}{4\pi\epsilon_{0}}
		\int_0^{\pi} {\left[ z^2 + R^2 - 2zR\cos{\theta} \right]}^{-\frac{1}{2}}
		\sin{\theta}\dd{\theta}
	\end{equation}

	Then we use inverse chain rule and factorise the terms inside
	the square roots when evaluating to get to
	\begin{equation}
		V_{\mathrm{avg}} =
		\frac{1}{2zR}\frac{q}{4\pi\epsilon_{0}}
		\sqrt{ z^2 + R^2 - 2zR\cos{\theta} }\biggm \rvert^{\pi}_{0} =
		\frac{1}{2zR}\frac{q}{4\pi\epsilon_{0}}
		\left[ (z + R) - (z - R) \right]
	\end{equation}

	Finally, we have a nice expression
	\begin{equation}
		V_{\mathrm{avg}} =
		\frac{1}{4\pi\epsilon_{0}}\frac{q}{z}
	\end{equation}

	What we just calculated is the potential at the centre of the
	sphere as the average of the sphere of an arbitrary radius
	$R$ due to a point charge
	$q$ outside $R$. We can
	get this same result if we calculate directly from the external
	charge itself.
\end{derivation}

\section{Uniqueness theorems}%
\label{sec:uniqueness_theorems}

Here are the theorems without proofs.

\begin{definition}
	\textbf{First Uniqueness Theorem:}
	The solution to Laplace's equation in some volume
	$\tau$ is uniquely determined if
	$V$ is specified on the boundary surface S.
\end{definition}
\begin{definition}
	\textbf{Second Uniqueness Theorem:} In a volume $\tau$
	surrounded by conductors and containing a specified charge
	density, $\rho$, the electric field is uniquely
	determined if the \textit{total charge} is given.
\end{definition}

Let's consider examples to demonstrate the theorems.

\begin{example}
	Consider a surface $S$ which encloses some
	volume with a defined potential $V$ on that
	surface. The first uniqueness theorem tells us that there is a
	unique solution to Laplace's equation.
\end{example}

\begin{example}
	Now, consider 4 charges arranged in a way shown in
	Figure~\ref{fig:second_uniqueness_theorem}. Now we connect them. What
	happens? From the second uniqueness theorem, we can say that the
	volume enclosed by the conductors (wires) has a determined total
	charge (of 0), so we know there is a unique solution for the
	electric field.
\end{example}

\begin{figure}[htpb]
	\centering
	\includesvg[width=0.9\linewidth]{./second_uniqueness_theorem.svg}
	\caption{An example to demonstrate the second uniqueness theorem.}%
	\label{fig:second_uniqueness_theorem}%
\end{figure}

\section{Finding the potential---the method of images}%
\label{sec:finding_the_potential}%

Let's consider the classic example: ``What is the potential from
a positive charge $+q$ sitting at a distance
$d$ above a flat and infinite grounded
conductor''?

We have the following boundary conditions:
\begin{itemize}
	\item $V = 0$ when $z = 0$ (level of the
	      grounded conductor)
	\item $V \rightarrow 0$ a long way from the
	      charge
\end{itemize}

We can use a trick to solve this problem without solving
Laplace's equation. By symmetry, if we remove the conductor and
replace it with a negative charge $-q$ at
$-d$. At $z = 0$ (like a
mirror image), the potential is still 0. From the first
uniqueness theorem, we still have the same boundary potential
(an invisible conductor), and thus the solution for the
potential above the plane will be the same in both cases.

The solution is therefore
\begin{equation}
	\label{eq:charge_on_infinite_sheet_mirror}
	V(x,y,z) = \frac{1}{4\pi\epsilon_{0}}\left[\frac{+q}{\sqrt{r^2 +
	{(z-d)}^2}} + \frac{-q}{\sqrt{r^2 + {(z+d)}^2}} \right],\, z
	\geq 0
\end{equation}

where $r^2 = x^2 + y^2$. This means by using a mirror
image of our charge, we can determined the potential strictly
above the plane.

What about the surface charge, $\sigma(r)$? Recall
in Section~\ref{sec:boundary_conditions_for_the_electric_field}, the surface charge
distribution is related by the
Equation~\ref{eq:boundary_conditions}, except we only care about
what's above the sheet. The direction normal to the surface is
$\hat{k}$, thus
\begin{equation}
	\sigma = -\pdv{V}{z}\biggr \rvert_{z=0}
\end{equation}

We then evaluate the partial derivative in
Equation~\ref{eq:charge_on_infinite_sheet_mirror}, which gives us
\begin{equation}
	\sigma(R) =
	-\frac{qd}{2\pi{(r^2+d^2)}^{\frac{3}{2}}}
\end{equation}

From this we can find out the \textit{total} induced
charge $Q$. Using the
relation~\ref{eq:continuous_charge}, we integrate in spherical
coordinates (in the
$x$-$y$ plane, this
means keeping $theta$ constant) to find
\begin{equation}
	Q = \int_0^{2\pi} \int^\infty_0 -
	\frac{qd}{2\pi{(r^2+d^2)}^{\frac{3}{2}}}
	r\dd{r}\dd{\phi} =
	-\frac{qd}{2\pi\sqrt{r^2+d^2}}\biggr \rvert_0^\infty = -q
\end{equation}

How about the forces and energy in the system? First the force
between the two charges is
\begin{equation}
	\bm{F} =
	\frac{1}{4\pi\epsilon_{0}}\frac{q^2}{{{(2d)}^2}}
	\hat{ \bm{z}  }
\end{equation}

Multiply by $2d$ to get the energy. But this
is actually not correct. We have to halve the volume in the
actual system because we are only considering
$z \geq 0$, thus
\begin{equation}
	W = -\frac{1}{4\pi\epsilon_{0}}\frac{q^2}{4d}
\end{equation}

We can also get to this result by finding the work needed to
bring the charge in from infinity.
\begin{equation}
	W = -\int_\infty^d \bm{F}
	\dd{ \bm{l}  } = -\frac{1}{4\pi\epsilon_{0}} \int_\infty^d
	\frac{q^2}{4z^2} \dd{z}
\end{equation}

\section{Finding potentials---separation of variables}%
\label{sec:finding_potentials_separation_of_variables}

\textbf{Note:} There will not be any examinable
questions in solving PDEs. But we must be comfortable with
techniques such as separation of variables. We use this to solve
Laplace's equation. Furthermore, we must be able to do this in
spherical and cylindrical coordinate systems.

\section{Multipole expansions---potentials at large distances}%
\label{sec:multipole_expansions_potentials_at_large_distances}

If we go far enough away from any charge distribution with a
total net charge $Q$, the potential will
always tend to that of a single point charge, i.e.
\begin{equation}
	V_{\mathrm{net\,charge\,Q}}{\left(r\right)}_{r
	\rightarrow \infty} = \frac{Q}{4\pi\epsilon_{0}} \rightarrow 0
\end{equation}

But how do we expect the potential to look like at large
$\bm{r}$? For a distribution of charges we can
image a hierarchy of structures, starting from a single point
charge.

\begin{itemize}
	\item \textbf{Monopole:}  Potential tends to
	      $1/r$.
	\item \textbf{Dipole:}  Potential tends to
	      $1/{r}^2$.
	\item \textbf{Quadrupole:}  Potential tends to
	      $1/{r}^3$.
	\item \textbf{Octopole:}  Potential tends to
	      $1/{r}^4$.
\end{itemize}

\textbf{Note:} We can't reduce a dipole to a monopole,
a quadrupole to a dipole, octopole to a quadruple, etc. However,
we can imagine representing the potential of any charge
distribution as an expansion (sum or superposition) of these
multipoles.

Clearly, if we have what appears as a single charge at large
distances, the field looks like a point charge, if we have
separation of net charge then it will look like the dipole at
large distance. If our charge distribution cannot be reduced to
a monopole or dipole distribution then the next likely is
quadrupolar, etc.

\subsection{Multipole expansion (Not examinable)}%
\label{sub:multipole_expansion_not_examinable_}

The multipole expansion of the field from a charge distribution
is
\begin{equation}
	V\left(\bm{r}\right) =
	\frac{1}{4\pi\epsilon_{0}}\sum_{n=r_0}^{\infty}\frac{1}{r^{n+1}}\int\limits_{V}{\left(r^\prime\right)}^n
	P_n\left(\cos{\alpha}\right)\rho\left(\bm{r}^\prime\right)
	\dd{\tau^\prime}
\end{equation}

$\alpha$ is the angle between
$\bm{r}$ and $\bm{r}^\prime$ and
$P_n$ are the Legendre polynomials. The
further we go away, the more we can use only the first terms in
the series.

\section{The dipole potential}%
\label{sec:the_dipole_potential}

For two separated charges we can write the potential at some
point \textbf{P} at $\bm{r}$ as
\begin{equation}
	V\left(\bm{r}\right) =
	\frac{1}{4\pi\epsilon_{0}}\left(\frac{q}{r_{+}} - \frac{q}{r_{-}}\right)
\end{equation}
where $r_{+}$ and $r_{-}$ are
the distances from the charge to the point
\textbf{P}, see Figure~\ref{fig:dipole_distances}.
But these two are not centred at the origin of the system. The
distance can be calculated from cosine rule.
\begin{equation}
	r^2_{\pm} = r^2\left(1 \mp\frac{d}{r}\cos{\theta} + \frac{d^2}{4r^2}\right)
\end{equation}

\begin{figure}[htpb]
	\centering
	\includesvg[width=0.9\columnwidth]{./dipole_distances.svg}
	\caption{The distances in a dipole system}%
	\label{fig:dipole_distances}
\end{figure}

At large $r$ the third term is negligible,
reducing this to
\begin{equation}
	\frac{1}{r_{\pm}}
	\simeq
	\frac{1}{r}{\left(1 \mp \frac{d}{r} \cos{\theta}\right)}^{-\frac{1}{2}}
\end{equation}

Using the binomial expansion, we have
\begin{equation}
	\frac{1}{r_\pm}
	\simeq
	\frac{1}{r}\left(1 \pm \frac{d}{2r} \cos{\theta}\right)
\end{equation}

If we write this expansion for $r_+$ and
$r_-$ separately, and subtract one from the
other.
\begin{equation}
	\frac{1}{r_+} - \frac{1}{r_-}
	\simeq
	\frac{d}{r^2}\cos{\theta}
\end{equation}

The potential of a dipole measured from its centre for large
$r$ is
\begin{equation}
	V \left(\bm{r}\right)\simeq
	\frac{1}{4\pi\epsilon_{0}}\frac{q d \cos{\theta}}{r^2}
\end{equation}
where $\theta$ is the relative orientation of the
dipole to the position vector of \textbf{P}.

Since this seems to imply that we have to work in spherical or
polar coordinates, we can avoid this by using the dipole moment,
measured as a vector $\bm{d}$ from one end to the
other.
\begin{equation}
	\label{eq:dipole_moment_def}
	\bm{p} = q \bm{d}
\end{equation}

\begin{definition}
	At large $r$, the potential of a dipole
	measured from its centre, using the dipole moment is
	\begin{equation}
		\label{eq:potential_of_dipole}
		V \left(\bm{r}\right) =
		\frac{1}{4\pi\epsilon_{0}}\frac{ \bm{p}\vdot \hat{ \bm{r}  }  }{r^2}
	\end{equation}
\end{definition}
\textbf{Note:} This is only the approximate potential
of the physical dipole we described. If we get too close to the
separated charges, this equation no longer applies. We would
have to use multipole expansion to get a better approximation.

When we solve problems, it is sensible to align
$\bm{p}$ with an axis to simply the mathematics
needed to carry out the calculation.

\section{The electric field from a dipole}%
\label{sec:the_electric_field_from_a_dipole}

The electric field is the gradient of the potential. We have
calculated the potential in Equation~\ref{eq:potential_of_dipole}.
We need to work out the gradient in spherical coordinates.

\begin{definition}
	The electric field of a dipole, aligned along the
	$z$-axis is
	\begin{equation}
		\label{eq:electric_field_dipole_z}
		\bm{E}_{\mathrm{dipole}}\left(r,\theta\right)
		=
		\frac{p}{4\pi\epsilon_{0}r^3}\left(2\cos\theta \hat{ \bm{r}  } + \sin\theta \hat{ \bm{\theta}  }\right)
	\end{equation}

	The coordinate-free version, which does not depend on the
	orientation of the system, is
	\begin{equation}
		\bm{E}_{\mathrm{dipole}}\left(r\right)
		=
		\frac{p}{4\pi\epsilon_{0}r^3}\left(3\left(\bm{p}\vdot\hat{\bm{r}  }\right)\hat{\bm{r}} - \bm{p}\right)
	\end{equation}
\end{definition}

See Figure~\ref{fig:derive_E_of_dipole} for a derivation.

We can see that if the dipole is aligned with the
$z$-axis, the terms from other axis
disappears and is what we expect.

\begin{figure}[htpb]
	\centering
	\includesvg[width=1\columnwidth]{./derive_E_of_dipole.svg}
	\caption{Handwritten derivation of the electric field of a dipole}%
	\label{fig:derive_E_of_dipole}
\end{figure}

\chapter{Fields in Matter}%
\label{cha:fields_in_matter}

\section{Introduction to fields in matter}%
\label{sec:introduction_to_fields_in_matter}

We might ask: how does an electric field interact with different
materials?

\begin{itemize}
	\item A proper theory requires us to embrace quantum mechanics, we
	      will not do this but stick to the nineteenth-century
	      interpretation.
	\item Broadly, we have \textbf{conductors} (charges can move freely) and
	      \textbf{insulators} (charges can move but only in small displacements since they are fixed to atoms).
	\item Insulators are often called dielectrics but other types exist,
	      such as paraelectric, ferroelectric, piezoelectric.
	\item There are materials between these extremes: semiconductors.
	      These need to be treated with quantum theory but the action of
	      fields in them is very important in devices.
\end{itemize}

\section{Polarised atom}%
\label{sec:polarised_atom}

Assume that we understand an atom as neutral with a positive
nucleus and surrounding electrons, what happens if an electric
field $\bm{E}$ is placed across it?

The field will try to move the nucleus in one direction and the
electrons in the opposite direction. I.e., the centre of the
charge distributions separate and they create their own dipole
field.

This means if we apply an external field on an atom, we will
induce a dipole $\bm{p}$ in the atom. In other
words, the atom becomes polarised.

\begin{definition}
	A polarised atom has the dipole moment
	\begin{equation}
		\bm{p}_{\mathrm{atom}}
		=
		\alpha \bm{E}
	\end{equation}
	where the constant of proportionality $\alpha$ is
	the atomic polarisability, given by
	\begin{equation}
		\alpha
		=
		4\pi\epsilon_{0}a^3 =
		3\epsilon_{0}v_{\mathrm{atom}}
	\end{equation}
\end{definition}

\textbf{Note:} we have assumed here that there is
\textit{linear response}. This is fine for `small' fields but as
the field gets large enough it would eventually pull the atom
apart and ionise the atom.

\begin{example}
	Assume a simple model for an atom with a positive charge
	$q+$ for the nucleus and a sphere of charge
	$-q$ and radius $a$ to
	represent the electrons. Calculate the atomic polarisability of
	this atom.

	The net effect of this induces a dipole in the atom. We need to
	find the field on our displaced nucleus from the electron cloud
	due to the displacement. We can deduce that if the nucleus is
	displaced by $d$, then the sphere with
	radius $d$ will enclose a charge of
	$Q_{\mathrm{enc}}$ given by
	\begin{equation}
		Q_{\mathrm{enc}} = \left(\frac{4\pi d^3}{3} \div \frac{4\pi a^3}{3}\right) q =
		{\left(\frac{d}{a}\right)}^3 q
	\end{equation}

	Then, from Gauss' Law, we can calculate the electric field. We
	can then express $q d \hat{\bm{r}} = \bm{p}$ in terms of
	$\bm{E}$, whose proportional relationship gives
	us the atomic polarisability $\alpha =
		4\pi\epsilon_{0} a^3 = 3 \epsilon_{0}
		V_{\mathrm{atom}}$.
\end{example}

\section{Polarising molecules and crystals}%
\label{sec:polarising_molecules_and_crystals}

For atoms, polarisation is simple. However, for molecules, the
situation is more complicated.

We have chemical bonding (fundamentally electromagnetic
interactions on the atomic scale treated using quantum theory).

In this case we may find that the polarisation of the molecule
does not necessarily align with the direction of the field.
I.e., the relation between $\bm{E}$ and
$\bm{p}$ is a tensor relationship.
\begin{equation}
	\begin{pmatrix}
		p_{x} \\
		p_{y} \\
		p_{z}
	\end{pmatrix} =
	\begin{pmatrix}
		a_{xx} & a_{xy} & a_{xz} \\
		a_{yx} & a_{yy} & a_{yz} \\
		a_{zx} & a_{zy} & a_{zz}
	\end{pmatrix}
	\begin{pmatrix}
		E_{x} \\
		E_{y} \\
		E_{z}
	\end{pmatrix}
\end{equation}

Tensor relationships are common for real materials, especially
for molecules and crystals with low symmetry. When we have two
vector quantities that are related and it's not the magnitude
that changes, but also the direction, we need a tensor
relationship.

\section{Linear polarisation}%
\label{sec:linear_polarisation}

What happens when we place a material (many atoms) in an
electric field?

The simplest idea is to imagine that they all get polarised in
the field independently.

If a dipole is induced on each atom, the effect is that we have
a lot of dipoles aligned in the field. The material, as a whole,
becomes polarised. This combined effect is usually described as
a polarisation per \textit{unit volume}
$\bm{P}$.

If the material is made of molecules that already have a dipole
moment (water molecules), the net effect is the same, the
dipoles still align and we can still describe it in the same
way.

\textbf{Note:} There is always thermal energy in our
material. The dipoles, induced or not, are always wriggling
around. Inevitably, to understand $\bm{P}$ at a
fundamental level, we need to use statistical physics arguments.

\section{The field from a polarised object}%
\label{sec:the_field_from_a_polarised_object}

To find the field from the whole object we need to `sum' up the
effect of all these dipoles (the principle of superposition)
\begin{equation}
	V_{\mathrm{dipole}}\left(\bm{r}\right) =
	\frac{1}{4\pi\epsilon_{0}}\frac{\bm{p}\vdot\hat{\bm{r}}}{r^2}
\end{equation}

Let's try to derive an expression for the field from the
potential.

\begin{derivation}

	\textbf{Note:} A dipole moment from a volume
	in our material may be written as $\bm{p} =
		\bm{P}\dd{\tau^\prime}$ if we
	assume a continuous distribution. Then to get a whole material,
	we can integrate.
	\begin{equation}
		V\left(\bm{r}\right) = \frac{1}{4\pi\epsilon_{0}}
		\int\limits_{V}
		\frac{\bm{P}\left(\bm{r}^\prime\right) \vdot
			\hat{\bm{r}}}{r^2}\dd{\tau^\prime}
	\end{equation}

	But we notice that
	\begin{equation}
		\gradient^\prime\left(\frac{1}{r}\right)
		=
		\frac{\hat{\bm{r}}}{r^2}
	\end{equation}

	So we can write this as
	\begin{equation}
		V\left(\bm{r}\right) = \frac{1}{4\pi\epsilon_{0}}
		\int\limits_{V}
		\bm{P}\left(\bm{r}^\prime\right)\vdot
		\gradient^\prime\left(\frac{1}{r}\right)\dd{\tau^\prime}
	\end{equation}

	If we integrate by parts, and then apply the divergence theorem,
	we can rewrite this integral as a surface integral.
	\begin{equation}
		\begin{aligned}
			V & = \frac{1}{4\pi\epsilon_{0}}\left[\int\limits_V\gradient^\prime
				\vdot\left(\frac{\bm{P}}{r}\right)\dd{\tau^\prime} -
				\int\limits_V\frac{1}{r}
			\left(\gradient^\prime\vdot\bm{P}\right)\dd{\tau^\prime} \right]       \\
			  & = \frac{1}{4\pi\epsilon_{0}}\left[\oint\limits_S\frac{1}{r} \bm{P}
				\vdot\dd{\bm{a}^\prime} - \int\limits_V
				\frac{1}{r}\left(\gradient^\prime\vdot\bm{P} \right)\dd{\tau^\prime}\right]
		\end{aligned}
	\end{equation}
\end{derivation}

\begin{definition}
	The electric field from a polarised object is
	\begin{equation}
		V\left(r\right) = \frac{1}{4\pi\epsilon_{0}}
		\left[\oint\limits_S\frac{\sigma_b}{r}\dd{a^\prime} +
			\int\limits_V\frac{\rho_b}{r}\dd{\tau^\prime}\right]
	\end{equation}
	where $\sigma_b\equiv\bm{P}\vdot
		\hat{\bm{n}}$ is the `bound' surface charge and
	$\rho_b\equiv - \divergence{\bm{P}}$ is the `bound' charge density.
\end{definition}

If we imagine an object that has aligned dipoles, then there
will be surface charges, opposite sides will have opposite
charges, thus we have an electric field.

\textbf{Note:} Bigger dipoles mean
$\rho\left(r\right)$ is not constant, so
$\divergence{\bm{P}}\neq0$.

\section{The electric field displacement D}%
\label{sec:the_electric_field_displacement_d}

Consider the charge density within a material as due to the
bound charges, $\rho_b$, that we described in
Section~\ref{sec:the_field_from_a_polarised_object}, and that due to anything else
(like free electrons), $\rho_f$.

Now, Gauss' law applied to both charge densities (the total
charge density) $\rho = \rho_b + \rho_f$ gives
\begin{equation}
	\epsilon_{0}\divergence{\bm{E}}
	=
	\rho
	=
	\rho_b + \rho_f
	=
	-\divergence{\bm{P}} + \rho_f
\end{equation}
where $\bm{E}$ is the \textit{total electric field}
from both dipoles and free charges.

If we rearrange the left-most and right-most side, we get
\begin{equation}
	\divergence{\left(\epsilon_{0}\bm{E} + \bm{P}\right)}
	=
	\rho_f
\end{equation}

Let us define a new quantity $\bm{D}$.
\begin{definition}
	The displacement field $\bm{D}$ is
	\begin{equation}
		\bm{D} =
		\epsilon_{0}\bm{E} +
		\bm{P}
	\end{equation}
\end{definition}

Thus, as a result, we have derived the Gauss' law, in
differential and integral forms, for the $\bm{D}$
field.

\begin{definition}
	Gauss' law for the $\bm{D}$ field in both forms
	are
	\begin{equation}
		\divergence{\bm{D}} = \rho_f
	\end{equation}
	and
	\begin{equation}
		\oint\bm{D}\vdot\dd{\bm{a}}
		=
		Q_{\mathrm{free}}
	\end{equation}

	\textbf{Note:} We cannot assume that
	$\curl{\bm{D}} = 0$. So we cannot assume
	$\bm{D}$ is determined exclusively by the free
	charge. Normally, for high symmetry, the curl might be zero. As
	a consequence, there is also no scalar potential for
	$\bm{D}$.
\end{definition}

\begin{example}
	A long straight wire, with line charge $\lambda$,
	is surrounded by rubber insulation out to a radius
	$a$. Find the electric displacement.

	We write out the divergence relationship between
	$\bm{D}$ and $\rho_f$. Then we use
	Gauss' Law, and consider a smaller cylinder of radius
	$s$ and length $L$ inside
	our insulated wire as our Gaussian surface, ignoring the fields
	coming from the ends.

	Then, the field is simply $\bm{D}2\pi s L = \lambda L$, the rearrange
	for $\bm{D}$, keeping in mind it is a vector
	pointing radially away from the axis of the cylinder.

	Outside the rubber, our field is given by
	$\bm{D} = \epsilon_{0}
		\bm{E} + \bm{P}$, but $\bm{P} = 0$ outside the
	rubber. For regions inside the rubber, we cannot determine
	$\bm{D}$ as we don't know $\bm{P}$.
\end{example}

\section{Boundary conditions for the displacement field}%
\label{sec:boundary_conditions_for_the_displacement_field}

\begin{definition}
	We establish the boundary conditions for the field
	$\bm{D}$ as follows
	\begin{equation}
		\label{eq:d_perp}
		D^\perp_{\mathrm{above}} - D^\perp_{\mathrm{below}} = \sigma_f
	\end{equation}
	and
	\begin{equation}
		\label{eq:d_parallel}
		\bm{D}^\parallel_{\mathrm{above}} -
		\bm{D}^\parallel_{\mathrm{below}}
		=
		\bm{P}^\parallel_{\mathrm{above}} -
		\bm{P}^\parallel_{\mathrm{below}}
	\end{equation}
	Note the discontinuity in the field above and below a surface
	and that there is also a discontinuity in the parallel field,
	unlike the electric field.
\end{definition}

\begin{derivation}
	Derive the boundary conditions for $\bm{D}$.

	Consider a sheet with a box that bisects the surface, where the
	top half is above the surface, and the bottom half below the
	surface. If we make our Gaussian surface close to the box of
	charge, then we can use Gauss' Law for our displacement vector.
	We deduce that $D_{\mathrm{above}} = \frac{\sigma}{2}$ and
	$D_{\mathrm{below}} = -\frac{\sigma}{2}$. We can rearrange to get
	Equation~\ref{eq:d_perp}.

	To get the expressions for the parallel faces, we consider the
	side parallel vectors tending to zero as the box gets thinner
	and thus, our line integral around the box is just for the
	parallel faces above and below. Write out the equation for
	$\bm{D}$, and taking the curl of both sides, then
	we can use Stoke's theorem to get
	equation~\ref{eq:d_parallel}.
\end{derivation}
\section{Linear dielectrics}%
\label{sec:linear_dielectrics}

If we did not know abut atoms, we could still make the
observation that the net polarisation is proportional to the
field.
\begin{equation}
	\bm{P}
	=
	\epsilon_{0}\chi_e\bm{E}
\end{equation}
where $\chi_e$ is known as the electric
susceptibility and is an example of \textit{linear response}.
Hence, we call materials with this behaviour
\textit{linear dielectrics}.

This is not always the case (like in ferroelectrics), but it is
truer for most materials under \textit{small} fields.

\textbf{Note:} In some low symmetry crystals,
$\bm{P}$ and $\bm{E}$ may not lie
in the same direction and we need to define a susceptibility
tensor to establish the relationship between the two fields even
if the relationship is linear.

\begin{definition}
	If we do this, we can generate some relationship
	\begin{equation}
		\label{eq:d_e_field_relation}
		\bm{D} = \epsilon_{0}
		\bm{E} + \bm{P}
		=
		\epsilon_{0}\bm{E}	+
		\epsilon_{0}\chi_e\bm{E}
		=
		\epsilon_{0}\left(1 + \chi_e\right)
		\bm{E}
		=
		\epsilon\bm{E}
	\end{equation}
	where $\epsilon = \left(1 + \chi_e\right)$ is the \textit{permittivity} of
	the material.

	The \textit{relative permittivity} is
	\begin{equation}
		\epsilon_r
		=
		1 + \chi_e = \frac{\epsilon}{\epsilon_{0}}
	\end{equation}
\end{definition}

\begin{example}
	Examples from videos
	\begin{enumerate}
		\item A metal sphere of radius $a$ carries a
		      charge $Q$ and is surrounded by a linear
		      dielectric of permittivity $\epsilon$ out to a
		      radius $b$. Find the potential at the centre
		      of the sphere assuming the potential is zero at infinity.
		      \begin{itemize}
			      \item To find the potential, we need to know the electric field. We
			            use a Gaussian surface $r > a$ and Gauss' law for
			            the $\bm{D}$ field. Our field inside the
			            dielectric uses the permittivity $\epsilon$ instead
			            of $\epsilon_{0}$.
			      \item Write out the potential for each region from the field derived
			            for each region as integrals so that it is in effect the
			            integral of the field from infinity to 0.
		      \end{itemize}
		\item A parallel plate capacitor is filled with a dielectric with a
		      dielectric constant $\epsilon_{r}$. What is the change
		      in capacitance?

		      \begin{itemize}
			      \item We know that $E = V/d$, we can derive that the
			            capacitance is $C = A\epsilon_{0}/d$
			      \item If our gap is filled with a dielectric instead, we can use the
			            $\bm{D}$ field, giving the relationship
			            $\bm{D} =
				            \epsilon_{0}\epsilon_{r}
				            \bm{E} = \sigma$.
			      \item From $C = Q/V$, we can rewrite using
			            $V$ derived from the relationship for
			            $\bm{E}$ above.
		      \end{itemize}
	\end{enumerate}
\end{example}

\section{Comments on the differences between E and D}%
\label{sec:comments_on_the_differences_between_e_and_d}

Why do we choose to use $\bm{D}$ rather than
$\bm{E}$?
\begin{itemize}
	\item If we wish to use $\bm{E}$ in the presence of a
	      dielectric we would need to determine the bound surface and bulk
	      charge distributions. By defining $\bm{D}$ we
	      only need to consider the free charges in the system---albeit at
	      the expense of not knowing details about what is happening in
	      the dielectric (in effect we talk about an average macroscopic
	      field in the dielectric).
	\item As we have the relationship $\divergence{\bm{D}} = \rho_{\mathrm{free}}$ and
	      $\oiint\bm{D}\dd{\bm{a}} =
		      Q_{\mathrm{free}}$ we can use similar techniques to finding
	      $\bm{D}$ as we did for $\bm{E}$
	      (using Gaussian surfaces etc.).
	\item However, we cannot write
	      \begin{equation}
		      \bm{D}\left(r\right)
		      =
		      \frac{1}{4\pi}
		      \int\frac{\hat{\bm{r}}^{\prime\prime}}{r^{\prime\prime 2}}
		      \rho_{\mathrm{free}}\left(\bm{r^\prime}\right)
		      \dd{\tau^\prime}
	      \end{equation}
	      in other words, there is no Coulomb's law equivalent for the
	      free charge.
	\item For example, in the case of the electric field we always have
	      $\curl{\bm{E}} = 0$, from which it follows
	      $\bm{E}$ is a conservative field meaning we can
	      express it as the gradient of a scalar field
	      $-\gradient{V}$. However, for the
	      $\bm{D}$ field we have $\curl{\bm{D}} =
		      \epsilon_{0}\left(\curl{\bm{E}}\right) +
		      \curl{\bm{P}} =
		      \curl{\bm{P}}$.
	      It is not always the case that $\curl{\bm{P}} = 0$. For
	      example when there is some permanent polarisation as for example
	      in an electret.
\end{itemize}

As a general rule, if our system has high symmetry we may still
apply Gauss law as previously. But, if for example the spherical
symmetry is broken by a permanent alignment of dipoles such that
$\curl{\bm{P}}\neq 0$ we cannot just assume
$\bm{D}$ is determined solely by the free charge.

\section{Energy in dielectrics}%
\label{sec:energy_in_dielectrics}

The energy needed to charge a capacitor is
\begin{equation}
	W = \frac{1}{2} CV^2
\end{equation}
With a dielectric, we have
\begin{equation}
	C = \epsilon_r C_{\mathrm{vac}}
\end{equation}
The work necessary to charge a capacitor increases because the
bound charges cancelled off part of the field, so we have to
move more free charges to achieve the same potential. In other
words, by adding a dielectric in between the capacitor we
increase the energy that can be stored for the same amount of
potential difference.

If we can make $\epsilon_r$ very high and use high
voltages (so there is no breakdown in the material) we can make
a `supercapacitor'.

It may also be shown that
\begin{equation}
	\label{eq:work_done_dielectric}
	W = \frac{1}{2}\int\bm{D}\vdot
	\bm{E}\dd{\tau}
\end{equation}

Remember that in free space, work is given
by~\ref{eq:work_continuous}.

We could ask why we can't use the free space definition and
imagine we bring all the charges together one by one. This would
be OK but it doesn't take into account any of the energy needed
in stretching the atoms in the dielectric in order to polarise
them.

If we imagine we started with an unpolarised dielectric and then
bring in the free charges to their final positions then the
dielectric will respond by the charges displacing to form the
induced dipoles. This extra energy is then taken into account in
the work done. This quantity is encapsulated in the
$\frac{1}{2}\int\bm{D}
	\vdot\bm{E}\dd{\tau}$. It is larger than just doing the
calculation using the free space equation.

\begin{example}
	A sphere of radius $R$ is filled with
	material of dielectric constant $\epsilon_r$ and
	uniform embedded free charge $\rho_{\mathrm{free}}$. What is the
	energy of the configuration?

	Which version of the integration of work should we
	use,~\ref{eq:work_done_dielectric} or~\ref{eq:work_continuous}? We find
	the $\bm{D}$ field using its Gauss' law, and then
	try both equations.

	We work out the total free charge from the geometry, apply
	Gauss' law for a spherical shell as the Gaussian surface, and
	with the $\bm{D}$ field, we can work out
	$\bm{E}$ using~\ref{eq:d_e_field_relation}. With both
	these fields, we can find their dot product and integrate this
	over the entire volumetric sphere to get the work done.

	We find that work done using the $\bm{E}$ field
	only is smaller than the other equation. This is because they
	are defining slightly different things. The first considers a
	dielectric sphere with separated surface and volume bound
	charges, fixed them in space and find the energy of the electric
	field. We let the system relax and find the electric field at
	some point in the system.

	The second equation involves starting off with a dielectric and
	then we introduce the free charge slowly. As we introduce the
	free charge, this gives us a separation in terms of bound
	charges on the dielectric itself \textit{and} also
	the energy in stretching the molecules out to form the dipoles.

	Here, we have extra energy when we include the
	$\bm{D}$ field, this is not captured if we just
	take the charge distributions as they were.
\end{example}

\section{Forces on dielectric}%
\label{sec:forces_on_dielectric}

As with conductors a dielectric is drawn into an electric field.
This is best understood by considering a simple parallel plate
capacitor into which a dielectric field is partially inserted.
See Figure~\ref{fig:dielectric_in_capacitor}.

\begin{figure}[htpb]
	\centering
	\includesvg[width=0.9\columnwidth]{./dielectric_in_cap.svg}
	\caption{A partially inserted dielectric in a capacitor.}%
	\label{fig:dielectric_in_capacitor}
\end{figure}

In our ideal system there would be no force as the field is
perfectly perpendicular to the plates. In reality, it is the
small fringing field that allows force to be exerted.

In air, the capacitance is given by
\begin{equation}
	C = \frac{\epsilon_{0}A}{d}
\end{equation}
while with a dielectric, it is
\begin{equation}
	C = \frac{\epsilon_{0}\epsilon_r A}{d}
\end{equation}

This fringing field at the edges gives rise to the forces which
exerts a force on the dielectric.

In order to pull the dielectric slab out, we need to exert a
force $F_m$ on the dielectric, so we do the
total amount of work
\begin{equation}
	\dd{W}
	=
	F_m \dd{x}
\end{equation}

Then, the electrical force on the dielectric is
$F = -F_m$. Thus, the external force due to the
electric field on the slab is
\begin{equation}
	F = -\frac{\dd{W}}{\dd{x}}
\end{equation}

Since the energy stored in the capacitor is
$E = CV^2/2$, a partially filed capacitor would have
a partial area (total capacitance depends on the total
dielectric in the capacitors), and thus total energy
\begin{equation}
	C = \frac{\epsilon_{0} w}{d}\left(\epsilon_r l - \chi_e x\right)
\end{equation}

We can understand this result. The proportion of the capacitor
that is still in air is given by the area
$A_a = xw$, and the proportion that is dielectric
is given by the area $A_d = \left(l - x\right)w$, thus, we sum up
the capacitance for each region. We recall that
$\chi_e = 1 - \epsilon_r$.

For a capacitor that is already charged up
$W = Q^2/2C$, then
\begin{equation}
	F = -\dv{W}{x} =
	\frac{1}{2}\frac{Q^2}{C^2}\dv{C}{x}\frac{1}{2}V^2\dv{C}{x}
\end{equation}
\textbf{Note:} we are keeping $Q$
as a constant here. If we keep $V$ constant,
there would be work from the battery used in changing
$Q$. In this derivation,
$V$ changes. The derivative of
$C$ w.r.t. $x$ is
\begin{equation}
	\dv{C}{x} =
	-\frac{\epsilon_{0} \chi_e w}{d}
\end{equation}

\begin{definition}
	The force on a dielectric due to an external electric field is
	\begin{equation}
		F =  -\frac{\epsilon_{0} \chi_e w}{d}V^2
	\end{equation}
\end{definition}

We can also get the same result if we take
$V$ to be constant (maintained by the
battery), and $Q$ will change as the
dielectric moves.

\chapter{Magnetostatics}%
\label{cha:magnetostatics}

\section{Introduction to magnetostatics}%
\label{sec:introduction_to_magnetostatics}

So far, we have only considered fixed charges. When charge
starts to move, it is subject to an electrostatic force. We have
also magically conjured up charge distributions to solve
problems without considering how the charges may have arrived in
the first place.

The way in which moving charges create time varying fields
cannot be considered until we get to Maxwell's equation. The
consequences of Maxwell's equations will be considered in later
courses.

In this section we will restrict ourselves to the case of
electrical currents (mostly conductors) that as you know will
give rise to magnetic fields. Moreover, we will for the most
part restrict ourselves to steady currents, that will give rise
to static magnetic fields---i.e. \textit{magnetostatics}.

We can imagine a situation where a battery sends a current
through a very long wire set apart by a distance. Suppose the
parallel parts are very long so we can ignore the ends. Current
flows up one wire and flows down the other wire in the other
direction.

We see empirically that the wire tends to push each other apart.
We may conclude there are electrostatic charges that are
repulsing each other. But if we bring a test charge, there is no
effect on the test charge.

We can also alter the wire so that both currents are now facing
the same direction. Now, the wires are attracted to each other.

We conclude then, that moving charges generate magnetic fields.
Magnetic fields can be detected by using a compass needle. The
direction of the magnetic field lines is determined by the
right-hand screw rule.

The magnetic field is always perpendicular to the direction of
current flow. We also find that the force exerted is
perpendicular to both the magnetic field vector and the
direction of the current.

\section{The Lorentz force law}%
\label{sec:the_lorentz_force_law}

\begin{definition}

	The force produced by a charge $Q$ moving
	with velocity $\bm{v}$ in a magnetic field
	$\bm{B}$ is expressed as
	\begin{equation}
		\bm{F_{\mathrm{mag}}}
		=
		Q\left(\bm{v}\cross\bm{B}\right)
	\end{equation}

	In the presence of an electric field $\bm{E}$,
	the total charge becomes
	\begin{equation}
		\bm{F_{\mathrm{total}}}
		=
		Q\left(\bm{E} + \bm{v}\cross\bm{B}\right)
	\end{equation}
\end{definition}

\textbf{Note:} The use of the cross product removes
the need for remembering the right and left-hand rules. If we
remember the Lorentz Law, the direction of the force is given by
the cross-product convention.

The Lorentz force law is a result of careful observation. It is
a fundamental axiom of the theory.

\begin{example}
	Examples from video lectures
	\begin{enumerate}
		\item Consider the case of a particle moving in a magnetic field. Let
		      $\bm{B}$ act downwards into the page. Consider a
		      particle of charge $Q$ moving at a velocity
		      $\vec{v}$. We can find out the Lorentz force on
		      the particle. The force is always constant in magnitude and its
		      direction is given by the cross product $\bm{b}\cross\bm{B}$.
		      This gives us circular motion. We can equate equation for
		      circular motion with the Lorentz force to find the radius. We
		      find that the radius is $R = p/QB$. The direction
		      of the curve is indicative of the charge and the radius of
		      curvature tells the strength of the momentum. The path it takes
		      is a helical shape.
		\item Cycloid motion in $\bm{E}$ and
		      $\bm{B}$ field. What happens to a particle which
		      is moving in the $y-z$ plane under the
		      influence of both electrostatic and magnetic fields? Again, we
		      can use the Lorentz force law. We find that as the particle gets
		      deflected due to the $\bm{B}$ field, the
		      $\bm{E}$ field resists this motion. It appears to
		      oscillate in direction.

		      Here, we are being careful by writing out the cross product of
		      $\bm{v}\cross\bm{B}$ in full, equate this to Newton's second
		      law, and compare coefficients of each component separately. What
		      we get are differential equations that link the motion of the
		      particle, but they are coupled. To solve this, we would have to
		      decouple them, we can do this by isolating
		      $y$. The general solution with boundary
		      conditions is
		      \begin{equation}
			      y\left(t\right) = \frac{E}{\omega B}
			      \left(\omega t - \sin\omega t \right)
		      \end{equation}
		      \begin{equation}
			      z\left(t\right) = \frac{E}{\omega B}
			      \left(1 - \cos{\omega t}\right)
		      \end{equation}
		      Since $R\equiv E/B\omega$,
		      \begin{equation}
			      {\left(y - R \omega t\right)}^2 + {\left(z - R \right)}^2 = R^2
		      \end{equation}
		      This is a circle of radius $R$ with a centre
		      at $\left(0, R\omega t, R \right)$. The centre moves with a velocity
		      $u = \omega R = E/B$. This means the particle is moving as if
		      it is a spot on the rim of the wheel rolling on the
		      $x$-axis. This is cycloid motion.

		      The most important thing is breaking down the motion by
		      understanding how the electric field slows down the motion and
		      deflection of the magnetic field, and this motion repeats.
	\end{enumerate}
\end{example}

\section{The work done by magnetic fields}%
\label{sec:the_work_done_by_magnetic_fields}

What is the work done on a charge when a magnetic force acts
upon it?
\begin{equation}
	\dd{W_{\mathrm{mag}}}
	=
	\bm{F}_{\mathrm{mag}}\vdot
	\dd{\bm{l}}
	=
	Q\left(\bm{v}\cross\bm{B}\right)\vdot\bm{v}
	\dd{t} = 0
\end{equation}

Since $\bm{v}\cross\bm{B}$ is by definition perpendicular to
$\bm{v}$ and $\bm{B}$ unless they
are co-linear, or zero, in which case there is no force. Hence,
the dot product is zero.

The consequence is that \textbf{magnetic forces do no work}.

This is not uncommon. In a gyroscope, gravity acts to change the
direction of the angular momentum vector but does not work.
Similarly, central forces do no work.

\begin{example}
	Show that the magnetic force do no work.

	We consider a loop of wire carrying a current sitting in an area
	where there's a magnetic field, which points into the page. At
	the bottom, there is a mass with weight $m\bm{g}$
	What current do we need to balance the weight?

	We consider the force along the wire given by
	Equation~\ref{eq:force_along_wire}. The force on the vertical
	sections cancel and we get only the horizontal section width
	$a$. Thus, the magnitude of the force is
	$F = IBa$. Balancing the force gives us
	$I = mg/Ba$. If we increase the current, we expect
	the mass to be lifted a height $h$.

	With this, we can write that the work done by the magnetic field
	is $W = IBah$. But this clearly contradicts our
	claim. What truly happens when the wire rises? As the wire
	rises, the charges carrying the current gain a vertical velocity
	$\bm{u}$, in addition to its horizontal velocity
	$\bm{w}$. Thus, the Lorentz force from this is
	pointing slightly backwards as the drift direction is slightly
	diagonal.

	If we split this force into two components,
	$qwB$ vertically and $quB$
	horizontally. This horizontal force is acting to slow the
	carriers down. Without assistance, the charge carriers will slow
	down if the wires rise, and thus our current will decrease.
	Thus, in actuality, we have to put some work in to maintain the
	current as the wire rises. The work we need to put in comes from
	whatever is driving the current $W_\mathrm{battery} = \lambda a B \int u \vdot w dt = IBah$.

	In conclusion, for this movement to take place, the battery,
	\textit{not} the magnetic field, does work.
\end{example}

\section{Currents and line current density}%
\label{sec:currents}

We usually consider a current in a wire and define it as the
amount of charge \textit{passing per unit time}  at a point in the
wire.

We can see how this will be proportional to the velocity
$\bm{v}$ of the charge carriers in the wire and
charge $q$.

\textbf{Note:} We now understand current to be the
movement of negatively charged electrons in our wires. The
electrons move in the opposite way to the conventional
definition of current. In the development of the theory of
magnetism the current was assumed to be the movement of positive
charges and this has remained.

We all accept and have a conceptual idea of a wire. In practice,
unless we say otherwise, we understand a wire as the object that
guides the charge carriers but with no physical size of its own.

Along a wire, the current at a particular point P may be thought
of as a line charge $\lambda$ per unit length
moving along a wire with velocity $\bm{v}$.
\begin{equation}
	\bm{I} = \lambda\bm{v}
\end{equation}

The Lorentz force on a segment of wire is
\begin{equation}
	\bm{F} = \int
	\left(\bm{v}\cross\bm{B}\right)\dd{q}
	=
	\int\left(\bm{v}\cross\bm{B}\right)\lambda\dd{l}
	=
	\int\left(\bm{I}\cross\bm{B}\right)\dd{l}
\end{equation}

Because the current is constrained to flow along the wire we can
write
\begin{equation}
	\bm{I}\dd{l} = I
	\dd{\bm{l}}
\end{equation}

\begin{definition}
	The force along a wire with a current is
	\begin{equation}
		\label{eq:force_along_wire}
		\bm{F} = \int I \left(\dd{\bm{l}}\cross\bm{B}\right)
	\end{equation}
\end{definition}

\begin{example}
	Example video lectures: A uniform current is flowing through a
	cylinder of area $A$. We get
	$J = I/\pi a^2$. Now, if our current is non-uniform, for
	example $J = ks$, what is the current? We would
	need to integrate for the answer.
	\begin{equation}
		\dd{I} = J\dd{a_\perp} = J s
		\dd{s} \dd{\phi} = ks\vdot
		s\dd{s}\dd{\phi}
	\end{equation}

	The answer is $I = 2\pi ka^3 / 3$.
\end{example}

\section{Surface and Volume current densities}%
\label{sec:surface_and_volume_current_densities}

\subsection{Surface current density}%
\label{sub:surface_current_density}

Currents may not necessarily be thought of as flowing through
very thin wires. They may be considered as flowing over a
surface or through the bulk of a material (very thick cable).
Furthermore, the current may not flow uniformly over the surface
on in the bulk.
\begin{equation}
	\bm{K}\equiv\dv{\bm{I}}{l_\perp}
\end{equation}
where $\dd{l}$ is perpendicular to the current
flow, this is the current per unit length flowing through the
surface.

Alternatively
\begin{equation}
	\bm{K}
	=
	\sigma\bm{v}
\end{equation}
where $\sigma$ is the `moving' surface charge
density. The number of carriers move through the surface, but
$\sigma$ doesn't change. Thus, for a surface,
\begin{equation}
	\bm{F} = \int\left(\bm{v}\cross\bm{B}\right)\sigma
	\dd{a}
	=
	\left(\bm{K}\cross\bm{B}\right)\dd{a}
\end{equation}

\subsection{Volume current density}%
\label{sub:volume_current_density}

Similarly, we may define a volume current density
\begin{equation}
	J \equiv\dv{\bm{I}}{a_\perp}
\end{equation}
where $\dd{a_\perp}$ is a surface area perpendicular to
the current flow at the point in the surface. Then we have
\begin{equation}
	\bm{J}
	=
	\rho\bm{v}
\end{equation}
where $\rho$ is the mobile charge density.
Similarly,
\begin{equation}
	\bm{F} = \int
	\left(\bm{J}\cross\bm{B}\right)\dd{\tau}
\end{equation}

\section{The continuity equation}%
\label{sec:the_continuity_equation}

From the definition of volume current density
in~\ref{sec:surface_and_volume_current_densities} we can write
\begin{equation}
	I = \int\limits_{S}\bm{J}\vdot
	\dd{\bm{a}}
\end{equation}

Now, from the divergence theorem we have
\begin{equation}
	\oint\limits_{S}\bm{J}\vdot
	d\dd{\bm{a}} =
	\iiint\limits_{V}\divergence{bm{J}}\dd{\tau}
\end{equation}

The left-hand term corresponds to the charge per unit time
leaving the enclosed surface and the right-hand term is the
corresponding charge leaving the volume per unit time. Now, as
charge is conserved this must represent the change in the charge
density in the volume per unit time, the minus sign represents
the charge density decreasing as the charge flows out.
\begin{equation}
	\iiint\limits_V\left(\divergence{\bm{J}}\right)\dd{\tau}
	=
	-\dv{}{t}\iiint\limits_V\rho
	\dd{\tau}
	=
	-\iiint_V \left(\dv{\rho}{t}\right)\dd{\tau}
\end{equation}

If we equate the kernels of the two integrals, then we get the
continuity equation.
\begin{definition}
	The continuity equation is
	\begin{equation}
		\divergence{J} = -\pdv{\rho}{t}
	\end{equation}

	If $\rho$ is constant, then we have a steady
	state condition.
\end{definition}

For electrostatics, we must have no change in
$\rho$---electrostatic problems are steady
state problems.

\section{The Biot-Savart Law}%
\label{sec:the_biot_savart_law}

Magnetostatics refers to the regime where magnetic fields do not
vary with time. Consequently, the currents producing these
fields must be steady. Hence, we have
\begin{equation}
	\pdv{\rho}{t} = 0\hspace{1em}
	\mathrm{and}\hspace{1em}\pdv{\bm{J}}{t}
	= 0
\end{equation}

We don't ask how we get to this steady state but consider what
happens when we are there.

\textbf{Note:} by these definitions a moving point
charge does not give a steady current---we need a charge
distribution.

If we have steady currents then from the continuity equation and
$\pdv{\rho}{t} = 0$
\begin{equation}
	\divergence{J} = 0
\end{equation}

\begin{definition}
	Putting this all together, we have the equivalent of the
	integral form of Coulombs law the Biot-Savart Law.
	\begin{equation}
		\label{eq:the_biot_savart_law}
		\bm{B}\left(\bm{r}\right)
		=
		\frac{\mu_0}{4\pi}\int\frac{\bm{I}\cross\bm{r^{\prime\prime}}	}{r^{\prime\prime 2}}
		\dd{l^\prime}
		=
		\frac{\mu_0 I}{4\pi}\int\frac{\dd{\bm{l}}\cross \bm{r^{\prime\prime}}}{r^{\prime\prime 2}}
	\end{equation}

	Where $\mu_0$ is the permeability of free space.
\end{definition}

The superposition principle also applies to
$\bm{B}$. It is measured in tesla
($T$).

\begin{example}
	Examples from the video lectures:
	\begin{enumerate}
		\item Find the magnetic field at a distance $s$
		      from a long straight wire that carries a steady current
		      $I$.

		      Let the point $P$ be at a distance
		      $s$ from the `centre' of the wire, and let a
		      segment $\dd{l^\prime}$ on a wire be at a distance
		      $\bm{r^{\prime\prime}}$. Let also the length of the wire be
		      $2 L$.

		      The cross product in the kernel of the Biot-Savart can be
		      written as
		      \begin{equation}
			      \dd{\bm{l^\prime}}\cross\bm{\hat{r}^{\prime\prime}}
			      =
			      \dd{l^\prime}\vdot\absolutevalue{\bm{\hat{r}^{\prime\prime}}}
			      \sin{\alpha}
		      \end{equation}
		      Where $\alpha$ is the angle between
		      $r^{\prime\prime}$ and the axis of the wire. With some
		      trigonometry and algebra, we can rewrite $\sin\alpha$
		      and $r^{\prime\prime 2}$. This comes to
		      \begin{equation}
			      \bm{B}\left(\bm{r}\right)
			      =
			      \frac{\mu_0 I s}{4\pi}\int^L_{-L}\frac{\dd{l}}{{s^2 + l^2}^{\frac{3}{2}}}
		      \end{equation}
		      This evaluates to
		      \begin{equation}
			      \bm{B}\left(\bm{r}\right)
			      =
			      \frac{\mu_0 I}{2\pi s}\vdot\frac{L}{\sqrt{s^2+L^2}}
		      \end{equation}
		      Let $L \rightarrow \infty$ to get
		      \begin{equation}
			      \bm{B}\left(\bm{r}\right)
			      =
			      \frac{\mu_0 I}{2\pi s}\bm{\hat{r}}
		      \end{equation}

		\item Find the magnetic field a distance $z$ above
		      a circular wire of radius $R$ that carries a
		      current $I$.

		      This is similar to the previous setup, however, we work in
		      polar/cylindrical system instead due to the geometry of the
		      circular wire. Like how we derived the magnetic field of the
		      circular ring, components in the direction that changes as you
		      move along the circle will cancel.

		      We then apply the Biot-Savart law and integrate around the
		      circle. The result is
		      \begin{equation}
			      B_z = \frac{\mu_0 I R^2}{2 {\left(R^2 + Z^2 \right)}^{\frac{3}{2}}}
		      \end{equation}

	\end{enumerate}

\end{example}

\section{The divergence and curl of B}%
\label{sec:the_divergence_and_curl_of_b}

With the result for $\bm{B}$ for a long straight
wire, we can write
\begin{equation}
	\oint\bm{B}\dd{\bm{l}}
	=
	\oint\frac{\mu_0 I}{2\pi s}\dd{l} =
	\frac{\mu_0 I}{2\pi}\oint\dd{l} = \mu_0 I
\end{equation}
where we have taken a circular path of radius
$s$ (the result holds true for any closed
path enclosing the wire). We can add any number of wires within
the loop so that we have Ampère's law:
\begin{equation}
	\label{eq:amperes_law}
	\oint\bm{B}\vdot\dd{\bm{l}}
	=
	\mu_0 I_\mathrm{enc}.
\end{equation}

In terms of the current density $\bm{J}$ we may
express the enclosed current as
\begin{equation}
	I_{\mathrm{enc}} =
	\iint\bm{J}\vdot\dd{\bm{a}}
\end{equation}
where the integral is over the area enclosed by the loop. We may
apply Stokes' theorem to get
\begin{equation}
	\mu_{0}\iint\bm{J}\vdot\dd{\bm{a}}
	=
	\iint\left(\curl{\bm{B}}\right)\vdot\dd{\bm{a}}
\end{equation}
from which we find
\begin{definition}
	The curl of the magnetic field is given by Ampère's law in
	differential form
	\begin{equation}
		\label{eq:amperes_law_differential}
		\curl{\bm{B}}
		=
		\mu_{0}\bm{J}
	\end{equation}
\end{definition}

The curl of the magnetic is not zero, and thus it is not a
conservative field, but magnetic forces do no work.

This proof is only for a long straight wire. However, this is
indeed the general result. We can quote another result that may
be obtained from the Biot-Savart law.
\begin{definition}
	The divergence of the magnetic field is
	\begin{equation}
		\label{eq:magnetic_field_divergence}
		\divergence{B} = 0
	\end{equation}
	the physical meaning of this is that there are no magnetic
	monopoles.
\end{definition}

\begin{example}
	Use Ampère's law to find the magnetic field in a very long
	solenoid, that has $n$ turns per unit length
	on a cylinder of radius $R$, that carries a
	steady current $I$.

	Consider a long solenoid as described, the field outside is
	zero. Thus, we are interested in what's inside. We expect there
	is a net magnetic field going through the coil.

	By drawing some Ampèrian loops, we can see how the magnetic
	field is with respect to the number of loops when we apply
	Ampère's law.

	The result is a field that is uniform inside the solenoid
	\begin{equation}
		B_{\mathrm{inside}} = \mu_0 n I \bm{k}.
	\end{equation}
\end{example}

\section{Maxwell's equation for electrostatics and magnetostatics}%
\label{sec:maxwell_s_equation_for_electrostatics_and_magnetostatics}

\begin{definition}
	This is a summary of all differential forms
	\begin{equation}
		\label{eq:maxwells_equation}
		\begin{aligned}
			 & \divergence{\bm{E}} = \frac{\rho}{\epsilon_{0}} \\
			 & \curl{\bm{E}} = 0                               \\
			 & \divergence{\bm{B}} = 0                         \\
			 & \curl{\bm{B}} = \mu_0\bm{J}
		\end{aligned}
	\end{equation}

	Along with the Lorentz force law, these constitute the most
	concise formulation of electrostatics and magnetostatics.
	\begin{equation}
		\bm{F}
		=
		q\left(\bm{E} + \bm{v}\cross\bm{B}\right)
	\end{equation}
\end{definition}

\section{The magnetic vector potential}%
\label{sec:the_magnetic_vector_potential}

In Section~\ref{sec:maxwell_s_equation_for_electrostatics_and_magnetostatics} we established the complete
description of electrostatic and magnetostatic fields
($\bm{E}$ and $\bm{B}$) and their
associated charge and current densities.

In our treatment of electrostatics we made extensive use of the
electrostatic potential, $V$, a scalar
field, and its relation to the $\bm{E}$ field
through $\bm{E} = -\gradient{V}$.

As $\curl{\bm{B}} = \mu_0\bm{J}$ we know that $\bm{B}$
cannot be a conservative field and hence we cannot define a
scalar potential as we did for the case of the
$\bm{E}$ field.

It is possible to define a magnetostatic scalar potential but it
may only be used under restricted circumstances when away from
currents and for paths not looping a current. We will not
consider it here.

However, we do have the condition $\divergence{B} = 0$ unlike
the electrostatic case.

What if we consider the magnetic field as the curl of a vector
function?
\begin{equation}
	\label{eq:magnetic_vector_potential_curl}
	\bm{B} = \curl{\bm{A}}
\end{equation}

Hence, we have
\begin{equation}
	\curl{B} = \curl{\left(\curl{\bm{A}}\right)} =
	\gradient{\left(\divergence{\bm{A}}\right)} - \laplacian{\bm{A}}
	= \mu_0\bm{J}
\end{equation}

When we defined the electrostatic potential,
$V$, we explained that we could add any
constant to it and the electric field would not change as the
gradient of a constant is zero.

In the case of $\bm{A}$, the equivalent statement
is that we can add any function to $A$
provide its curl is zero (it is the gradient of a scalar). We
can use this to eliminate the $\gradient{\left(\divergence{\bm{A}}\right)}$ above by
setting
\begin{equation}
	\label{eq:magnetic_vector_potential_divergence}
	\divergence{\bm{A}} = 0.
\end{equation}

\textbf{Note:} this is a convenience in the same way
we set $V = 0$ at infinity in most cases, but
this does not mean it is a necessity.

We can use this because we specify that $\bm{B} = \curl{\bm{A}}$,
but this doesn't say anything about the
\textit{divergence}. We are at a liberty to pick whatever
and zero is the simplest choice. As long as we can say that
$\divergence{\bm{A}}$ goes to zero at infinity, it is always
possible to make the vector potential \textit{divergenceless}.

Once we do this, we may write that
\begin{definition}
	The Laplacian an of the magnetic vector potential is
	\begin{equation}
		\label{eq:magnetic_vector_potential_laplacian}
		-\laplacian{\bm{A}} = \mu_0\bm{J}
	\end{equation}
\end{definition}

This is just Poisson's equation for each component of
$\bm{A}$.

Hence, in analogy to the charge density in Poisson's equation
for $V$, we can write
\begin{definition}
	The magnetic vector potential in integral form is written as
	\begin{equation}
		\label{eq:magnetic_vector_potential_integral}
		\bm{A}\left(\bm{r}\right)
		=
		\frac{\mu_0}{4\pi}\int\frac{\bm{J}\left(\bm{r^\prime}\right) }{r^{\prime\prime}}
		\dd{\tau^\prime}
	\end{equation}
	Similar relationships hold for line and surface current
	densities, i.e.,
	\begin{equation}
		\bm{A}\left(\bm{r}\right)
		=
		\frac{\mu_0}{4\pi}\int\frac{\bm{I}\left(\bm{r^\prime}\right) }{r^{\prime\prime}}
		\dd{l^\prime}
	\end{equation}
	and
	\begin{equation}
		\bm{A}\left(\bm{r}\right)
		=
		\frac{\mu_0}{4\pi}\int\frac{\bm{K}\left(\bm{r^\prime}\right) }{r^{\prime\prime}}\dd{a^\prime}
	\end{equation}
\end{definition}

\textbf{Note:} $\bm{A}$ is the magnetic
vector potential at position $\bm{r}$ from the
origin, $\bm{r^\prime}$ is the position of the current
element $\bm{J}\dd{\tau^\prime}$ and $r^{\prime\prime}$ is
the distance from the current element to the point
$\bm{r^\prime}$.

In magnetostatics, $\bm{A}$ is not useful as
$V$, the potential difference that allows
you to find the work done in moving charges from point to point
(magnetic fields do no work).

We still have to deal with the three components of the
$\bm{A}$ field so the calculations are fiddly.
Nevertheless, the calculation of the $\bm{A}$
potential using these integrals is still often easier than using
the Biot-Savart Law (no cross-product to deal with), and
$\bm{B}$  may always be found from
$\bm{B} = \curl{\bm{A}}$ at the end of the calculation.

\section{Summary of the important relations between B, A and J}%
\label{sec:summary_of_the_important_relations_between_b_a_and_j}

Relating $\bm{J}$ and $\bm{A}$:
Equation~\ref{eq:magnetic_vector_potential_laplacian} and~\ref{eq:magnetic_vector_potential_integral}.

Relation $\bm{A}$ and $\bm{B}$:
Equation~\ref{eq:magnetic_vector_potential_curl} and~\ref{eq:magnetic_vector_potential_divergence}.

Relation $\bm{J}$ and $\bm{B}$:
Equation~\ref{eq:the_biot_savart_law} and~\ref{eq:amperes_law_differential}
and~\ref{eq:magnetic_field_divergence}.

\section{Boundary conditions for the magnetic field}%
\label{sec:boundary_conditions_for_the_magnetic_field}

The boundary conditions for $\bm{B}$ and
$\bm{A}$ are summarised below. They may be found
by consideration of Ampèrian loops and surfaces as we did for
the $\bm{E}$ field. We will not go into details
here.
\begin{definition}
	The boundary conditions for $\bm{B}$ are
	\begin{equation}
		\bm{B}_{\mathrm{above}} -
		\bm{B}_{\mathrm{below}}
		=
		\mu_{0}\left(\bm{K}\cross\hat{\bm{n}}\right)
	\end{equation}

	The boundary conditions for $\bm{A}$ are
	\begin{equation}
		\bm{A}_{\mathrm{above}}
		=
		\bm{A}_{\mathrm{below}}
	\end{equation}
	We also have the relation
	\begin{equation}
		\pdv{\bm{A}_{\mathrm{above}}}{n} -
		\pdv{\bm{A}_{\mathrm{below}}}{n}
		=
		\mu_{0}\bm{K}
	\end{equation}
\end{definition}
\textbf{Note:} $\bm{\hat{n}}$ is the normal
to the surface and $\bm{K}$ is the surface
current density, not the volume current density that is
required.

\section{Magnetic multipole expansion}%
\label{sec:magnetic_multipole_expansion}

As might be expected, the magnetic vector potential for a system
of currents may be expressed in terms of a multipole expansion.

\begin{definition}
	The magnetic multipole expansion in terms of
	$\bm{A}$ is
	\begin{equation}
		\bm{A}\left(\bm{r}\right)
		=
		\frac{\mu_{0}I}{4\pi}\oint\frac{1}{r^{\prime\prime}}\dd{\bm{l^\prime}}
		=
		\frac{\mu_{0}I}{4\pi}\sum_{n=0}^\infty\frac{1}{r^{n+1}}
		\oint{\left(r^\prime\right)}^n
		P_n\left(\cos{\alpha}\right)\dd{\bm{l^\prime}}.
	\end{equation}
\end{definition}

The first term in the expansion (the $1/r$
term), as in the case of electrostatics, is the monopole term.
As you might expect this turns out to always be zero (remember,
there are no magnetic monopoles). Hence, the second term
$1/r^2$ is the first non-zero term in the series
and corresponds to a magnetic dipole. In a system of currents
(with none going to infinity), we would expect all current
distributions to have this form.

\section{The magnetic dipole}%
\label{sec:the_magnetic_dipole}

The second term of the multipole expansion is the
\textit{magnetic dipole}, it is
\begin{equation}
	\bm{A}_{\mathrm{dipole}}
	=
	\frac{\mu_{0}I}{4\pi r^2}\oint\left(\bm{\hat{r}}\vdot\bm{r^\prime}\right)\dd{\bm{l^\prime}}
	=
	-\frac{\mu_{0}I}{4\pi r^2}\bm{\hat{r}}\cross\int\dd{\bm{a^\prime}}.
\end{equation}

\begin{definition}
	We may write that the $\bm{A}$ field of the
	dipole as
	\begin{equation}
		\label{eq:magnetic_vector_potential_dipole}
		\bm{A}_{\mathrm{r}}
		=
		\frac{\mu_{0}}{4\pi}\frac{\bm{m}\cross\bm{\hat{r}}}{r^2}
	\end{equation}
	where $\bm{m}$ is the magnetic dipole
	\begin{equation}
		\bm{m} = I \int\dd{\bm{a}}.
	\end{equation}
	This corresponds to a small loop of area $\bm{a}$
	carrying a current $I$ for which the
	direction of $\bm{a}$ is given by the usual right
	hand rule.
\end{definition}

Now, let's assume that the moment $m$ is
aligned along the $z$-axis so we can write
\begin{equation}
	\bm{A}_{\mathrm{dipole}}\left(\bm{r}\right)
	=
	\frac{\mu_{0}}{4\pi}\frac{m\sin{\theta}}{r^2}\bm{\hat{\phi}}.
\end{equation}

Working backwards, we see that
\begin{equation}
	\bm{B}_{\mathrm{dipole}}
	=
	\frac{\mu_{0}m}{4\pi r^3}\left(2\cos{\theta}\bm{\hat{r}}+\sin{\theta}\bm{\hat{\theta}}\right).
\end{equation}

This should be compared with the result from
equation~\ref{eq:electric_field_dipole_z}, where we notice the constant
$\rho/\epsilon_{0}$ is replaced by $\mu_{0}m$.

We can think of a magnetic dipole simply as a magnet, and we can
draw the field lines from that analogy.

If we dissect a bar magnet to its core elements, we find that
the origin of the magnetic field is from small current loops in
the atoms. So while electric dipoles comes from the separation
of two charges, the magnetic dipole originates from a tiny
current loop.

For both the magnetic and electric dipoles, we may lift the
restriction that the dipole axis is along the
$z$ direction, in which case, we can write
\begin{definition}
	The general magnetic dipole has a magnetic field expressed in
	coordinate-free form as
	\begin{equation}
		\bm{B}_{\mathrm{dipole}}
		=
		\frac{\mu_{0}m}{4\pi r^3}\left[3\left(\bm{m}\vdot\bm{\hat{r}}\right)\bm{\hat{r} - \bm{m}}\right].
	\end{equation}
\end{definition}

You should note that these results only apply when
$\bm{r}$ is large compared to the physical
dipole. You should also note the electric and magnetic dipoles
are quite distinct. One is caused by a separation of charge, the
other as a small current loop. Therefore, the fields are quite
different \textit{close} to the dipoles.

\section{Summary of magnetism}%
\label{sec:summary_of_magnetism}%

This concludes our discussion of magnetostatics. The discussion
has been brief than that of electrostatics. This is largely
because many of the ideas and concepts could be rolled over to
analysis.

\begin{definition}
	However, there are key and notable differences between the two:
	\begin{itemize}
		\item There are no magnetic monopoles.
		\item Magnetic fields do no work.
		\item Magnetic fields are not conservative so we cannot define a
		      universal scalar potential.
		\item We can define a magnetic vector potential $\bm{A}$
		      that can be applied usefully to solve problems in
		      magnetostatics.
		\item We have arbitrarily chosen $\divergence{\bm{A}} = 0$ to simplify
		      the form of $\bm{A}$ for our calculations. This is
		      not the only possible choice and for future reference, this is
		      known as the \textit{Coulomb gauge}.
	\end{itemize}
\end{definition}

\chapter{Magnetic Fields in Matter}%
\label{cha:magnetic_fields_in_matter}

\section{Introduction to magnetic fields in matter}%
\label{sec:introduction_to_magnetic_fields_in_matter}

We are probably more familiar with magnetic materials than the
electrostatic materials that we studied in
Section~\ref{cha:fields_in_matter}. For example, we have played
with permanent magnets, compasses, and know that the earth has a
magnetic field.

However, as we have seen in Section~\ref{cha:magnetostatics},
there is no such thing as a magnetic monopole and all magnetic
fields are produced by circulating currents. We might therefore
ask ourselves where the currents are, especially ones that
produce the magnetic field in a bar magnet.

The only answer to this question is that we need to understand
that atoms consist of nuclei around which currents (electrons)
flow in closed orbits and also that electrons have intrinsic
spin both of which can give rise to moments. At its simplest,
you might think of these orbits in terms of the Bohr atom (like
planets going around the sun). A full understanding necessitates
a full quantum physics treatment. However, the result is the
same---electrons have both spin and orbit the nucleus to produce
magnetic dipoles.

In atoms with odd numbers of electrons the electron spin may
give rise to small magnet moments that ordinarily would be in
random orientations due to thermal motion and hence the material
does not show an overall magnetic moment. However, in an
external magnetic field, these dipoles will tend to align in the
field (in an analogous way to electric dipoles in dielectrics)
to produce an overall moment such that the material appears
magnetised.

\begin{definition}
	The behaviour of these dipoles in materials is fundamental to
	their magnetic properties and three basic kinds of magnetic
	material are found (there are more than these three):
	\begin{itemize}
		\item \textbf{Paramagnets:} in which the dipoles due to spin tend to
		      line up in the external field.
		\item \textbf{Diamagnets:} in which the dipoles due to electron
		      orbits tend to line up opposite to the external field.
		\item \textbf{Ferromagnets:} in which the dipoles are aligned and
		      remain aligned even in the absence of an external field.
	\end{itemize}
\end{definition}

Analogues to these arise for insulators in electric fields but
are less common.

\section{Torques and forces on magnetic dipoles---origins of
  paramagnetism}%
\label{sec:torques_and_forces_on_magnetic_dipoles_origins_of_paramagnetism}

In the remainder of this section, we will discuss magnetic
materials in terms of idealised, classic current loops. This is
sufficient to understand what is happening. A proper treatment
with quantised systems reflects the characteristics we will find
here.

If we assume that the atoms in materials are magnetic dipoles
then the first thing we need to understand is how a magnetic
dipole moves in an external magnetic field.

The easiest to consider is a simple square loop carrying current
$I$. See Figure~\ref{fig:torque_magnetic}.
The torque $\bm{\Gamma}$ is
\begin{equation}
	\begin{aligned}
		\bm{\Gamma} & = aF\sin{\theta}\bm{i }                          \\
		F           & = IbB                                            \\
		\bm{\Gamma} & = IabB\sin{\theta}\bm{i} = mB\sin{\theta}\bm{i}.
	\end{aligned}
\end{equation}

\begin{figure}[htpb]
	\centering
	\includesvg[width=0.9\columnwidth]{./torque_magnetic.svg}
	\caption{The left shows the loop represented in all
		axes, the right shows a side view and the angles relative to the
		$z$ and $x$ axes.}%
	\label{fig:torque_magnetic}
\end{figure}

Note that $m = I\vdot ab$ is the magnetic moment of the
loop. The force comes from the Lorentz force due to the current
and the presence of a magnetic field (along the
$z$-axis.)

We can write this as
\begin{equation}
	\label{eq:magnetic_torque}
	\bm{\Gamma} =
	\bm{m}\cross\bm{B}.
\end{equation}

So the torque tries to align the moment parallel to the applied
field $\bm{B}$. This is the origin of the
\textit{paramagnetic} response of the electron spins in a
material. This would apply only for atoms/molecules with
`unpaired' electrons (as paired electrons have opposite moments
that cancel). In a uniform field, the net force is zero so there
is no linear movement of the dipole (only rotation).

If $\bm{B}$ is non-uniform there will be a force
exerted on the loop as well. For an infinitesimally small
dipole, this would be
\begin{definition}
	The force exerted on a infinitesimally small dipole is
	\begin{equation}
		\bm{F} = \gradient(\bm{m}\vdot\bm{B})
	\end{equation}
\end{definition}

\textbf{Aside:} the parallels between electric and
magnetic dipoles should be apparent. However, we should remember
that these fields are for regions well away from the sources
(separated charges or magnetic loops). As you approach a finite
size source the differences in the \textit{local}
field will become apparent.

We still speak of of \textit{north} and
\textit{south} poles of a magnet or the earth but we
know these don't exist (no magnetic monopoles). This model,
known as the \textit{Gilbert model} can sometimes be useful
for solving problems, especially if you know the results for
electrostatics (i.e.\ substitute positive and negative charges
for north and south poles) but does not work when we are close
to the dipole, in which case it is always better to work with
closed current loops.

\section{Diamagnetism}%
\label{sec:diamagnetism}

In Section~\ref{sec:torques_and_forces_on_magnetic_dipoles_origins_of_paramagnetism} we assumed the dipoles
originated from the intrinsic \textit{spin} of the
electrons in the atoms. However, the electrons also orbit the
nucleus which in turn also generates a current loop with an
associated moment. Again, this motion should be treated quantum
mechanically but the Bohr picture is easiest to visualise (and
the broad result is similar). Hence, we imagine the orbiting
electron to be a point negative charge. We also consider this
orbital motion to give rise to a steady current loop, that is we
further assume that rather than being a point charge it is a
continuously distributed charge in the orbit!

If we consider the orbit has a radius $R$,
then this current would be
\begin{equation}
	I = -\frac{e}{T} = -\frac{ev}{2\pi R}
\end{equation}
where $T$ is the orbital period and
$v$ the orbital speed. See
Figure~\ref{fig:diamagnetic}.

This will give rise to a magnetic moment $I\vdot A$
\begin{equation}
	\bm{m} = -\frac{1}{2} e v R
	\bm{\hat{z}}.
\end{equation}

In a field $\bm{B}$ this will be subject to a
torque as before, but it is more difficult to align the dipole
of the atomic orbit than the spin so paramagnetism from the
orbital contributions is weak.

However, what happens to $\bm{v}$ in a magnetic
field? Keeping to a classical (Bohr atom) picture there is an
additional force on the electron due to the Lorentz force.

For no magnetic field we have the central force equals the force
from the electric field
\begin{equation}
	\frac{1}{4\pi\epsilon_{0}}\frac{e^2}{R^2}
	=
	m_{e}\frac{v^2}{R}.
\end{equation}

But if we now add $\bm{B}$ in the
$\bm{\hat{z}}$ direction, we get a contribution to the
centripetal acceleration in the form of the Lorentz force, so
that
\begin{equation}
	\frac{1}{4\pi\epsilon_{0}}\frac{e^2}{R^2} +
	ev^{\prime}B
	=
	m e \frac{v^{\prime 2}}{R}.
\end{equation}

\begin{figure}[htpb]
	\centering
	\includesvg[width=0.9\columnwidth]{./diamagnetic.svg}
	\caption{The electron orbit around a positive nucleus and the magnetic
		moment induced in the presence of an external magnetic field
		along the $\bm{\hat{z}}$ direction.}%
	\label{fig:diamagnetic}
\end{figure}

Rearranging this expression, we get
\begin{equation}
	\begin{aligned}
		ev^\prime B & = \frac{m_e}{R}\left(v^{\prime 2} - v^2\right)\left(v^{\prime}+v\right) \\
		            & = \frac{m_e}{R}\Delta{v}\left(v^{\prime} + v\right).
	\end{aligned}
\end{equation}

If $\Delta{v}$ is small, then
$v^\prime + v \simeq v^\prime$, and this gives
\begin{equation}
	\Delta{v}\simeq\frac{eRB}{2m_e}.
\end{equation}

\begin{definition}
	The change in velocity indicates that the electron ``speeds up''
	and the dipole moment from the orbiting electron increases as
	\begin{equation}
		\Delta{\bm{m}}
		=
		-\frac{1}{2}e\left(\Delta{v}\right)R\bm{\hat{z}}
		=
		-\frac{e^{2}R^{2}}{4m_e}\bm{B}.
	\end{equation}
\end{definition}

So the moment $\bm{m}$ changes in the opposite
direction to $\bm{B}$. As usual, the
atom/molecule orbital moments are typically randomly oriented.
However, the net effect of the field is the orbital moments tend
to increase a little in the direction antiparallel to the field
(i.e.\ opposite to paramagnetism). The effect is known as
diamagnetism and it is much weaker than the paramagnetic effects
from unpaired electrons. Hence, it is commonly observed in
atoms/molecules with even numbers of atoms.

This is really a simplistic description of what is a more
complicated quantum effect, but the gist of the argument is
correct.

\section{Magnetisation}%
\label{sec:magnetisation}

From this point onwards, we will not look at the
atomic/molecular origins of the magnetisation of a material but
we will assume that we can describe it through a macroscopic
magnetisation vector $\bm{M}$. This would apply
where $\bm{M}$ would be aligned parallel to the
applied field (paramagnetism) or anti-parallel to the field
(diamagnetism). Apart from this difference in sign, the
treatment is the same. We can even use a permanent magnetisation
to describe ferromagnetism.
\begin{definition}
	We define the magnetisation vector $\bm{M}$ as
	\begin{equation}
		\label{eq:magnetisation_vector}
		\bm{M} =
		\lim\limits_{\Delta{\tau}\to 0}\frac{\sum\bm{m}_n}{\Delta{\tau}}
	\end{equation}
\end{definition}

\section{Bound currents}%
\label{sec:bound_currents}

In this section we will discuss \textit{bound currents} in
magnetised materials. These play a similar role to the bound
charge densities that we introduced in our discussion of
dielectrics.

\begin{derivation}

	Recall our equation for microscopic dipoles in
	Equation~\ref{eq:magnetic_vector_potential_dipole}, and note that
	$\bm{m}=\bm{M}\dd{\tau}$ is a small volume element as defined in
	Section~\ref{sec:magnetisation}, we extend this to all space by
	integrating
	\begin{equation}
		\bm{A}\left(\bm{r}\right)
		=
		\frac{\mu_{0}}{4\pi}\int\limits_{\mathrm{V}}\frac{\bm{M}\left(\bm{r^\prime}\right)\cross\bm{\hat{r}^{\prime\prime}}}{r^{\prime\prime 2}}
		\dd{\tau^\prime}
	\end{equation}

	As we did for dielectrics, we manipulate this further by writing
	\begin{equation}
		\gradient^\prime\left(\frac{1}{r^{\prime\prime}}\right)
		=
		\frac{\bm{\hat{r}^{\prime\prime}}}{r^{\prime\prime 2}}
	\end{equation}

	We can now expand the kernel using the product rule
	(rearranged):
	\begin{equation}
		\bm{A}\left(\bm{r}\right)
		=
		\frac{\mu_{0}}{4\pi}\left\{\iiint\frac{1}{r^{\prime\prime}}\left[\gradient^\prime\cross\bm{M\left(\bm{r}^{\prime}\right)}\right]\dd{\tau^\prime}
		-
		\iiint\gradient^\prime\cross\left[\frac{\bm{M\left(\bm{r}^{\prime}\right)}}{r^{\prime\prime}}\right]\dd{\tau^\prime}\right\}
	\end{equation}

	With Stoke's Law, we can rewrite the second term into
	\begin{equation}
		-\oint\frac{1}{r^{\prime\prime}}\left[\bm{M}\left(\bm{r^\prime}\cross\dd{\bm{a^\prime}}\right)\right].
	\end{equation}

	The first term looks like the potential of a volume current in
	terms of the magnetisation vector.
	\begin{equation}
		\bm{J}_b = \curl{\bm{M}}
	\end{equation}

	And the second looks the potential of a surface current
	\begin{equation}
		\bm{K}_b =
		\bm{M}\cross\bm{\hat{n}}
	\end{equation}

	By defining these two terms, we can write
	\begin{equation}
		\bm{A}\left(\bm{r}\right)
		=
		\frac{\mu_{0}}{4\pi}\left\{
		\int\limits_{V}\frac{\bm{J}_b\left(\bm{r^\prime}\right)}{r^{\prime\prime}}\dd{\tau^\prime}
		+
		\oint\frac{\bm{K}_b\left(\bm{r^\prime}\right)}{r^{\prime\prime}}\dd{\bm{a}^\prime}
		\right\}
	\end{equation}
	In other words, we have reduced the potential to that of a
	contribution coming from the \textit{volume bound} currents
	flowing through the material plus a contribution from the
	\textit{surface bound} currents on the boundary.
\end{derivation}

What we effectively did was sum up the contributions from every
individual dipole in the material and converted this into two
integrals, the first being a volume which expresses the vector
potential coming from the bulk of the material (the volume bound
currents), the second being a contribution from the surface
bound currents, which is magnetisation crossed with the normal
vector to the surface.

This treatment is analogous to the approach we used for the
bound surface and volume charge densities in dielectrics so we
can compare the expressions for both here.

\begin{center}
	\begin{tabular}{c  c}
		\textbf{Volume}                 & \textbf{Surface}                     \\
		\toprule
		$\rho_b = -\divergence{\bm{P}}$ & $\sigma_b = \bm{P}\vdot\bm{\hat{n}}$ \\
		$\bm{J}_b = \curl{\bm{M}}$      & $\bm{K}_b =
		\bm{M}\cross\bm{\hat{n}}$                                              \\
	\end{tabular}
\end{center}

For the volume terms, instead of having a divergence---a source,
we now have the curl. We can think of these as internal
\textit{currents} in the material.

\section{Visualisation of the bound currents}%
\label{sec:visualisation_of_the_bound_currents}

Note that as in the case of dielectrics the actual fields in a
material will vary a lot from point to point due to
fluctuations, so what we are describing here is really the
macroscopic mean field. See Figure~\ref{fig:visualising_magnetic_surface_volume_bound_currents}.

\begin{figure}[htpb]
	\centering
	\includesvg[width=0.9\columnwidth]{./visualising_magnetic_surface_volume_bound_currents.svg}
	\caption{The left diagram shows how the surface bound currents can be
		understood. The currents in the `centre' of a group of square
		loops cancel out, and what we have is a net effect at the surface of the material.
		The right diagram shows how if one square loop has a different
		magnetic dipole moment, and this creates tiny currents in every element due to
		incomplete cancellation, and we get a net effect seen as the volume bound currents.}%
	\label{fig:visualising_magnetic_surface_volume_bound_currents}
\end{figure}

\section{The auxiliary field H}%
\label{sec:the_auxiliary_field_h}

Just as we did for the electric field, we can now imagine a
\textit{new} field in which we only need to
calculate the effects of the free currents (i.e.\ those not
coming from the magnetisation).

This is analogous to the introduction of the
$\bm{D}$ field in the case of dielectrics where
we eliminated the need to determine the bound charges.

So, as we did with the charge densities and
$\bm{E}$, we write in analogy.
\begin{equation}
	\bm{J} = \bm{J}_b +
	\bm{J}_{\mathrm{free}}
\end{equation}

Our free current might flow through the wires embedded in the
magnetised substance, or through the material itself if it is a
conductor. The two currents are quite different, the free
current is there because we hooked up a wire to a battery---it
involves a transport of charge, while the bound current is there
because of \textit{magnetisation}---it results from many
aligned atomic dipoles.

From Ampère's Law~\ref{eq:amperes_law_differential}, we can write
\begin{equation}
	\frac{1}{\mu_{0}}\left(\curl{\bm{B}}\right)
	=
	\bm{J}
	=
	\bm{J}_b +
	\bm{J}_{\mathrm{free}}
	=
	\left(\curl{\bm{M}}\right) +
	\bm{J}_{\mathrm{free}}.
\end{equation}

Collect the curls to get
\begin{equation}
	\curl{\left(\frac{1}{\mu_{0}}\bm{B} - \bm{M}\right)}
	=
	\bm{J}_{\mathrm{free}}.
\end{equation}

We define the quantity in the curl to be the auxiliary field
$\bm{H}$.

\begin{definition}
	The field $\bm{H}$ is given as
	\begin{equation}
		\label{eq:auxiliary_field_h}
		\bm{H}
		\equiv
		\frac{1}{\mu_{0}}\bm{B} -
		\bm{M}.
	\end{equation}

	Its differential form, derived from Ampère's Law, reads
	\begin{equation}
		\label{eq:auxiliary_field_differential}
		\curl{\bm{H}} =
		\bm{J}_{\mathrm{free}}
	\end{equation}
	\begin{equation}
		\label{eq:auxiliary_field_integral}
		\oint\bm{H}\vdot\dd{\bm{l}}
		=
		I_{\mathrm{free}}
	\end{equation}
\end{definition}

The application and use of $\bm{H}$ in
magnetostatics is analogous to $\bm{D}$ in
electrostatics. In the same way as $\bm{D}$
allowed us to use Gauss' Law with just the free charges, we find
that we can use Ampère's Law with the free currents to calculate
$\bm{H}$.

\textbf{Note:} as discussed in the case of
$\bm{D}$, we can use the free currents to
calculate $\bm{H}$ using Ampère's Law in cases of
high symmetry but it would be false to assume we can do this in
\textit{all} cases.

The curl does not uniquely define the vector field---we would
also need to know the divergence. From our
definition~\ref{eq:auxiliary_field_h}, we see that the divergence
of $\bm{H}$ is
\begin{equation}
	\label{eq:auxiliary_field_divergence}
	\divergence{\bm{H}}
	=
	-\divergence{\bm{M}}
\end{equation}

Provided that $-\divergence{\bm{M}} = 0$, which is usually the case
of high symmetry, we can use Ampère's Law. Otherwise, we need to
work directly with the magnetisation.

\section{Boundary conditions for the auxiliary field}%
\label{sec:boundary_conditions_for_the_auxiliary_field}

Without proof, we can write the boundary conditions for
$\bm{H}$ and $\bm{B}$ as follows

\begin{definition}
	The boundary conditions for $\bm{H}$ and
	$\bm{B}$ are
	\begin{equation}
		H^\perp_{\mathrm{above}} - H^\perp_{\mathrm{above}} =
		-\left(M^\perp_{\mathrm{above}} - M^\perp_{\mathrm{above}}\right)
	\end{equation}
	\begin{equation}
		H^\parallel_{\mathrm{above}} - H^\parallel_{\mathrm{above}} =
		\bm{K}_{\mathrm{free}}\cross\bm{\hat{n}}
	\end{equation}
	\begin{equation}
		B^\perp_{\mathrm{above}} - B^\perp_{\mathrm{above}} = 0
	\end{equation}
	\begin{equation}
		B^\parallel_{\mathrm{above}} - B^\parallel_{\mathrm{above}} =
		\mu_{0}\left(\bm{K}_{\mathrm{free}}\cross\bm{\hat{n}}\right)
	\end{equation}
\end{definition}

\section{Linear magnetic materials}%
\label{sec:linear_magnetic_materials}

For paramagnetic and diamagnetic materials the magnetisation
disappears when the applied field is removed. As for
dielectrics, we may then consider the simplest case of
\textit{linear} induced magnetisation from the external
field, i.e.\ $M\propto\bm{B}$. However, the convention is
to take $\bm{M}\propto\bm{H}$ and to write
\begin{equation}
	\bm{M} =
	\chi_{m}\bm{H}
\end{equation}
where $\chi_m$ is the magnetic susceptibility, a
dimensionless quantity typically around \num{10e-4}
to \num{10e-6} for most materials. Materials that
obey this equation are known as linear magnetic materials or
media.

For a linear media, we can expand
Equation~\ref{eq:auxiliary_field_h} into
\begin{equation}
	\frac{\bm{B}}{\mu_{0}}
	=
	\bm{H} + \bm{M}
	=
	\bm{H} + \chi_m\bm{H}
\end{equation}

\begin{definition}
	For a linear medium, we can define the following:

	We define $\mu$ as the permeability of the
	material, and $\mu_r$ as the relative
	permeability.
	\begin{equation}
		\bm{B}
		=
		\mu_{0}\left(1+\chi_m\bm{H}\right)
		=
		\mu\bm{H}
		\hspace{1em}\mathrm{with}\hspace{1em}
		\mu\equiv\mu_{0}\left(1 + \chi_m\right)
	\end{equation}
\end{definition}

\section{Ferromagnetism}%
\label{sec:ferromagnetism}

We have mentioned ferromagnetism several times already. We know
it is very important technologically, but in the end it is
difficult to discuss it in terms of a simple macroscopic
magnetisation. It turns out that a material in which all the
dipoles are aligned with each other does not really occur.
Instead, it will tend to divide into `domains' that have
different orientations of the magnetisation (while leaving the
overall permanent magnetisation).

The formation of domains is the most energetically favourable
configuration and it is their behaviour that gives rise to
common properties of permanent magnets such as hysteresis.

\section{Summary of magnetism in matter}%
\label{sec:summary_of_magnetism_in_matter}

We take note of the following qualitative points:

\begin{definition}
	\begin{itemize}
		Magnetism in matter concerns the following:
		\item The magnetic response of materials is due to alignment of the
		      spin and orbital moments of electrons in atoms and molecules.
		\item Paramagnetism originates from the alignment in the field of the
		      spin angular momentum from unpaired electrons.
		\item Diamagnetism is weaker and originates from the alignment,
		      antiparallel to the field, of the orbital moments of the
		      electrons.
		\item We describe the magnetisation (polarisation) of the material in
		      terms of the magnetisation vector $\bm{M}$, the
		      dipole moment per unit volume.
		\item We can describe the effect of the magnetisation in terms of
		      bound surface and volume current densities
		      ($\bm{K}_b$ and $\bm{J}_b$).
		\item We introduce the $\bm{H}$ field as a device to
		      solve problems without the need to calculate the bound current
		      densities.
		\item In linear magnets we introduce the magnetic susceptibility
		      $\chi_m$ and the relative permeability
		      $\mu_r$ to describe the relationship between
		      $\bm{B}$ and $\bm{H}$.
	\end{itemize}
\end{definition}

%\bibliography{bibfile}
\end{document}

% vim: fen fdm=syntax
