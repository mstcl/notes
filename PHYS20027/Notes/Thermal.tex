%%%%%%%%%%%%%%%%%%%%%%%%%%%%%%%%%%%%%%%%%%%%%%%%%%%%%%%%%%%%%%%%%%%%%
\documentclass[12pt,chapterprefix=false,dvipsnames]{scrbook}
\KOMAoptions{twoside=false}
%%%%%%%%%%%%%%%%%%%%%%%%%%%%%%%%%%%%%%%%%%%%%%%%%%%%%%%%%%%%%%%%%%%%%

%%%%%%%%%%%%%%%%%%%%%%%%%%%%%%%%%%%%%%%%%%%%%%%%%%%%%%%%%%%%%%%%%%%%%
% Useful packages
\usepackage{mathtools}
\usepackage{booktabs}
\usepackage{caption}
\usepackage{svg}
\usepackage{siunitx}
\usepackage{amssymb,amsmath,physics,bm,bbold}
\usepackage{enumerate}

%--------------------------------------------------------------------
% Hyper ref
\usepackage{hyperref}
\usepackage{cleveref}

\colorlet{mylinkcolor}{NavyBlue}
\colorlet{mycitecolor}{Aquamarine}
\colorlet{myurlcolor}{Aquamarine}

\hypersetup{
	linkcolor  = mylinkcolor!,
	citecolor  = mycitecolor!,
	urlcolor   = myurlcolor!,
	colorlinks = true,
}
%--------------------------------------------------------------------
% Bibliography
\usepackage[]{natbib}
\bibliographystyle{chicago}

%=================================
% pre-defined theorem environments
\usepackage{amsthm}
\usepackage{framed}
\newtheoremstyle{dotless}{}{}{\itshape}{}{\bfseries}{}{ }{}
\theoremstyle{dotless}
\newtheorem{prototheorem}{Theorem}[section]

\newenvironment{theorem}
{\colorlet{shadecolor}{orange!15}\begin{shaded}\begin{prototheorem}}
			{\end{prototheorem}\end{shaded}}

\newtheorem{protolemma}[prototheorem]{Lemma}
\newenvironment{lemma}
{\colorlet{shadecolor}{blue!15}\begin{shaded}\begin{protolemma}}
			{\end{protolemma}\end{shaded}}

\newtheorem{protocorollary}[prototheorem]{Corollary}
\newenvironment{corollary}
{\colorlet{shadecolor}{pink!15}\begin{shaded}\begin{protocorollary}}
			{\end{protocorollary}\end{shaded}}

\theoremstyle{definition}
\newtheorem{protonotation}{Notation}[section]
\newenvironment{notation}
{\colorlet{shadecolor}{green!15}\begin{shaded}\begin{protonotation}}
			{\end{protonotation}\end{shaded}}

\newtheorem{protoderivation}{Derivation}[section]
\newenvironment{derivation}
{\colorlet{shadecolor}{purple!15}\begin{shaded}\begin{protoderivation}}
			{\end{protoderivation}\end{shaded}}

\newtheorem{protoexample}{Example}[section]
\newenvironment{example}
{\colorlet{shadecolor}{red!15}\begin{shaded}\begin{protoexample}}
			{\end{protoexample}\end{shaded}}

\newtheorem{protodefinition}{Definition}[section]
\newenvironment{definition}
{\colorlet{shadecolor}{black!15}\begin{shaded}\begin{protodefinition}}
			{\end{protodefinition}\end{shaded}}

%=================================
% useful commands
\DeclareMathOperator*{\argmin}{arg\,min}
\DeclareMathOperator*{\argmax}{arg\,max}
\DeclareMathOperator*{\supp}{supp}

\def\vec#1{{\ensuremath{\bm{{#1}}}}}
\def\mat#1{\vec{#1}}

%=================================
% convenient notations
\newcommand{\XX}{\mathbb{X}}
\newcommand{\Hamiltonian}{\mathcal{H}}
\newcommand{\RR}{\mathbb{R}}
\newcommand{\EE}{\mathbb{E}}
\newcommand{\PP}{\mathbb{P}}
\newcommand{\CC}{\mathbb{C}}

\newcommand{\sL}{\mathcal{L}}
\newcommand{\sX}{\mathcal{X}}
\newcommand{\sY}{\mathcal{Y}}

\newcommand{\ind}{\mathbb{1}}

\def\dbar{{\mathchar'26\mkern-10mu \mathrm{d}}}
%%%%%%%%%%%%%%%%%%%%%%%%%%%%%%%%%%%%%%%%%%%%%%%%%%%%%%%%%%%%%%%%%%%%%
% Typography, change document font
\usepackage[tt=false, type1=true]{libertine}
\usepackage[varqu]{zi4}
\usepackage[libertine]{newtxmath}
\usepackage[T1]{fontenc}

\usepackage[protrusion=true,expansion=true]{microtype}

% Disable paragraph indentation, and increase gap
\usepackage{parskip}

\title{Thermal Physics Notes}
\author{Compiled by Nhat Pham\\ based on lectures from PHYS20027 \\and
	Blundells' \textit{Concepts in Thermal Physics} }
\date{Last update: \today}
\begin{document}
\maketitle

\tableofcontents

\chapter{Lecture 1 (Week 13)}%
\label{cha:lecture_1}

\section{Definitions}%
\label{sec:definitions}

\subsection{Thermodynamic limit}%
\label{sub:thermodynamic_limit}

Thermodynamics apply to system that is \textit{in equilibrium}
and also within the thermodynamic limit

\begin{definition}
	The thermodynamic limit is when we study a large number of
	particles or a small number of particles for a long time. When
	we take this limit, we can describe the system without knowing
	all individual, \textit{microscopic}  properties of all the
	particles. Instead, we can completely describe the system using
	\textit{macroscopic} properties.
\end{definition}

Macroscopic properties can be variables such as pressure, time,
volume, etc.

\begin{figure}[htpb]
	\centering
	\includesvg[width=0.9\columnwidth]{thermo_limit.svg}
	\caption{Force per particle versus time as we reach the thermodynamic
		limit.}%
	\label{fig:thermo_limit}%
\end{figure}

\subsection{Thermodynamic system}%
\label{sub:thermodynamic_system}

\begin{definition}
	A thermodynamic system is a body which has well defined
	interactions with its surrounds and can be considered separately
	from them.
\end{definition}

Example of a thermodynamic system: gas in an enclosure, where
interactions with surrounding might be through heat or transfer
through walls.

\begin{figure}[htpb]
	\centering
	\includesvg[width=0.9\columnwidth]{thermo_system.svg}
	\caption{A thermodynamic system}%
	\label{fig:thermo_system}
\end{figure}

\subsection{Thermal equilibrium}%
\label{sub:thermal_equilibrium}

\begin{definition}
	A system is in thermal equilibrium when its macroscopic
	observables have ceased to change with time. Two systems are in
	thermal equilibrium if no heat flows between them when they are
	connected by a path permeable to heat.
\end{definition}

There are also metastable states, where variables my change
slowly with time, so the system is \textit{not} in
thermal equilibrium.

\subsection{Extensive and intensive variables}%
\label{sub:extensive_and_intensive_variables}

\begin{itemize}
	\item \textbf{Extensive} variables scales with the size of the
	      system. Examples are the number of particles, volume, entropy,
	      magnetic moment. They are mostly upper case.
	\item \textbf{Intensive} variables are independent of the size of
	      the system. Examples are number density, temperature,
	      magnetisation, pressure, entropy density. They are mostly lower
	      case.
\end{itemize}

\section{Functions and equations of state}%
\label{sec:functions_of_state}

\begin{definition}
	A state function assumes a unique value for each equilibrium
	state of the system. The value does not depend on how they
	system got to the state---it is path independent.
\end{definition}

Mathematically, functions of state have the following
properties.
\begin{itemize}
	\item Depend on $(x,y)$ but not the path taken to
	      $(x,y)$.
	\item Have exact differentials, the number of variables does not
	      matter. See~\ref{def:exact_differentials}.
\end{itemize}

\begin{derivation}%
	\label{def:exact_differentials}
	Recall the Taylor expansion of a function
	$f(x)$:
	\begin{equation}
		f(x + \delta x) = f(x) + f^\prime\delta x +
		\frac{1}{2!}f^{\prime\prime}{(\delta x)}^2 + \hdots
	\end{equation}

	In the limit $\delta x \rightarrow \dd{x}$ we can ignore higher order
	terms because they are sufficiently small. We are left with the
	first derivative, and rewrite it so that:

	\begin{equation}
		f(x + \dd{x}) - f(x) =
		\dv{f}{x}\dd{x} =
		\dd{f}
	\end{equation}

	We can use this result and expand it to 2 dimensions. We then
	arrive at the \textit{exact differential} of $f(x,y)$:

	\begin{equation}
		\dd{f}
		=
		f(x + \dd{x}, y + \dd{y}) -
		f(x,y) = \pdv{f}{x}\dd{x} +
		\pdv{f}{y}\dd{y}
	\end{equation}

	The condition for a function to have an exact differential is
	that its mixed partial derivatives are equal. In other words:
	\begin{equation}
		\pdv{f}{x}{y}
		=
		\pdv{f}{y}{x}
	\end{equation}
\end{derivation}

On the other hand, functions that are path-dependent are
\textit{non-state functions}. In thermodynamics, heat and work are
non-state functions. Their derivatives are inexact. The notation
for this is: \[
	\dbar W ,\,
	\dbar Q
\] In the thermodynamic limit
and thermodynamic equilibrium, the variables are dependent on
each other and are constrained by an equation of state. For
example $f(x,y,z) = 0$. The ideal gas equation is an
equation of state.
\begin{equation}
	\label{eq:ideal_gas}
	pV - nRT = 0
\end{equation}
This allows us to relate thermodynamic variables to each other.
For example: \[
	\pdv[2]{m}{T} =
	\frac{1}{T}\pdv{C}{B}
\] We can find out the value
of one side by knowing the other one.

\chapter{Lecture 2 (Week 13)}%
\label{cha:lecture_2}

A TL;DR overview of the laws of thermodynamics:
\begin{itemize}
	\item \textbf{Zeroth Law:} If two systems are separately in
	      equilibrium with a third system, then they are in equilibrium
	      with each other.
	\item \textbf{First Law:} Energy is conserved. Heat and work are
	      both forms of energy.
	\item \textbf{Second Law:} Heat cannot be converted to work with
	      100\% efficiency. Entropy of the universe cannot decrease (this
	      gives a definite direction of time).
	\item \textbf{Third Law:} You cannot cool anything to absolute zero
	      (0 Kelvins)
\end{itemize}

These are statements of empirical facts, not deeper theory that
sits below these laws.

\section{The First Law}%
\label{sec:the_first_law}

\textbf{Heat} is the energy in transfer between one
system and another or one system and the surroundings.
\begin{itemize}
	\item $\dbar Q$ is the heat supplied
	      \textbf{to} the system.
\end{itemize}

\textbf{Work} is the change in energy of a system
affected by changing its parameters.
\begin{itemize}
	\item $\dbar W$ is
	      the work done \textbf{on} the system.
\end{itemize}

\textbf{Internal energy} is the sum of all the components of
energy in a system.

\begin{definition}
	The conservation of energy of both heat ($Q$)
	and work ($H$):
	\begin{equation}
		\label{eq:first_law}
		\Delta U = \Delta Q + \Delta W
	\end{equation}
	In differential form:
	\begin{equation}
		\dd{U}
		=
		\dbar W + \dbar Q
	\end{equation}
\end{definition}

\subsection{Work done compressing a gas}%
\label{sub:work_done_compressing_a_gas}

Suppose a piston can be moved into a chamber with a gas. If it
moves fast, we generate sound waves, if there is friction, then
energy will be dissipated as heat. These losses will not end up
in the gas so the change is \textbf{irreversible}.

To have \textbf{reversible} changes, we must have a
frictionless piston and we must move it slowly.

\begin{definition}
	The reversible work done to compress a gas is expressed by
	\begin{equation}
		\label{eq:work_pdV}
		\dbar W
		=
		-pA \times \frac{\dd{V}}{A}
		=
		-p\dd{V}
	\end{equation}
	The sign is negative because if we are doing work
	\textbf{on} the gas, then $\dd{V}$ is
	negative because the volume is decreasing. For irreversible
	changes,
	\begin{equation}
		\dbar W
		\geq
		-p\dd{V}
	\end{equation}
\end{definition}

\subsection{Work done in other systems}%
\label{sub:work_done_in_other_systems}

\begin{itemize}
	\item Stretching a string: $\dbar W = \vec{F}\vdot\dd{\vec{l}}$
	\item Expanding a surface: $\dbar W = \gamma\dd{A}$
	\item Magnetic material: $\dbar W = \vec{B}\vdot\dd{\vec{m}}$ or
	      $\vec{m}\vdot\dd{\vec{B}}$
\end{itemize}

\section{Constraints}%
\label{sec:constraints}

\begin{itemize}
	\item \textbf{Adiabatic} Thermally isolated system---there is no
	      heat flow in or out.\, $\dbar Q = 0$ or
	      $\dd{S} = 0$
	\item \textbf{Isothermal:} Temperature is fixed by an external
	      reservoir.\, $\dd{T} = 0$
	\item \textbf{Isobaric:} Pressure is fixed.\,
	      $\dd{p} = 0$
	\item \textbf{Isochoric:} Volume is fixed.\, $\dd{V} = 0$
\end{itemize}

Paths are correspondingly called \textit{adiabat, isotherm, isobar,} and
\textit{isochar}.

\section{Heat capacities}%
\label{sec:heat_capacities}

A way to quantify the response of the system to external
changes. Two heat capacities we study are
\[
	C_{V} = {\left(\pdv{Q}{T}\right)}_V,
	C_{p} = {\left(\pdv{Q}{T}\right)}_p,
\]

The equation of state for a general gas (not necessarily ideal)
is $f(p,T,V) = 0$. We only have 2 independent variables
that we can write our internal energy as a function of any two
of those. For example, $U(p,V)$ or
$U(V,T)$.

Let's qualitatively arrive at the expressions of
$C_{V}$ and $C_{p}$.

For $C_{V}$, suppose a box with fixed walls.
Heat supplied into the system will increase the temperature of
the gas inside.

For $C_{p}$, suppose a box with a piston. The
gas is allowed to expand and move the piston to keep the
pressure constant. If we put heat into the system, the gas
expands, doing work on the system. Thus, not all of the heat
supplied increased the temperature of the system. We expect then
that $C_{p} \geq
	C_{V}$.

\begin{derivation}
	Assume $U = U(V,T)$ and $f(p,V,T) = 0$. The
	exact differential of $U$ is
	\begin{equation}
		\label{eq:exact_diff_U}
		\dd{U}
		=
		{\left(\pdv{U}{V}\right)}_T\dd{V} +
		{\left(\pdv{U}{T}\right)}_V\dd{T}
	\end{equation}
	From the equation of the first law,~\ref{eq:first_law},
	substitute\, $\dbar W$
	with~\ref{eq:work_pdV}, set $dV = 0$ for
	constant volume, and divide by $\dd{T}$,
	\begin{equation}
		{\left(\pdv{U}{T}\right)}_V
		=\,
		{\left(\pdv{Q}{T}\right)}_V
		=
		C_{V}
	\end{equation}
	Then for $C_{p}$, equate~\ref{eq:exact_diff_U}
	with~\ref{eq:first_law} then divide by
	$\dd{T}$ and set $\dd{p} = 0$:
	\begin{equation}
		\label{eq:cp_in_terms_of_cv}
		C_{p} =
			{\left(\pdv{Q}{T}\right)}_{p} = {\left(\pdv{V}{T}\right)}_{p}\left(p+{\left(\pdv{U}{V}\right)}_T\right) +
		C_{V}
	\end{equation}
	For a general gas, we can normally stop here.
\end{derivation}

With the general expression, we can evaluate this for different
gasses. The most simple is monotonic ideal gas.

\begin{example}
	For an ideal gas that is monotonic, $U = \frac{3}{2}nRT$ and
	thus is only a function of temperature. We can cancel any
	partial derivatives of $U$ where
	$T$ is kept constant.
	\begin{equation}
		C_{V} = \frac{3}{2}nR
	\end{equation}
	\begin{equation}
		C_{p} = {\left(\pdv{V}{T}\right)}_{p}p + \frac{3}{2}nR
	\end{equation}
	Using the ideal gas relation~\ref{eq:ideal_gas}, we have:
	\begin{equation}
		C_{p} = \frac{3}{2}nR + nR = \frac{5}{2}nR
	\end{equation}
\end{example}

We can also consider \textbf{Virial Expansion} and the
\textbf{Van de Waals} equation, which models real gas.

\section{Non-ideal gases}%
\label{sec:non_ideal_gases}

\subsection{Virial Expansion}%
\label{sub:virial_expansion}

This is a general form for an equation of state of gas or fluid.
For an ideal gas, $A = 1$ and all other
coefficients are zero. Gases become ideal when
$\rho \rightarrow 0$.

\begin{equation}
	\label{eq:virial_expansion}
	\frac{p}{RT\rho}
	=
	A + B\rho + C\rho^2 + D\rho^3 + \hdots
\end{equation}

\subsection{Van de Waals equation}%
\label{sub:van_de_waals_equation}

This is a special case of the virial expansion, for
$\rho \ll 1$, the we can ignore second order terms
and above.

\begin{equation}
	\label{eq:van_de_waals}
	\left(p + \frac{a}{V^2}\right)\left(V - b\right)
	=
	nRT
\end{equation}

%\bibliography{bibfile}

\chapter{Lecture 3 (Week 13)}%
\label{cha:lecture_3}

\section{The Second Law}%
\label{sec:the_second_law}

\begin{definition}
	There are several equivalent statements of the second law:
	\begin{itemize}
		\item \textbf{Lord Kelvin 1851:} Work can be converted to heat with 100\%
		      efficiency but not the reverse.
		\item \textbf{Clausius 1854:} Heat flows from hot to cold, and not
		      spontaneously the other way around.
	\end{itemize}
\end{definition}

Both of these statements are \textbf{equivalent}---to
understand this we need to look at thermodynamic cycles,
specifically the \textit{Carnot cycle}. This concept is used
in engines. Since internal combustion engines are difficult to
understand, we look at the \textit{external} combustion
engines. One type is the \textit{Stirling} engine.

Qualitatively, the source of heat is at the bottom (which can
come from a hot plate). The difference between the hot plate and
the cold plate runs the engine (cool video embedded in the
lecture). Here's a link to diagrams \& animations:
\href{http://animatedengines.com/ltdstirling.html}{Low Differential Stirling}.

\section{The Carnot Cycle}%
\label{sec:the_carnot_cycle}

There are four stages in the cycle. This is a
\textit{closed cycle}. Thus, the work done can be described
by integrating both sides of~\ref{eq:work_pdV} to get

\begin{equation}
	\label{eq:work_pdV_integral_form}
	W = \int p\dd{V}
\end{equation}

On the other hand, the total change $\Delta U = 0$
after one complete cycle.

Here are the steps in the Carnot cycle:
\begin{enumerate}
	\item \textbf{A to B:} Isothermal expansion at
	      $T_H$. Volume increases
	      as pressure decreases due to supplied heat
	      $Q_H$.
	\item \textbf{B to C:} Adiabatic expansion (gas cools to
	      $T_L$) as
	      pressure keeps decreasing and volume keeps increasing.
	\item \textbf{C to D:} Isothermal compression at
	      $T_L$. Heat leaves
	      the system and the volume decreases while pressure increases.
	\item \textbf{D to A:} Adiabatic compression (gas heats to
	      $T_H$).
	      Pressure increase as a result as temperature increases.
\end{enumerate}

\begin{figure}[htpb]
	\centering
	\includesvg[width=0.8\columnwidth]{carnot_cycle.svg}
	\caption{Carnot cycle P-V plot}%
	\label{fig:carnot_cycle}
\end{figure}

Physically, there must be a hot reservoir and a cold reservoir
connected to our system, which is an enclosed piston connected
to a flywheel. These two are connected to a switch each
(\ref{fig:carnot_cycle_example}).

Here's the physical interpretation:
\begin{enumerate}
	\item The cold switch is open, and heat is supplied to the system. The
	      gas must expand along the isotherm, pushes the piston and drives
	      the flywheel.

	\item In stage 2, both switches are open, the system is thermally
	      isolated. The flywheel continues to spin while the gas continues
	      to expand.

	\item In stage 3, the cold switch is now closed. Heat leaves the
	      system and the gas gets compressed.

	\item In stage 4, both switches are open, the system is thermally
	      isolated. The piston is back to its original position after the
	      flywheel compresses the gas.
\end{enumerate}

\begin{figure}[htpb]
	\centering
	\includesvg[width=0.9\columnwidth]{carnot_engine.svg}
	\caption{An example of what how the Carnot cycle might be implemented}%
	\label{fig:carnot_cycle_example}
\end{figure}

\subsection{The Carnot relation}%
\label{sub:the_carnot_relation}

We can also analyse the Carnot cycle mathematically.

Let's consider an isothermal expansion. An isothermal expansion
(see~\ref{sec:constraints}) demands $\dd{T} = 0$,
but $U = U(T)$ thus $\Delta U = 0$.
Therefore, $\Delta Q = -\Delta W$
from~\ref{eq:first_law}. From~\ref{eq:work_pdV_integral_form}, we
evaluate the integral using~\ref{eq:ideal_gas}.

Now let's consider an adiabatic expansion. Recall from first
year that we can write $PV^\gamma = \mathrm{const}$ and thus
$TV^{\gamma - 1} = \mathrm{const}$. Since it is equal to a constant, the
product $TV$ at each stages are equal.
Remember that $\gamma = \frac{C_p}{C_V}$.

\begin{definition}%
	\label{eq:carnot_stages}
	We have the following mathematical relations for each stages of
	the Carnot cycle, with $\gamma = \frac{C_p}{C_V}$.
	\begin{enumerate}
		\item $Q_H = nRT_H\naturallogarithm\frac{V_B}{V_A}$
		\item $T_H V_B^{\gamma - 1} = T_L V_C^{\gamma - 1}$
		\item $-Q_L = nrT_L\naturallogarithm\frac{V_D}{V_C}$
		\item $T_L V_D^{\gamma - 1} = T_H V_A^{\gamma - 1}$
	\end{enumerate}
	Use stages 2 and 4 from~\ref{eq:carnot_stages} to eliminate
	$\frac{T_H}{T_L}$. We arrive at:
	\begin{equation}
		\label{eq:the_carnot_relation_intermediate}
		\frac{V_C}{V_B}
		=
		\frac{V_D}{V_A}
	\end{equation}
	Finally, using stages 1 and 3 and~\ref{eq:the_carnot_relation_intermediate} which
	we just found, we arrive at the \textit{Carnot relation}:
	\begin{equation}
		\frac{Q_H}{Q_L}
		=
		\frac{T_H\naturallogarithm\left(\frac{V_B}{V_A}\right)}{T_L\naturallogarithm\left(\frac{V_C}{V_A}\right)}
		=
		\frac{T_H}{T_L}
	\end{equation}
	Finally, we can derive the efficiency of the cycle, which is the
	ratio between work done and heat available. The work done is the
	difference between heat flowing in and heat flowing out. The
	heat available is what is supplied.
	\begin{equation}
		\label{eq:carnot_efficiency}
		\eta = \frac{Q_H - Q_L}{Q_H} = 1 - \frac{T_L}{T_H}
	\end{equation}
\end{definition}

It could also be useful to consult simplified diagrams of Carnot
engines and a reversed Carnot engine (a refrigerator).
\begin{figure}[htpb]
	\centering
	\includesvg[width=0.9\columnwidth]{./carnot_simplified.svg}
	\caption{(a) shows the Carnot engine where the engine performs work and
		(b)
		shows the reversed Carnot engine where we perform work on the
		engine.}%
	\label{fig:name}
\end{figure}

\subsection{The Carnot theorem}%
\label{sub:the_carnot_theorem}

The consequences of the Carnot engine are:

\begin{theorem}
	No engine operating between two reservoirs can be more efficient
	than the Carnot engine.
\end{theorem}

Proof by contradiction will lead to this theorem. In essence:

\begin{proof}
	Suppose there is an engine \textbf{X}  such that
	its efficiency exceeds that of the Carnot engine, so that
	$\eta_x > \eta_c$. Connect engine \textbf{X}
	to the Carnot engine in reverse, and expand out the inequality
	to get \[
		\frac{W}{Q^x_H} - \frac{W}{Q^C_H} > 0
	\] From the first law, we find
	\[
		W \equiv Q^C_H - Q^C_L = Q^x_H - Q^x_L
	\] We rearrange this into
	\[
		Q^C_H - Q^x_H = Q^C_L - Q^x_L
	\] The LHS is $> 0$ as part
	of our assumption, this means the RHS is also
	$> 0$. However, the LHS is heat
	\textit{dumped} from the hot reservoir, while the RHS is
	heat \textit{removed} into the cold reservoir. If cold
	reservoir is losing heat and hot reservoir is gaining heat. Then
	energy is being transferred from cold to hot. This is a direct
	contradiction with the Second Law (Clausius' statement). Thus,
	we must abandon our initial assumption---there cannot be such an
	engine \textbf{X}.
\end{proof}

This leads to the second theorem.

\begin{theorem}
	All reversible engines have exactly the Carnot efficiency.
\end{theorem}

It doesn't matter if we use an ideal gas or a non-ideal gas, if
an engine is reversible, then the theorem holds.

Proof by contradiction will show this holds.

\begin{proof}
	Suppose we now have an engine $R$ which is
	less efficient than the Carnot engine. Connect the Carnot engine
	in forward direction to engine $R$ in
	reverse. Like the previous proof, we have \[
		Q^C_H - Q^R_H = Q^C_L - Q^R_L
	\]
	Due to our assumption, both sides are negative. Once again, we
	have heat which is extracted from the hot reservoir and dumping
	it into the cold reservoir. This violates the second law, so we
	must abandon our assumption.
\end{proof}

The Carnot efficiency is exactly 1 if and only if
$T_L = 0$.

\begin{equation}
	\label{eq:carnot_efficiency_inequality}
	\eta_{c}
	=
	1 - \frac{T_L}{T_H} < 1
\end{equation}

\section{Thermodynamic absolute temperature}%
\label{sec:thermodynamic_absolute_temperature}

Kelvin used the Carnot relation in~\ref{eq:carnot_efficiency_inequality} to
define absolute temperature. This sets the
\textbf{zero} of the temperature scale.

Prior to 2019, the size of Kelvin was set by the triple point of
water. This all changed to the new S.I.\@ system which is based
on universal constants.

We use an ideal gas and ideal gas laws as a primary thermometer.

\chapter{Lecture 4 (Week 14)}%
\label{cha:lecture_4}

\section{Clausius' Theorem}%
\label{sec:clausius_theorem}

Results from a general analysis of the Carnot cycle. For each
part of the cycle by index $i$, calculate
\begin{equation}
	\sum_{i}\frac{\Delta Q_i}{T_i} =
	\frac{Q_H}{T_H} - \frac{Q_L}{T_L}
\end{equation}

For a Carnot cycle, the first term is equal to the second term,
thus the whole sum is 0. If we look at the small addition of
$Q$ then $\Delta Q \rightarrow \dbar Q$ then the
sum becomes an integral. We have.

\begin{equation}
	\oint \frac{\dbar Q_{rev}}{T} = 0
\end{equation}

\begin{definition}
	For a general cycle (either reversible or irreversible),
	Clausius' theorem states

	\begin{equation}
		\label{eq:clausius_theorem}
		\oint \frac{\dbar Q}{T} \leq 0
	\end{equation}
\end{definition}

We can prove this theorem through a set up of a generalised
cycle (multiple parts that are either reversible or
irreversible), as shown in Figure~\ref{fig:clausius_theorem_engine}.

Consider the following \textit{general} cycle:

\begin{figure}[htpb]
	\centering
	\includesvg[width=0.9\columnwidth]{./clausius_theorem_engine.svg}
	\caption{An engine with multiple sources of heat $\delta Q_i$.}%
	\label{fig:clausius_theorem_engine}
\end{figure}

\begin{derivation}
	Because this is a closed cycle, it follow the first law, where
	$\Delta U = 0$, or $W_s = \sum_i \delta Q_i$ where
	$W_s$ is the total work done by the system.

	Assuming each reservoir is connected to the master reservoir at
	the temperature $T_0$ via an individual Carnot
	engine, we can form a schematic as shown in
	Figure~\ref{fig:multiple_carnot_engines}.

	For each of the engine, there is a bit of work done, and the
	rest is heat that goes into our system. This shows that
	\begin{equation}
		\delta Q^0_i = \delta W_i + \delta Q_i
	\end{equation}

	If we sum over all the heat engines, we get
	\begin{equation}
		Q_0 = \sum_i \delta Q_i^0 = \sum_i \delta W_i + \sum_i \delta
		Q_i
	\end{equation}

	We can group the first sum in $W_C$ for the
	work done by the Carnot engines, and the second sum as
	$W_S$ for the work done by the system, as
	established at the start.
	\begin{equation}
		Q_0 = W_C + W_S
	\end{equation}

	Thus, the total heat that flows out from the master reservoir is
	the sum of the work done by the Carnot engine and the work done
	by the system.

	Effectively, this is the perfect conversion of heat into work.
	From the second law, we insist that
	\begin{equation}
		W_C + W_S \leq 0
	\end{equation}
	otherwise there would be a violation. This inequality comes from
	the fact that the work that is coming out of the Carnot engine
	must actually be negative, so that we are doing work
	\textit{on} these engines to convert heat from the
	master reservoir to work done by the system. If this quantity is
	negative, we expect that $W_S \leq W_C$ where both sides
	are positive, if this shows the magnitude of both terms, then it
	makes sense that the work we are doing on the system must either
	result in the same amount of work done by the system (perfect
	efficiency), or less than that. If $W_S > W_C$, this
	shows that the work we are doing on the system is resulting in
	more work than what we are putting in, which is forbidden.

	Therefore, $Q_0  = \sum_i \delta Q^0_i \leq 0$. Using the Carnot relation,
	which is that the ratio of the heat is equal to the ratio of the
	temperature, we get
	\begin{equation}
		T_0 \sum_i \frac{\delta Q_i}{T_C}  = \sum_i
		\frac{\delta Q_i}{T_C}\leq 0
	\end{equation}

	If we take the limit as each $\delta Q_i$ gets
	incredibly small or $i \rightarrow \infty$, the temperature of
	the reservoir becomes the temperature of the system. We arrive
	at
	\begin{equation}
		\oint \frac{dQ}{T} \leq 0
	\end{equation}
\end{derivation}

\begin{figure}[htpb]
	\centering
	\includesvg[width=0.9\columnwidth]{./multiple_carnot_engines.svg}
	\caption{A master reservoir connected to the system via multiple  Carnot
		engines.}%
	\label{fig:multiple_carnot_engines}
\end{figure}

We have proved that any valid system, reversible or
irreversible, must obey this inequality.

\section{Entropy}%
\label{sec:entropy}

\begin{definition}
	For a reversible closed cycle, the LHS expression
	in~\ref{eq:clausius_theorem} is exactly equal to 0, which
	indicates that it is a state function. We write this state
	function as entropy $S$

	\begin{equation}
		\begin{aligned}
			\label{eq:entropy}
			\dd{S}  & = \frac{\dbar Q_{\mathrm{rev}}}{T} \\
			T\dd{S} & = \dbar Q_{\mathrm{rev}}
		\end{aligned}
	\end{equation}
\end{definition}

Like any state function, the difference of entry at two points
is just the integral evaluated at those two points.

This is the thermodynamic definition of entropy. Later, we will
look at the statistical interpretation in terms of disorder.

We can directly measure $S$ as a function of
temperature if we keep V the same. Taking the partial derivative
of $T$ on each side, we arrive at

\begin{equation}
	C_V = {\left(\pdv{Q}{T}\right)}_V =
	T{\left(\pdv{S}{T}\right)}_V
\end{equation}

\begin{definition}
	Entropy is expressed as a function of temperature as
	\begin{equation}
		\label{eq:entropy_from_temperature}
		S(T) = \int_0^T\frac{C(T)}{T}\dd{T}
	\end{equation}
\end{definition}

In other words, entropy is the area under a
$(C/T)$-$T$ curve.

Substitute the differential of entropy into the first law,
expressed by equation~\ref{eq:first_law}, we can also
write the first law for as

\begin{equation}
	\label{eq:first_law_entropy}
	\dd{U} = T\dd{S} -
	p\dd{V}
\end{equation}

This applies for irreversible changes as well as reversible
processes. How can we justify this?

For reversible changes,\, $\dbar W = -p\dd{V}$ and\,
$\dbar Q = T \dd{S}$, while for irreversible changes\,
$\dbar W \geq -p\dd{V}$ and\, $\dbar Q \leq T\dd{S}$. If we add
the irreversible terms together it all balances out and we still
get $\dd{U}$. Another way of looking at it is
that~\ref{eq:first_law_entropy} is composed of only state
function,s thus it must be valid for all changes, independent of
path.

\section{Irreversibility---entropy cannot decrease}%
\label{sec:irreversibility}

We can prove that entropy cannot decrease. Consider a cyclic
process which has irreversible and reversible parts, as shown in
Figure~\ref{fig:entropy_cannot_decrease}.

\begin{figure}[htpb]
	\centering
	\includesvg[width=0.9\columnwidth]{./entropy_cannot_decrease.svg}
	\caption{A cyclic process that has irreversible process from A to B and
		reversible process from B to A.}%
	\label{fig:entropy_cannot_decrease}
\end{figure}

\begin{derivation}

	From Clausius' theorem in Equation~\ref{eq:clausius_theorem},

	\begin{equation}
		\begin{aligned}
			\int_A^B \frac{\dbar Q}{T} + \int_B^A
			\frac{\dbar Q_{\mathrm{rev}}}{T}          & \leq 0        \\
			\int_A^B \frac{\dbar Q_{\mathrm{rev}}}{T} & \geq \int_A^B
			\frac{\dbar Q}{T}
		\end{aligned}
	\end{equation}

	If we let A tend to B so the points become infinitesimally close
	together, we have the differential form instead, substitute in
	$\dd{S}$, we get that for an
	\textit{isolated system}, where\, $\dbar Q = 0$

	\begin{equation}
		\label{eq:entropy_cannot_decrease}
		\dd{S} \geq \frac{\dbar Q}{T} \equiv
		\dd{S} \geq 0
	\end{equation}

	This means for an isolated system, entropy
	\textbf{cannot} decrease.
\end{derivation}

\section{The thermodynamic arrow of time}%
\label{sec:the_thermodynamic_arrow_of_time}

Hence, the entropy of the universe must increase with time,
since the universe is an isolated system. Increasing entropy is
synonymous with increasing ``time''. We can consider this to be
the thermodynamic arrow of time.

Later, we will consider the statistical interpretation of
entropy.

\section{Joule expansion}%
\label{sec:joule_expansion}

Suppose we have two chambers that are thermally isolated. They
are connected by a tube with a valve. Initially, all the gas is
in the left chamber. When we suddenly open the valve, the change
is \textit{irreversible}, so the pressure and volume are not
well-defined during the expansion. This is called a
\textit{free expansion} so the gas \textbf{does not do work on the surroundings.}

We now ask: What is the change in entropy when the system
reaches its final state (it is in internal equilibrium)?

We don't know specifically the path, it is not well-defined.
What we know, however, is the start and end points. Since
entropy is a state function, we fortunately only need the start
and end points. Effectively, we can use a well-defined path
which would give us the same answer since we are dealing with
path-independent functions.

We choose a reversible isothermal path. The path are different
but if pressure and volume are the same at the start and end
points, it will be equivalent to the irreversible path.

Since the path is isothermal, $U = U(T) = 0$ so from
the first law and the ideal gas relations, we have
\begin{equation}
	\dd{S} = nR\frac{\dd{V}}{V}
\end{equation}

We can further integrate this to give us a simple result.

\begin{definition}
	In the \textbf{isothermal} case, the change in entropy is
	given as
	\begin{equation}
		\label{eq:isothermal_entropy_change}
		\Delta S = nR\naturallogarithm{\frac{V_1}{V_0}}
	\end{equation}
\end{definition}

We only calculated this for a system---the change in the entropy
of the universe is not necessarily the same for the reversible
and irreversible expansion.

\chapter{Lecture 5 (Week 14)}%
\label{cha:lecture_5}

\section{Thermodynamic Potentials}%
\label{sec:thermodynamic_potentials}

We can combine state functions to make another state function.
There are infinitely many possible combinations but only a few
are useful. Here are several examples:
\begin{center}
	\begin{tabular}{l l}
		$U$               & Internal energy       \\
		$F = U - TS$      & Helmholtz free energy \\
		$G = U - TS + pV$ & Gibbs free energy     \\
		$H = U +pV$       & Enthalpy              \\
	\end{tabular}
\end{center}

These definitions are only correct where the work term\,
$\dbar W = -p \dd{V}$. Also, we assume that it is a closed
system with a fixed number of particles (gas cannot leave the
system). We will now look at each of the following examples in
turn.

\section{Internal energy}%
\label{sec:internal_energy}

Looking at~\ref{eq:first_law_entropy}, we see that
$U = U(S,V)$. We call $S$ and
$V$ as the natural variables of
$U$. Recall from
Definition~\ref{def:exact_differentials}, we can find this for
$U$ since it is a state function.

Comparing the exact differential and the equation expressing the
first law, we find that

\begin{equation}
	T = {\left(\pdv{U}{S}\right)}_V\,\,\text{and}\,\, p
	= -{\left(\pdv{U}{V}\right)}_S
\end{equation}

\section{Helmholtz Free energy}%
\label{sec:helmholtz_free_energy}

Using the first law, we can expand the equation for the
Helmholtz free energy. We find that
\begin{equation}
	\label{eq:helmholtz_free_energy}
	F = F(V,T)\,\, \text{and}
	\,\,\dd{F} = -p\dd{V} -
	S\dd{T}
\end{equation}
where its natural variables are $V$ and
$T$.

Its exact differential leads to the results that
\begin{equation}
	p = -{\left(\pdv{F}{V}\right)}_T\,\, \text{and}
	\,\, S = -{\left(\pdv{F}{T}\right)}_V
\end{equation}
This is particular useful because $V$ and
$T$ are often common experiment conditions.
The system will evolve to minimise F at constant
$V$ and $T$. This means
as the system evolves, it will do work on the environment in
order to minimise $F$, so that
$F$ is the maximum work you can get out of a
system (hence ``free'' energy).

\section{Gibbs Free energy}%
\label{sec:gibbs_free_energy}

Doing similar steps as before, we find that
\begin{equation}
	\label{eq:gibbs_free_energy}
	G = G(p,T)\,\, \text{and} \,\,
	\dd{G} = -S\dd{T} +
	V\dd{p}
\end{equation}
Its exact differentials show that

\begin{equation}
	S = -{\left(\pdv{G}{T}\right)}_p\,\, \text{and}\,\,
	V = {\left(\pdv{G}{p}\right)}_T
\end{equation}

Same with F, the system will evolve to minimise G. So G is the
maximum work you can get out of the system with constant
$p$, and $T$

\section{Enthalpy}%
\label{sec:enthalpy}

Doing similar steps as before, we find that
\begin{equation}
	H = H(S,p)\,\, \text{and} \,\,
	\dd{H} = T\dd{S} +
	V\dd{p}
\end{equation}
Its exact differentials show that

\begin{equation}
	T = {\left(\pdv{H}{S}\right)}_p\,\, \text{and}\,\,
	V = {\left(\pdv{H}{p}\right)}_S
\end{equation}

Note that constant $S$ is the same as
constant $Q$, the cycle is adiabatic.
Enthalpy as a quantity is useful because if you have an
adiabatic change at constant pressure, $\Delta H$
is the heat absorbed by the system at constant
$p$ (e.g., a chemical reaction). At constant
pressure, $\dd{H} = \dd{Q}$.

\section{Maths of partial derivatives}%
\label{sec:maths_of_partial_derivatives}

Proofs aren't necessarily, but these following identities are
extremely useful.

\begin{definition}
	Below are some partial derivative identities:
	\begin{enumerate}
		\item Chain rule (all terms must have the same constraints)
		      \begin{equation}
			      {\left(\pdv{z}{x}\right)}_y ={\left(\pdv{z}{y}\right)}_y {\left(\pdv{y}{x}\right)}_y
		      \end{equation}
		\item Reciprocal theorem
		      \begin{equation}
			      {\left(\pdv{z}{x}\right)}_y = \frac{1}{{\left(\pdv{x}{z}\right)}_y}
		      \end{equation}
		\item Reciprocity theorem or triple product rule
		      \begin{equation}
			      {\left(\pdv{x}{y}\right)}_z	     {\left(\pdv{y}{z}\right)}_x
			      {\left(\pdv{z}{x}\right)}_y	      = -1
		      \end{equation}
		\item Convert an exact differential into a partial derivative into a
		      partial one
		      \begin{equation}
			      \dd{S} =
			      {\left(\pdv{S}{T}\right)}_V\dd{T} +
			      {\left(\pdv{S}{V}\right)}_T\dd{V}
			      \,\Rightarrow\,
			      {\left(\pdv{S}{T}\right)}_p
				      = {\left(\pdv{S}{T}\right)}_V +
			      {\left(\pdv{S}{V}\right)}_T
			      {\left(\pdv{V}{T}\right)}_p
		      \end{equation}
		\item Maxwell relations (exchange derivatives w.r.t.\ natural
		      variables). See~\ref{sub:maxwell_relations}.
	\end{enumerate}
\end{definition}

\subsection{Pseudo-proofs}%
\label{sub:pseudo_proofs}

These proofs are not presented explicitly. They are meant to
help understand the identities in~\ref{sec:maths_of_partial_derivatives}.

To prove the \textbf{reciprocal} and
\textbf{reciprocity} theorems, take a function
$f(x,y,z) = 0$, with $x = x(y,z)$,
$y = y(x,z)$, and $z = (x,y)$. If we
write $x$ and $z$ in
differential forms, and substitute $\dd{x}$ into
the expression for $\dd{z}$, we notice that we
can equate the coefficients of $\dd{z}$, which
shows the \textbf{reciprocal} theorem holds.

Then if we look at the coefficients of $\dd{y}$,
we can rearrange and use \textbf{reciprocal} theorem to
show the \textbf{reciprocity} theorem.

\subsection{Maxwell relations}%
\label{sub:maxwell_relations}

To demonstrate the Maxwell relations, let's use the Helmholtz
relation, see Section~\ref{sec:helmholtz_free_energy} for the
differential forms.

If we take the derivatives again with respect to other natural
variables,
\begin{equation}
	{\left(\pdv{S}{V}\right)}_T =
	-{\left(\pdv{F}{V}{T}\right)}_{V,T}\,\, \text{and}
	\,\,{\left(\pdv{p}{T}\right)}_V = {\left(\pdv{F}{T}{V}\right)}_{V,T}
\end{equation}
we can see that
\begin{equation}
	{\left(\pdv{S}{V}\right)}_T = {\left(\pdv{p}{T}\right)}_V
\end{equation}

This is a Maxwell relation. We don't have to memorise all the
Maxwell equations, but we can easily derive them when needed.

\chapter{Lecture 6 (Week 14)}%
\label{cha:lecture_6}

\section{Cooling gases and Joule-Thomson expansion}%
\label{sec:cooling_gases_joule_thomson_expansion}

There are several ways to cool gases. One way is to expand gas
\textit{adiabatically} and make it do work on the environment.
This is different from a Joule expansion, where the gas expands
freely. A Joule expansion does not do work on the environment.
We can either do this using a piston and a flywheel
(Figure~\ref{fig:piston_flywheel}) or by using a turbine (turbo
expander) (Figure~\ref{fig:turbine_cool_gas}).

\begin{figure}[htpb]
	\centering
	\includesvg[width=0.9\columnwidth]{./piston_flywheel.svg}
	\caption{We open the high pressure valve, gas comes in the chamber and moves the piston up (doing work on
		the piston), the flywheel spins with friction to remove energy and once the
		piston reaches the highest position, it moves down. The low pressure valve is opened and low pressure
		gas is forced out by the lowering piston.}%
	\label{fig:piston_flywheel}
\end{figure}

\begin{figure}[htpb]
	\centering
	\includesvg[width=0.9\columnwidth]{./turbine_cool_gas.svg}
	\caption{High pressure gas comes in, drives the turbine by doing work on it, heat is dissipated through friction, and then low pressure gas comes out.}%
	\label{fig:turbine_cool_gas}
\end{figure}

Adiabatic expansion is quite inefficient at low temperature. A
better alternative is the \textit{Joule-Thomson} or
\textit{Joule-Kelvin} expansion (JT or JK process). See
Figure~\ref{fig:joule_thomson_expansion}.

\begin{figure}[htpb]
	\centering
	\includesvg[width=0.9\linewidth]{./joule_thomson_expansion.svg}
	\caption{High pressure gas comes in and goes the rough a porous plug
		which removes a lot of energy through
		friction, low pressure gas comes out of the other end and is pumped back
		into the engine. The pressure difference is maintained with a
		compressor. This principle of cooling is used in many devices
		like refrigerators, or helium liquefiers.}%
	\label{fig:joule_thomson_expansion}
\end{figure}

We assume the JT process is adiabatic---high pressure gas comes
through and expands to the other side without any heat exchange.

\subsection{Conservation of enthalpy}%
\label{sub:conservation_of_enthalpy}

We might want to find the work done as gas moves through the
plug in the centre.

Imagine, for the sake of representation, there are piston on
both sides of the engine. The left piston maintains pressure
$p_1$ as it forces gas through the plug, while
the right piston moves to the right at $p_1$
as gas moves through plug. Force is $F = pA$,
therefore:

\begin{tabular}{l l}
	The work done on gas to move distance $L_1$ in the left chamber  & $W = p_1 A L_1 = p_1 V_1$   \\
	The work done on gas to move distance $L_2$ in the right chamber & $W = P_2 A L_2$ = $p_2 V_2$ \\
\end{tabular}\\

\begin{definition}
	The net work done on the gas in a Joule-Thomson expansion is
	\begin{equation}
		\label{eq:joule_thomson_expansion_work}
		\Delta W = p_1 V_1 - p_2 V_2
	\end{equation}
\end{definition}

As the gas moves through the plug, there is a change in internal
energy, $\Delta U = U_2 - U_1$. From the first law, for an
adiabatic process (no heat transfer), $\Delta U = \Delta W$.
We know $\Delta W$ from~\ref{eq:joule_thomson_expansion_work},
so
\begin{equation}
	p_1 V_1 + U_1 = p_2 V_2 + U_2
\end{equation}

This is the definition of \textit{enthalpy} which we
mentioned in Section~\ref{sec:enthalpy}, which means
$H_1 = H_2$. In other words, enthalpy is conserved
in a JT expansion.

\subsection{The Joule-Thomson coefficient}%
\label{sub:the_joule_thomson_coefficient}

We can quantify how much cooling takes place by defining a
coefficient, called the Joule-Thomson coefficient as

\begin{equation}
	\label{eq:joule_thomson_coefficient_1}
	\mu_{\mathrm{JT}}
	=
	{\left(\pdv{T}{p}\right)}_H
\end{equation}

Since the constant $H$ is difficult to use
so we can derive an expression at constant pressure instead. We
can use what we know from previous lectures
and~\ref{sec:maths_of_partial_derivatives} to get to this.

\begin{derivation}
	From the reciprocity theorem, we can write
	\begin{equation}
		{\left(\pdv{T}{p}\right)}_H
		{\left(\pdv{p}{H}\right)}_T
		{\left(\pdv{H}{T}\right)}_P = -1
	\end{equation}
	Substitute this into our definition for the JT coefficient so
	far to get
	\begin{equation}
		\mu_{\mathrm{JT}}
		=
		-\frac{1}{{\left(\pdv{p}{H}\right)}_T
		{\left(\pdv{H}{T}\right)}_p}
	\end{equation}
	We then borrow our results from Section~\ref{sec:enthalpy}
	the differential form of enthalpy, divide by
	$\dd{T}$ at constant $p$ to get
	\begin{equation}
		{\left(\pdv{H}{T}\right)}_p =
		T{\left(\pdv{S}{T}\right)}_p =
			{\left(\pdv{Q}{T}\right)}_p = C_p
	\end{equation}
	We also divide by $\dd{p}$ at constant
	$T$
	\begin{equation}
		{\left(\pdv{H}{p}\right)}_T =
		T{\left(\pdv{S}{p}\right)}_T + V
	\end{equation}
	Substitute back into what we got from reciprocity to get
	\begin{equation}
		\mu_{\mathrm{JT}}
		=
		-\frac{1}{C_p}\left[T{\left(\pdv{S}{p}\right)}_T + V\right]
	\end{equation}
	We use a Maxwell relation to make this nicer
	\begin{equation}
		{\left(\pdv{S}{p}\right)}_T = -{\left(\pdv{V}{T}\right)}_p
	\end{equation}
\end{derivation}

We can then use this for our final expression:
\begin{definition}
	The JT coefficient at constant pressure is
	\begin{equation}
		\label{eq:joule_thomson_coefficient}
		\mu_{\mathrm{JT}}
		=
		\frac{1}{C_p}\left[T{\left(\pdv{V}{T}\right)}_p-V\right]
	\end{equation}
\end{definition}

To get cooling we need $\mu_{\mathrm{JT}} > 0$. For an ideal
gas, $\mu_{\mathrm{JT}} = 0$, so there's never any cooling. But
for a real gas, this varies with temperature. We call the
temperature below which you will get cooling via the JT process
the \textbf{inversion temperature}.

\begin{definition}
	The inversion temperature $T_i$ for a JT
	process is
	\begin{equation}
		{\left(\pdv{V}{T}\right)}_p = \frac{V}{T_i}
	\end{equation}
	Below this temperature, we can cool gas using the JT process,
	beyond this temperature, we can use any adiabatic expansion to
	get to $T_i$.
\end{definition}

There are JT coefficient curves at various temperatures for
different gases.

\subsection{Helium liquefier (not examined)}%
\label{sub:helium_liquefier}

From a gas storage, the gas travels through a compressor. Next,
it goes through a heat exchanger, then through the Turbine 1,
then through another heat exchanger and finally through Turbine
2. It is then pushed back around the circuit once again, in
order to cool down other incoming gases (thus it functions as
the heat reservoir in the heat exchangers). It goes through this
step until the final low-pressure gas that leaves Turbine 2
cools from \si{300\kelvin} to \si{10\kelvin}.

At the same time, the compressor also has another stream coming
into the liquefier, but only going through the heat exchanger
and not through the turbines (the first stream is used to
maintain heat exchange for this stream), after it leaves this,
it has the same temperature as the other stream, but maintains a
high-pressure.

After that this high-pressure gas enters the JT process to
become liquefied.

\chapter{Lecture 7 (Week 15)}%
\label{cha:lecture_7}

\section{Chemical potential}%
\label{sec:chemical_potential}

Consider 2 systems which can exchange heat and particles. When
we add particles, we change the energy so we need to modify the
first law. In a similar way to~\ref{cha:lecture_5}, we
notice the natural variables when we write the expression in
differential form.

\begin{equation}
	\label{eq:first_law_chemical_potential}
	\dd{U} = T\dd{S} -
	p\dd{V} + \mu\dd{N}
\end{equation}

If we find the exact differential of $U$, we
notice the new natural variable $N$, gives
us a coefficient, $\mu$. This is the chemical
potential, it is given by
\begin{equation}
	\label{eq:chemical_potential}
	\mu
	=
	{\left(\pdv{U}{N}\right)}_{S,V}
\end{equation}

We add this extra potential to all the other equations we have
looked at for potentials and energies. Thus, we end up with the
following equations.

\begin{equation}
	\mu
	=
	{\left(\pdv{U}{N}\right)}_{S,V} +
	{\left(\pdv{F}{N}\right)}_{T,V} +
	{\left(\pdv{G}{N}\right)}_{T,p}+
	{\left(\pdv{H}{N}\right)}_{S,p}
\end{equation}

However, these equations are not all the same. While
$V,S$ are extensive and scale with the size of
the system, $p,T$ are intensive and do not
scale.

Thus, only $G$ scales with the size of the
system, since its intensive variables are kept constant.

\begin{definition}
	For a homogeneous system, $G$ is proportional
	to $N$, and we have the relation
	\begin{equation}
		\mu
		=
		\frac{G}{N}
	\end{equation}
\end{definition}

Therefore, an alternative definition of the chemical potential
is the \textit{Gibbs} free energy per particle for an
homogeneous system.

\subsection{Physical meaning of chemical potential}%
\label{sub:physical_meaning_of_chemical_potential}

Consider a box with a sliding partition which is porous and
thermally conductive. The two sides can exchange heat and
particles. Both partitions are isolated from the environment.
See Figure~\ref{fig:chemical_potential_box}.

\begin{figure}[htpb]
	\centering
	\includesvg[width=0.9\linewidth]{./chemical_potential.svg}
	\caption{A movable partition separating A and B}%
	\label{fig:chemical_potential_box}
\end{figure}

We can establish a few relations:

\begin{equation}
	\begin{aligned}
		\dd{V}_A & = -\dd{V}_B \\
		\dd{N}_A & = -\dd{N}_B \\
		\dd{U}_A & = -\dd{U}_B \\
	\end{aligned}
\end{equation}

Since $U = U(S,V,N)$, $S = S(U,V,N)$, the
exact differential of $S$ would be

\begin{equation}
	{\left(\pdv{S}{U}\right)}_{V,N}\dd{U} +
	{\left(\pdv{S}{V}\right)}_{V,N}\dd{V} +
	{\left(\pdv{S}{N}\right)}_{U,v}\dd{N}
\end{equation}

Rearrange the first law with the chemical potential for
$dS$. We recognise that the total entropy the
entropy of A plus the entropy of B. We can combine the relations
derived above and get
\begin{equation}
	\dd{S} = \dd{S}_A +
	\dd{S}_B =
	\left(\frac{1}{T_A} - \frac{1}{T_B}\right)\dd{U}_A +
	\left(\frac{P_A}{T_A} - \frac{P_B}{T_B}\right)\dd{V}_A
	-\left(\frac{\mu_A}{T_A} - \frac{\mu_B}{T_B}\right)\dd{N}_A
\end{equation}

At equilibrium, the change in entropy $\dd{S}$ is
0. This means we know at this state, $T_A = T_B$,
$P_A = P_B$ and $\mu_A = \mu_B$. This is a
simple case. We will now look at special cases.

\subsection{Heat flow}%
\label{sub:heat_flow}

Now, the partitions are fixed, and the holes are closed. We get
rid of the $\dd{V}$ and $\dd{N}$
terms. In other words, we only have heat flow. Since the system
evolves, and it is not in equilibrium $\dd{S} > 0$.
Thus, if $T_B > T_A$ then $\dd{U}_A > 0$.
All that means if partition B is hotter than partition A, then
heat flows from B to A.

\subsection{Particle flow}%
\label{sub:particle_flow}

If we fix the partition again, but let the holes open, then
temperature will remain the same for both partition. So we are
left with only the $\dd{N}$ term. Apply the same
logic as the first case of heat flow, then we conclude that if
chemical potential in B is higher than in A than the particles
flow from B to A.

\section{Gibbs-Duhem equation}%
\label{sec:gibbs_duhem_equation}

Using Equation~\ref{eq:chemical_potential} we can write
\begin{equation}
	\dd{G} = \mu \dd{N} +
	N\dd{\mu}
\end{equation}

If we combine this with~\ref{eq:gibbs_free_energy}, we get the
\textit{Gibbs-Duhem} equation.
\begin{equation}
	N\dd{\mu} = V\dd{p} -
	S\dd{T}
\end{equation}

\begin{definition}
	The Gibbs-Duhem equation in terms of intensive variables is
	\begin{equation}
		\label{eq:gibbs-duhem}
		\dd{\mu} =
		\hat{v}\dd{p} -
		\hat{s}\dd{T}
	\end{equation}
	where $ \hat{v} = V/N $ and $ \hat{s} = S/N $, which
	are specific volumes and specific entropy.
\end{definition}

\section{Clausis-Clapeyron equation}%
\label{sec:clausis_clapeyron_equation}

This equation is used to analyse the following situation.
Consider a system where two phases of matter coexist. There is a
first order discontinuous transition between the two phases with
a latent heat. A common example would be a gas/liquid transition
in $\mathrm{H}_2\mathrm{O}$. See Figure~\ref{fig:clausis-clapeyron}.

\begin{figure}[htpb]
	\centering
	\includesvg[width=0.9\columnwidth]{./clausis-clapeyron.svg}
	\caption{A system with pressure and temperature on the phase coexisting
		line. At any other $(p,T)$ that does not lie on
		this line, there cannot be coexistence of the two phases.}%
	\label{fig:clausis-clapeyron}
\end{figure}

Now, consider a single point on the line at fixed
$(p,T)$. The total Gibbs energy is the sum of
that in each phase.
\begin{equation}
	\dd{G} = \dd{G}_1 +
	\dd{G}_2
\end{equation}

In equilibrium, $\dd{G} = 0$ so using the expression
for chemical potential in~\ref{eq:chemical_potential}, we can
establish that
\begin{equation}
	\dd{G} = \mu_1 \dd{N}_1 +
	\mu_2 \dd{N}_2 = 0
\end{equation}

Because we have a closed system, $\dd{N}_1 = -\dd{N}_2$, thus,
this proves that along the phase co-existence line
$\mu_1 = \mu_2$. Now we can use the Gibbs-Duhem equation
(~\ref{eq:gibbs_free_energy}), we have
\begin{equation}
	\begin{aligned}
		\hat{v}_1\dd{p} - \hat{s}_1\dd{T} & = \hat{v}_2\dd{p} - \hat{s}_1 \dd{T}                      \\
		\dv{p}{T}                         & = \frac{ \hat{s}_1 - \hat{s}_2 }{ \hat{v}_1 - \hat{v}_2 }
	\end{aligned}
\end{equation}

So the slope $\dv{p}{T}$ is related to the
difference in specific entropy and specific volumes of the two
phases.

We can also write the equation in terms of latent heat (the heat
absorbed going from phase 1 to 2). Latent heat is expressed as
\begin{equation}
	\hat{l} = \frac{L}{N} =
	\frac{\Delta Q}{N} = \frac{T}{N} \Delta S =
	\frac{T}{N}( S_2 - S_1 ) = T( \hat{s}_2 -
	\hat{s}_1 )
\end{equation}

Thus, our slope is now
\begin{equation}
\end{equation}

\begin{definition}
	Therefore, we can write the Clausis-Clapeyron equations in two
	different ways:
	\begin{equation}
		\label{eq:clausis-clapeyron-equations}
		\dv{p}{T} =
		\begin{cases}
			( \hat{s}_1 - \hat{s}_2 )/( \hat{v}_1 - \hat{v}_2 ) \\
			{ \hat{l} }/{T ( \hat{v}_2 - \hat{v}_1 )}
		\end{cases}
	\end{equation}
	The signs of L has to correspond to the change in volume.
\end{definition}

\section{Phase diagrams}%
\label{sec:phase_diagrams}

For most materials, the density of the solid phase is higher
than the liquid phase. However, for $\mathrm{H}_2\mathrm{O}$, the
opposite is true due to hydrogen bonding in ice.

Relevantly, 2 phases exist along the $p\mathrm{-}T$
line but 3 phases exist at the triple point, which is a single
point of intersection between the different lines of coexisting
phases.

\begin{example}
	\textbf{How can we melt ice with pressure?}
	How much pressure do I need to liquefy ice at
	\num{-5} \unit{\celsius}? Looking at the
	phase diagram of water, we can draw a line from
	\num{-5} \unit{\celsius} to the line of
	coexistence (that's the critical point at which ice starts to
	melt), and find what temperature this is. To do this, we first
	must know the slope $\dv{p}{T}$.
	\begin{center}
		\includesvg[width=0.9\columnwidth]{./ice_melting.svg}%
		\captionof{figure}{Phase diagram with a coexisting line between solid and liquid
			phases.}
	\end{center}

	We know Clausis-Clapeyron equations from~\ref{eq:clausis-clapeyron-equations}
	and we also know about some quantities about water,
	$ \hat{l} = \SI{3.3e5}{\joule \per\kilogram}$. The specific volume
	$ \hat{v} $, when using kilograms, is inversely
	proportional to the density, so that
	\begin{equation}
		\Delta \hat{v}
		=
		\frac{1}{\rho_\mathrm{water}} - \frac{1}{\rho_\mathrm{ice}}
	\end{equation}
	We also know the corresponding densities at
	\SI{0}{\celsius}. This gives us a value for
	$\dv{p}{T}$. We need $\Delta p = \dv{p}{T} \Delta T$. The
	answer is about \SI{68}{\mega\pascal}. This is equivalent to a
	person who weights \SI{80}{\kilogram} wearing ice-skating
	shoes of length \SI{20}{\centi\metre} with width
	\SI{60}{\micro\metre}.
\end{example}

\chapter{Lecture 8 (Week 15)}%
\label{cha:lecture_8}

\section{The Third Law}%
\label{sec:the_third_law}

Walter Nernst (1906): \textit{At absolute zero all reactions take place with no change in
	entropy}. In other words:
\begin{equation}
	\Delta G - \Delta H \rightarrow 0
	\,\,\mathrm{as}\,\, T \rightarrow 0
\end{equation}

\begin{definition}
	The third law is stated by Max Planck (1911) as:

	\textit{Entropy of all systems in internal equilibrium tends to a
		constant at absolute zero, and so may be taken at zero.}
\end{definition}

Although note that not everything is in \textit{internal}
equilibrium, e.g., a glass. It is a disordered solid. It does
not have a crystallise structure but rather a random
arrangement. It would take a very long time to come into
internal equilibrium (a meta-state).

Remember that we can calculate entropy by integrating the
specific heat, as stated in Equation~\ref{eq:entropy_from_temperature}.

\subsection{Consequences of the third law}%
\label{sub:consequences_of_the_third_law}

\begin{enumerate}
	\item Any system must have only one available state at
	      $T = 0$, in other words, there must be no
	      degenerate ground states. $S = 0$ at
	      $T = 0$.
	\item Heat capacity must go to zero at $T = 0$.
	      $\naturallogarithm \rightarrow -\infty$ as $T \rightarrow 0$.
	      \begin{equation}
		      C = \dv{Q}{T} = T\dv{S}{T} =
		      \dv{S}{\naturallogarithm{T}}
	      \end{equation}
	\item It is impossible to cool something to $T=0$ in
	      a finite number of steps.
\end{enumerate}

From these consequences, we see that for an ideal gas where
$C = \frac{3}{2}k_{B}N$. This is inconsistent with the third
law. When we look later at Fermi gas, or a crystal lattice,
these are all $T$-dependent. Thus, they are
consistent with the third law.

\begin{example}
	Let's apply the third law for a magnetic system. Where
	\begin{equation}
		\dbar W = B\dd{m}
	\end{equation}
	Since $B$ replaced $p$
	and $m$ replaced $V$, we
	can use the relations in Section~\ref{sec:gibbs_free_energy} to
	arrive at
	\begin{equation}
		S = {\left(\pdv{G}{T}\right)}_B\,\, \text{and}\,\,
		m = - {\left(\pdv{G}{B}\right)}_T
	\end{equation}

	Taking the second derivative to get the Maxwell equation
	\begin{equation}
		{\left(\pdv{S}{B}\right)}_T
			=
			{\left(\pdv{m}{T}\right)}_B
	\end{equation}

	We can use Curie's Law which states that the susceptibility,
	$\chi$ of a paramagnet is inversely
	proportional to $T$, where
	$\chi$ is
	\begin{equation}
		\chi = \frac{M}{\mu_0 B} = \frac{m}{\mu_0 vB}
	\end{equation}

	Equate the two and we get that
	\begin{equation}
		{\left(\pdv{S}{B}\right)}_T \equiv
		\pdv{m}{T} \propto -\frac{1}{T^2}
	\end{equation}

	But at $T = 0$, $S = 0$ thus
	Curie's Law must break down at absolute zero.

	We can explore this by looking at magnetic cooling with a
	paramagnetic solid. For a paramagnet, the magnetic field causes
	spins to align with field, thus lowering entropy. The higher the
	field, the more aligned the spins are.

	If we start a low $B$ state, we magnetise it
	isothermally (increase $B$) by connecting it
	to a cold bath, this decreases the entropy. If we release the
	switch and turn off the field, then the system cools down
	adiabatically. We end up again at low $B$.
	We essentially just cooled down the paramagnet we started with,
	with $B$ unchanged. But both curves get
	closer and closer together, so we need an infinite number of
	steps to reach 0. See the curves below

	\centering
	\includesvg[width=0.8\columnwidth]{./paramagnet_cooling.svg}%
	\label{fig:cooling_paramagnets} \captionof{figure}{$S\mathrm{-}T$ curves for low and high
		$B$.}
\end{example}

\chapter{Lecture 9 (Week 15)}%
\label{cha:lecture_9}

\section{Introduction}%
\label{sec:int}

Now that we have covered Classical Thermal Physics, we will look
at Statistical Mechanics

Statistical mechanics involve statistical averages of properties
of the system, such as energy. We are expected to work out the
set of possible configurational states of the system, calculate
the probability distribution of these states, and calculate the
macroscopic thermodynamic properties $(p, S, U)$
from the probability distribution.

\section{Counting states}%
\label{sec:counting_states}

A microstate is the number of permutations. It is similar to
asking for the number of ways of putting $n$
balls into $r$ boxes (permutations). It
concerns \textit{individual} arrangement. Whether the balls
are identical or different, or that $r \neq n$,
will affect our formula for calculating the answers.

A macrostate is the number of balls and boxes. It is the overall
picture of the system.

\begin{example}
	Electrons have spin which can either be up or down. If we have
	$N = 4$ lattice sites, then this is our
	macrostate. Let's say we have 4 electrons, two are up and two
	are down, $N_u = 2$ and $N_d = 2$.
	There are 6 distinguishable microstates. We get this through the
	following formula.
	\begin{equation}
		\Omega = \frac{N!}{N_d! N_u!}
	\end{equation}
	There are $N$ electrons and
	$N$ sites, that means there are
	$N!$ ways of rearranging them. However, since
	all the $N_u$ are identical and all the
	$N_d$ are identical, we divide by
	$N_u! N_d!$

	Later, we find that $S = k_B \naturallogarithm{\Omega}$. So if we know
	$\Omega$, we can find $S$.

\end{example}

\section{Stirling's formula}%
\label{sec:stirling_s_formula}

We often work with $\naturallogarithm{\Omega}$ so approximations are
useful, note that the second one is better than the first one.

\begin{definition}
	Stirling's approximations for $\naturallogarithm{n}!$ are
	\begin{equation}
		\begin{aligned}
			\naturallogarithm{n}! & \simeq n\naturallogarithm{n} - n,\,\,\,\, n \gg 1                                        \\
			\naturallogarithm{n}! & \simeq n\naturallogarithm{n} - n + \frac{1}{2}\naturallogarithm{2\pi n},\,\,\,\, n \gg 1
		\end{aligned}
	\end{equation}
\end{definition}

\section{Statistical Ensembles}%
\label{sec:statistical_ensembles}

\begin{definition}
	An ensemble is a large number of virtual copies of a system
	which are all the same but configured in random ways.

	An ensemble average is then the average state of all these
	virtual systems.

	An statistical ensemble is then a probability distribution of
	all possibles states of the system
\end{definition}

Averages are calculated by two following ways (we assume they
give the \textit{same} answer)

\begin{enumerate}
	\item Take a \textbf{single} system and follow it over a long
	      time, so that the system goes through all possible
	      configurations.
	\item Take a large number of identical (but randomly configured)
	      systems and average over these at a single snapshot in time.
	      This is called the ensemble average.
\end{enumerate}

\subsection{Types of Ensemble}%
\label{sub:types_of_ensemble}

Gibbs categorised three types of ensembles which refer to a
system and how it interact with the surroundings. See
Figure~\ref{fig:ensemble_types} for diagrams.

\begin{enumerate}
	\item \textbf{Micro-canonical ensemble:} This is an isolated system with fixed
	      $U, V, N$. We can apply this ensemble to a whole
	      gas as an isolated system.
	\item \textbf{Canonical (standard) ensemble:} System is in contact with a thermal
	      reservoir. $V, N, T$ fixed for the system. We can
	      apply this ensemble if we consider each atom in the gas is a
	      separate system, and the rest of the gas is the reservoir.
	\item \textbf{Grand-canonical ensemble:} System can exchange heat and particles
	      with the reservoir. $V, T, \mu$  are fixed for the
	      system. We can apply this ensemble if a microstate of the
	      configuration of the gas is a system. The rest of gas is
	      reservoir.
\end{enumerate}

\begin{figure}[htpb]
	\centering
	\includesvg[width=0.9\columnwidth]{./ensembles.svg}
	\caption{Different types of Ensembles}%
	\label{fig:ensemble_types}
\end{figure}

\section{Principle of equal equilibrium probability}%
\label{sec:principle_of_equal_equilibrium_probability}

\begin{definition}
	When a thermally isolated system comes into thermal equilibrium,
	then the state probabilities of any set of mutually accessible
	states become \textbf{equal}.

	In a microcanonical ensemble, there is equal probability of any
	state begin occupied.
\end{definition}

\section{Ergodic hypothesis}%
\label{sec:ergodic_hypothesis}

\begin{definition}
	Given enough time the systems will explore all possible
	microstates and will spend an equal amount of time in each of
	them.

	Implication of this is that in equilibrium an isolated system
	will choose the macrostate with the most microstates. This is
	the Second Law (to maximise entropy).
\end{definition}

\chapter{Lecture 10 (Week 15)}%
\label{cha:lecture_10}

\section{Energy and states}%
\label{sec:energy_and_states}

The problem for $x$ quanta and
$N$ boxes is the same as asking how many
ways you can arrange $x$ quanta and
$N-1$ partitions in a line.

The partitions separate the boxes, excluding the walls. Thus if
there are $N$ boxes, there must be
$N-1$ partitions.

The total number of ways to arrange this would be
$ \left(x+N-1\right)! $. Partitions are identical and so are
quanta, thus we have to divide by $ \left(N-1\right)! x! $.

Our final result is therefore

\begin{definition}
	To arrange $x$ quanta in
	$N-1$ partitions or $N$
	boxes, there would be
	\begin{equation}
		\Omega
		=
		\frac{\left(x+N-1\right)!}{x!\left(N-1\right)!}
	\end{equation}
	ways of doing this.
\end{definition}

The number of states is a very rapidly (slightly slower than
exponential) increasing function of the total energy
$x$ (if you plot a log plot versus
$x$ it would appear almost linear).
See~\ref{fig:energy_and_states}

\section{Statistical Temperature}%
\label{sec:statistical_temperature}

Consider 2 systems \textbf{A} and
\textbf{B} in thermal contact with each other but
isolated from the surroundings, i.e.\ a Microcanonical ensemble.

Find the conditions where the number of available states is
maximum. This gives condition of thermal equilibrium.

Total energy at any point is fixed because the system is
thermally isolated.
\begin{equation}
	U_A + U_B = U_0 \Rightarrow \dv{U_A}{U_B} = -1
\end{equation}

The number of microstates is given by $\Omega(U_A)$ or
$\Omega(U_B)$. The total number of states as a
function of $U_A$ would be
\begin{equation}
	\Omega_{AB}
	=
	\Omega\left(U_A\right)\Omega\left(U_B\right)
	=
	\Omega\left(U_A\right)\Omega\left(U_0-U_A\right)
\end{equation}
Since $\Omega\left(U_0\right)$ is a very fast increasing
function, the negative of that would be a fast decreasing
function.

\begin{figure}[htpb]
	\centering
	\includesvg[width=0.9\columnwidth]{./energy_and_states.svn}
	\caption{\textbf{(a)} shows the log plot of
		$\Omega$ and
		$x$, the total energy, \textbf{(b)} shows the number of states of two
		microcanonical systems as a function of the energy of one.}%
	\label{fig:energy_and_states}
\end{figure}

We can actually derive the mathematical relations for the
maximum number of states, where the derivative of
$\Omega_{AB}$ is zero. The steps are outlined briefly.

\begin{derivation}
	We start of with our definition for the total states and find
	its derivative and rearrange a bit.
	\begin{equation}
		\begin{aligned}
			\Omega_{AB}                          & = \Omega\left(U_A\right)\Omega\left(U_B\right)                                                               \\
			\dv{\Omega_{AB}}{U_A}                & = \dv{\Omega_A}{U_A}\Omega_B + \dv{\Omega_B}{U_A}\Omega_A = 0                                                \\
			\dv{\Omega_{AB}}{U_A}                & = \dv{\Omega_A}{U_A}\Omega_B - \dv{\Omega_B}{U_B}\Omega_A = 0,\,\,\, \mathrm{since}\,\,\, \dv{U_A}{U_B} = -1 \\
			\frac{1}{\Omega_A}\dv{\Omega_A}{U_A} & = \frac{1}{\Omega_B}\dv{\Omega_B}{U_B}                                                                       \\
		\end{aligned}
	\end{equation}

	This is equivalent to
	\begin{equation}
		\dv{\naturallogarithm{\Omega_A}}{U_A} = \dv{\naturallogarithm{\Omega_B}}{U_B}
	\end{equation}
	This is our condition for thermal equilibrium. Think of
	Figure~\ref{fig:energy_and_states} (b), if the two derivatives are the
	same then that's where the two curves meet.

	Now, we can define the \textbf{statistical temperature} as
	\begin{equation}
		\label{eq:statistical_temperature}
		\frac{1}{k_b T_A} = \dv{\naturallogarithm{\Omega_A}}{U_A}
	\end{equation}
	This is a bit of a leap in definition, but it is dimensionally
	correct, and the Boltzmann constant $k_B$ is
	included so that it agrees with the kelvin scale.
\end{derivation}

\section{Boltzmann entropy}%
\label{sec:boltzmann_entropy}

If we use the first law equation~\ref{eq:first_law_entropy}, we
can write $T$ as
\begin{equation}
	\frac{1}{T} = {\left(\pdv{S}{U}\right)}_{V}
\end{equation}

If we compare this to Equation~\ref{eq:statistical_temperature}, we get
the Boltzmann's equation for entropy.

\begin{definition}
	The Boltzmann's equation for entropy state that
	\begin{equation}
		\label{eq:boltzmann_entropy}
		S = k_b \naturallogarithm{\Omega}
	\end{equation}

	A system will evolve to maximise $\Omega$. This
	is consistent with the second law---the system will evolve to
	maximise $S$.
\end{definition}

\chapter{Lecture 11 (Week 16)}%
\label{cha:lecture_11}

\section{Boltzmann distribution}%
\label{sec:bolztmann_distribution}

We note that random exchange lead to an exponential
distribution. From the definition of entropy, we can confirm it
does lead to a logarithmic distribution.

Consider a small system \textbf{S}, with a single
occupied state, in contact with a large reservoir
\textbf{R}  but otherwise thermally isolated. This
is a \textit{canonical ensemble}. See~\ref{sec:statistical_ensembles} for
further definitions.

We can say a few things about this system.
\begin{itemize}
	\item The total energy is $E_0$
	\item The energy in \textbf{S} is $E_i$
	\item The energy in \textbf{R} is $E_0 - E_i$
	\item Assume $E_i \ll E_0$
\end{itemize}

What is the probability that the system has a particular energy
$E_i$?

We expect the probability to be proportional to the product of
the number of states available to the small system (exactly one)
and the number of states available to the reservoir (equals to
$\Omega\left(E_0 - E_i\right)$). We can find this using Taylor
expansion.

\begin{derivation}
	Expand $\naturallogarithm{\Omega\left(E_0 - E_i\right)}$
	\begin{equation}
		\naturallogarithm{\Omega\left(E_0 - E_i\right)}
		=
		\naturallogarithm{\Omega\left(E_0\right)} +
		\dv{\naturallogarithm{\Omega\left(E_0\right)}}{E}\left(-E_i\right) +
		\frac{1}{2}\dv[2]{\naturallogarithm{\Omega\left(E_0\right)}}{E}E_i^2
	\end{equation}

	From our definition of statistical temperature in
	Equation~\ref{eq:statistical_temperature}, we can rewrite this into
	\begin{equation}
		\naturallogarithm{\Omega\left(E_0 - E_i\right)}
		=
		\naturallogarithm{\Omega\left(E_0\right)} -
		\frac{E_i}{k_B T} -
		\frac{1}{2k_B T^2}\dv{T}{E}E_i^2
	\end{equation}

	We can neglect the second order term if the reservoir is
	sufficiently large. Then we can take the exponential of both
	sides.
	\begin{equation}
		\Omega\left(E_0 - E_i\right)
		=
		\Omega\left(E_0\right) e^{-\frac{E_i}{k_B T}}
	\end{equation}

	Thus, the probability of of a given state being occupied can be
	given as
	\begin{equation}
		P\left(E_i\right) \propto
        e^{-\frac{E_i}{k_B T}}
	\end{equation}
	we call this the Boltzmann distribution function.
\end{derivation}

We also have to normalise this probability distribution function
such that the sum of all probabilities is
\num{1}. This gives us the normalisation
constant $1/Z$. Where Z is called the
\textit{partition function} and is given as
\begin{equation}
	Z = \sum_i e^{-\frac{E_i}{k_B T}}
\end{equation}

Note that this is the sum of \textit{all} states,
not just all energy levels. Thus, if an energy level is
degenerate (can have multiple states), we need to factor in the
degeneracy factor $g_i$, so that the energy
level is counted the correct number of times.

\begin{definition}
	For a canonical system, the Boltzmann Distribution is given as
	\begin{equation}
		\label{eq:boltzmann_canonical}
		P\left(E_i\right) =
        \frac{1}{Z}e^{-\frac{E_i}{k_B T}}
	\end{equation}
	where $Z$ is given as
	\begin{equation}
		Z = \sum_i g_i e^{-\frac{E_i}{k_B T}}
	\end{equation}
\end{definition}

\begin{example}
	Consider a system with two energy levels that are
	non-degenerate. Then the partition function
	$Z$ would be
	\begin{equation}
		Z = \sum_i e^{-\beta \epsilon_i} =
		e^{-\beta \epsilon_1} + e^{-\beta \epsilon_2}
	\end{equation}

	Next, consider rotational energy levels of molecules. We have
	calculated the quantum energy levels $E_J = \hbar^2 J \left(J + I\right)/2I$,
	where $J$ is the angular momentum quantum
	number and $I$ is the constant moment of
	inertia. The levels are degenerate, for $J$
	there are $2J + 1$ states. Then, the partition
	function is
	\begin{equation}
		Z = \sum_{J=0}^\infty
        \left(2J + I\right)e^{- \frac{J\beta \hbar^2}{2I}\left(J+1\right)}
	\end{equation}
\end{example}

\section{Boltzmann-Gibbs Entropy}%
\label{sec:boltzmann_gibbs_entropy}

For an isolated system (a microcanonical ensemble), we have
\begin{equation}
	S = k_B \naturallogarithm{\Omega}
\end{equation}
The entropy can be viewed as a lack of information, i.e., the
system can be in any of the $\Omega$ different
microstates.

The probability that the system is in any particular state is
inversely proportional to $\Omega$.

For the canonical ensemble, however, the probability is not the
same for all states of the system. Instead, it is given by the
Boltzmann factor in Equation~\ref{eq:boltzmann_canonical}.

\begin{definition}
	The general equation for Boltzmann-Gibbs entropy is
	\begin{equation}
		\label{eq:boltzmann_gibbs_entropy}
		S  = -k_B \sum_i P_i \naturallogarithm{P_i}
	\end{equation}
	This applies for a canonical system. For a microcanonical
	distribution, $S = k_B \naturallogarithm{\Omega}$.
\end{definition}

\chapter{Lecture 12 (Week 16)}%
\label{cha:lecture_12}

\section{The average of a distribution function}%
\label{sec:the_average_of_a_distribution_function}

\begin{definition}
	For a discrete distribution, the average is given by
	\begin{equation}
		\left< X \right>
		=
		\sum_i P_i X_i
	\end{equation}

	For a continuous distribution, the average is given by
	\begin{equation}
		\left< x \right>
		=
		\int^\infty_{-\infty}
		P\left(x\right)x\dd{x}
	\end{equation}
\end{definition}

\section{Internal energy in terms of Z}%
\label{sec:internal_energy_in_terms_of_z}

The internal energy can be calculated as
\begin{equation}
	U = \sum_i P_i E_i = \frac{\sum_i E_i e^{-\beta E_i}}{\sum_i e^{-\beta E_i}}
\end{equation}

But we can express this in terms of the partition function as
well.
\begin{definition}
	The internal energy is given as
	\begin{equation}
		\label{eq:internal_energy_partition}
		U = -\frac{1}{Z} \dv{Z}{\beta} = -
		\dv{\naturallogarithm{Z}}{\beta}
	\end{equation}
\end{definition}

\section{Entropy in terms of Z}%
\label{sec:entropy_in_terms_of_z}

We start with the Boltzmann-Gibbs entropy expression, and with a
bit of rearranging we can derive entropy in terms of the
partition function as well.

\begin{definition}
	The entropy is given as
	\begin{equation}
		S = \frac{U}{T} + k_B \naturallogarithm{Z}
	\end{equation}
\end{definition}

\section{Helmholtz Free energy in terms of Z}%
\label{sec:helmholtz_free_energy_in_terms_of_z}

\begin{definition}
	The Helmholtz Free energy is given as
	\begin{equation}
		F = -k_B T \naturallogarithm{Z}
	\end{equation}
\end{definition}

\section{Combining partition functions}%
\label{sec:combining_partition_functions}

A system may have two or more contributions to the energy
(rotation, vibration, motion along $x,y,z$).
Thus, we have to sum these contributions.
\begin{equation}
	Z = \sum_i \sum_j e^{- \beta \left( E^x_i + E^y_j \right)} = Z_x Z_y
\end{equation}

So, partition functions from independent contributions
\textit{multiply}.

But for the free energy, depends on the logarithm of
$Z$, thus the contributions are
\textit{additive}.
\begin{equation}
	F = F_x + F_y
\end{equation}

\section{Classical Equipartition theorem}%
\label{sec:classical_equipartition_theorem}

Each `degree of freedom' of a classical system adds
$\frac{1}{2}k_b T$ to the total energy. Works in the
classical limit as long as the degree of freedom adds an extra
term to the energy which varies as the variable squared (like
kinetic energy, elastic potential energy).

We can derive this by computing some Gaussian integrals.

With this, we find that the internal energy
$U = \left<E\right>$ for one degree of freedom is
\begin{equation}
	\left<E\right>
	=
	\frac{k_B T}{2}
\end{equation}

\subsection{Gaussian Integrals (not examinable)}%
\label{sub:gaussian_integrals_not_examinable_}

\begin{definition}
	Here are important identities for Gaussian integrals:
	\begin{equation}
		\int_{-\infty}^\infty e^{-\alpha x^2}
		\dd{x}
		=
		\sqrt{\frac{\pi}{\alpha}}\hspace{2em}
		\int_{0}^\infty xe^{-\alpha x^2}
		\dd{x}
		=
		\frac{1}{2\alpha}
	\end{equation}
	Notice the limits on odd powers of $x$. For
	further powers, take derivative w.r.t. $\alpha$
	on both sides.
\end{definition}

\subsection{Does the equipartition theorem work in a quantum system?}%
\label{sub:does_the_equipartition_theorem_work_in_a_quantum_system_}

We can answer this question by considering an example.

\begin{example}
	Consider a mass $m$ on a spring where the
	modes of oscillation are quantised by the boundary conditions.
	I.e., only certain oscillation frequencies are allowed.

	Recall the \textit{Hamiltonian}. Applying it to a spring
	system, we have
	\begin{equation}
		\Hamiltonian = \frac{ \hat{p}^2 }{2m} + \frac{k}{2}
		\hat{x}^2
	\end{equation}

	We recall our harmonic oscillators have energy levels
	\begin{equation}
		E_n = \left( n + \frac{1}{2}\right) \hbar \omega_0
		\hspace{2em} \mathrm{with}
		\hspace{.5em} \omega_0 = \sqrt{\frac{k}{m}}
	\end{equation}

	Our partition function would be
	\begin{equation}
		Z = \sum_{n=0}^\infty e^{-\beta\left(n + \frac{1}{2} \right)\hbar\omega_0} =
		\frac{e^{-\beta\hbar\omega_0/2}}{1 - e^{-\beta\hbar\omega_0}}
	\end{equation}

	We can arrive at this result by treating the double sum as an
	infinite geometric series $r^n$ where
	$r = \exponential{\left(-\beta\hbar\omega_0\right)}$.  Then if we find the internal energy by
	finding $\left<E\right>$, we find that it is
	\begin{equation}
        U = -\pdv{\naturallogarithm{Z}}{B} = \hbar^{2}\omega_0
		\left[\frac{1}{2} +
		\frac{e^{-\beta\hbar\omega_0}}{1-e^{-\beta\hbar\omega_0}}
		\right]
	\end{equation}

	Remember, we get to this result by computing the derivative of
	$Z$ w.r.t. $\beta$.

	The result we get is clearly quite different from the classic
	result. But for high temperature limit, we can expand the
	exponential in terms of its Taylor series and find that it
	\textit{does} agree with the equipartition theorem for
	2 degrees of freedom.
\end{example}

\chapter{Lecture 13 (Week 16)}%
\label{cha:lecture_13}

\section{Kinetic theory of an ideal gas}%
\label{sec:kinetic_theory_of_an_ideal_gas}

We apply the Boltzmann distribution to calculate the velocity
distribution of a gas.

\subsection{Assumptions of our model}%
\label{sub:assumptions_of_our_model}

\begin{itemize}
	\item Only translational degrees of freedom---no rotations
	\item Each molecule/atom has a unique velocity state
	\item Allowed velocity states are uniformly distributed in velocity
	      space
\end{itemize}

Consider a canonical ensemble. Each molecule is a (sub)system
connected to the thermal bath which is physically all the other
molecules in the gas.

The probability of a given $x$-component of
the velocity $v_x$
\begin{equation}
	P\left(v_x\right) \propto
    e^{-\frac{\epsilon}{k_B T}}\propto
    e^{-\frac{mv^2_x}{2 k_B T}}
\end{equation}

To work this out, we integrate the Boltzmann distribution with
respect to all three translational components
$v_{x,y,z}$ and set it to 1 since probability
distribution functions are normalised.
\begin{derivation}
	\begin{equation}
		P = A \iiint
		e^{-\beta \epsilon}\dd{v_x}\dd{v_y}\dd{v_z}
		= 1
	\end{equation}
	We have
	\begin{equation}
		\epsilon
		=
		\frac{m}{2} \left(v_x^2 + v_y^2 + v_z^2\right)
	\end{equation}

	Which gives us
	\begin{equation}
		P = A
		\int_{-\infty}^{\infty}\int_{-\infty}^{\infty}\int_{-\infty}^{\infty}
		e^{-\frac{\beta m v^2_x}{2}}e^{-\frac{\beta m v^2_y}{2}}e^{-\frac{\beta m v^2_z}{2}}\dd{v_x}\dd{v_y}\dd{v_z}
		= 1
	\end{equation}

	We use Gaussian integral identities to get
	\begin{equation}
		P\left(v_x, v_y, v_z\right) =
        {\left(\frac{m}{2\pi k_B T}\right)}^{\frac{3}{2}}e^{-\frac{\beta m}{2}\left(v_x + v_y + v_z\right)}
	\end{equation}
\end{derivation}

\section{Speed distribution}%
\label{sec:speed_distribution}

For some application of the kinetic theory, we don't need the
velocity distribution, but only the speed distribution.

Now, we assume that the number of allowed velocity states is
uniform in velocity space. Our integral now becomes an
integration over all velocity space. In 2D, we consider a small
element, a ring of a circle. In 3D, we want to integrate over a
sphere, thus, we consider a small surface element (then we want
to extent the radius of this sphere, which is our velocity
$v$, to infinity).

So, we have
\begin{equation}
	f(v)\dd{v} \propto v^2
    e^{-\frac{mv^2}{2k_B T}}
\end{equation}

\begin{derivation}
	To start, write out the integral as
	\begin{equation}
		f\left(v\right) = A \int_{0}^{\infty} v^2
		e^{-\beta \frac{mv^2}{2}} = 1
	\end{equation}

	Then once again, using Gaussian integral identities, we arrive
	at
	\begin{equation}
		f\left(v\right) = \frac{4}{\pi}{\left(\frac{\beta m}{2}\right)}^{\frac{3}{2}}v^2
		e^{-\frac{\beta mv^2}{2}}
	\end{equation}
\end{derivation}

\begin{definition}
	The Maxwell-Boltzmann speed distribution curve is
	\begin{equation}
		f\left(v\right) = \frac{4}{\pi}{\left(\frac{\beta m}{2}\right)}^{\frac{3}{2}}v^2
		e^{-\frac{\beta mv^2}{2}}
	\end{equation}

	See Figure~\ref{fig:maxwell_boltzmann_speed_distribution} for the shape of this
	distribution. It is a slightly asymmetric curve. The lower
	$T$ is, the less symmetrical it becomes. The
	higher $T$ is, the lower the highest
	probability peak. As expected, the average speed increases at
	higher temperature.
\end{definition}

\begin{figure}[htpb]
	\centering
	\includesvg[width=0.9\columnwidth]{./maxwell_boltzmann_speed_distribution.svg}
	\caption{The Maxwell-Boltzmann speed distribution curve for high and low
		temperatures}%
	\label{fig:maxwell_boltzmann_speed_distribution}
\end{figure}

Because this is a probability distribution, we can calculate the
mean average. We are particularly interested in the
root-mean-square speed. Here's how to derive it.

\begin{derivation}
	Begin with the standard integral for the mean-square, and apply
	the Gaussian integral to immediately get the result.
	\begin{equation}
		\left<v^2\right>
		=
		\int^{\infty}_0 v^2
		f\left(v\right)\dd{v} =
		\frac{3k_B T}{m}
	\end{equation}

	We can then find the average kinetic energy, which is given as
	\begin{equation}
		\left<E\right>
		=
		\frac{1}{2}m\left<v^2\right>
		=
		\frac{3}{2}k_B T
	\end{equation}

	We expected this expression, since we have derived this in the
	equipartition theorem for three degrees of freedom. See
	Section~\ref{sec:classical_equipartition_theorem}.
\end{derivation}

\section{Pressure distribution}%
\label{sec:pressure_distribution}

We break this calculation down into several steps. First, we
start with the definition of pressure as the force over the
area.

Next, when a particle hits the wall of the container, the change
in momentum is double the incoming momentum.
\begin{equation}
	\Delta P_m = 2mv\cos{\theta}
\end{equation}

If we have $n$ atoms per unit volume moving
at speed $v$ at angle
$\theta$, then the number that hit a wall of area
$A$ in time $\dd{t}$ is
\begin{equation}
	N_{\theta} = nv\cos{\theta}A\dd{t}
\end{equation}

We can rationalise this by thinking that if we take a snapshot
of time $\dd{t}$, each atom is currently on the
same angle towards the wall. But after that interval of time,
they would have moved $v\dd{t}$ and hit the wall.
So all the particles in that region hit the wall in that
interval of time. This forms a parallelepiped of volume
$v\dd{t}\cos{\theta}A$, and we can get the total atoms by
multiplying it by the atom per unit volume
$n$. See Figure~\ref{fig:pressure_distribution}.

\begin{figure}[htpb]
	\centering
	\includesvg[width=0.9\columnwidth]{./pressure_distribution.svg}
	\caption{Consider a unit volume of particles that hit the wall in a short
		time interval $\dd{t}$. It forms a parallelepiped.}%
	\label{fig:pressure_distribution}
\end{figure}

Consider a collection of particles with $\left|v\right| = 1$,
the velocity vectors will be on a sphere with the unit radius.
The velocity vectors are all random, but it will be within this
sphere.

If we want to know how many are moving at the angle
$\theta$, then we need to consider a ring on the
sphere between the angle $\theta$. In essence, we
want to consider a fraction of atoms travelling at an angle
$\theta \rightarrow \theta +\dd{\theta}$. All the atoms which travel at that
angle is in the ring, thus this would be the area of the ring
divide by the area of the whole sphere
\begin{equation}
	\frac{2\pi\sin{\theta}\dd{\theta}}{4\pi}
	=
	\frac{1}{2}\sin{\theta}\dd{\theta}
\end{equation}

Putting it all together, we have the following derivation

\begin{derivation}
	We integrate over the velocity function in that ring
	\begin{equation}
		P = \frac{\dv{p}{t}}{A} =
		\int_0^\infty\int_0^{\frac{\pi}{2}}2\frac{1}{2}
		f\left(v\right) mv\cos{\theta}
		nv\cos{\theta} A
		\dd{t}\sin{\theta}
		\frac{1}{A \dd{t}}
		\dd{\theta}\dd{v}
	\end{equation}

	If we use the substitution $y = \cos{\theta}$, and use a
	Gaussian integral identity, we get the following
	\begin{equation}
		P = \frac{4}{\sqrt{\pi}} {\left(\frac{\beta m}{2}\right)}^{\frac{3}{2}} \frac{nm}{3}
		\frac{3\sqrt{\pi}}{8} {\left(\frac{\beta m}{2}\right)}^{-\frac{5}{2}}
	\end{equation}

	Simplifying the answer gives us
	\begin{equation}
		P = nk_B T
	\end{equation}

	Where $n = N/V$ and $N = N_a n$. We
	notice that we get the ideal gas law from this.
\end{derivation}

\chapter{Lecture 14 (Week 17)}%
\label{cha:lecture_14}

\section{Statistical mechanics of ideal gas}%
\label{sec:statistical_mechanics_of_ideal_gas}

The basic idea is to:
\begin{itemize}
	\item Calculate the allowed energy modes of the system,
	      $E_i$
	\item We calculate the partition function $Z$.
	\item Finally, we can calculate the thermodynamic properties from the
	      partition function. These were derived
	      in~\ref{cha:lecture_12}. We recall them as:
	      \begin{equation}
		      F = -k_B T \naturallogarithm{Z}
	      \end{equation}
	      \begin{equation}
		      p = -{\left(\pdv{F}{V}\right)}_{T}
	      \end{equation}
	      \begin{equation}
		      S = -{\left(\pdv{F}{T}\right)}_{V}
	      \end{equation}
\end{itemize}

\section{Calculate the energy modes}%
\label{sec:calculate_the_energy_modes}

First, we treat the gas quantum mechanically. Consider a
particle in a box of side $L$, with volume
$ = L^3 $. We solve the Schrödinger equation for
the system.
\begin{equation}
	\label{eq:schrodingers_equation}
	-\frac{\hbar^2}{2m}\laplacian\Psi =
	i\hbar\pdv{\Psi}{t}
\end{equation}

The solutions of these are plane waves
\begin{equation}
	\Psi
	=
    \Psi_{0}e^{i\left(\vec{k}\vdot\vec{r} - wt\right)}
\end{equation}

If we apply the momentum operator to the solution, we can find
that it is
\begin{equation}
	\hat{p}\Psi
	=
	p\Psi \hspace{1em} p = \hbar k
\end{equation}

Finally, the energy is
\begin{equation}
	E = \frac{1}{2}mv^2 = \frac{p^2}{2m} =
	\frac{\hbar^2 k^2}{2m}
\end{equation}

Now, we impose the \textit{periodic} boundary conditions.
Recall that the quantisation of $k$ and
$E$ comes from the periodic boundary
conditions.
\begin{equation}
    e^{i k_x \left(x+L\right)} =
    e^{ik_x}
\end{equation}

Thus, the allowed values are $k_x = n_x 2\pi / L$. Extending
this to 3D, we get
\begin{equation}
	\Psi\left(x,y,z\right)
	\propto
    e^{i k_x x + k_y y + k_z z}
	\hspace{1em}
	\mathrm{where}
	\hspace{1em}
	-\infty < n_{x,y,z} < \infty
\end{equation}

\section{Density of states}%
\label{sec:density_of_states}

The allowed states are on a uniform grid in
$k$-space (`frequency' space). In 2D, the
number of states per unit area of $k$-space
is $ = {\left( L/2\pi\right) }^2 $ and in 3D, the number of states is
$ = { \left( L/2\pi\right) }^3$. See Figure~\ref{fig:frequency_space}

\begin{figure}[htpb]
	\centering
	\includesvg[width=0.9\columnwidth]{./frequency_space.svg}
	\caption{The frequency space in 2D}%
	\label{fig:frequency_space}
\end{figure}

This is called the \textit{density of states}
$g\left(k\right)$. If we write $k = \abs{ \vec{k}}$,
then we have in 3D the number of states between
$k$ and $k + \dd{k}$ presented as
a sphere (with $k$ as the radius). Since we
are considering a small element, we consider a shell with
surface area $4\pi k^2$ (for
$k$ from 0 to $\infty$).
Then, the general density of states is
\begin{equation}
	g\left(k\right)\dd{k} = 4\pi k^2
	\dd{k} \frac{V}{{\left(2\pi\right)}^3}
\end{equation}

\section{Calculate the partition function of a single atom}%
\label{sec:calculate_the_partition_function_of_a_single_atom}

We performed the intermediate step of finding
$g\left(k\right)$ so that we can find
$Z$. Since we have the density of states, we
can sum the energy levels, this becomes an integral of states
over all volume.
\begin{equation}
	\sum_i \rightarrow \int^\infty_0 g\left(k\right)
	\dd{k}
\end{equation}

\begin{derivation}
	Let's derive the expression for $Z_1$
	\begin{equation}
		Z_1 = \sum_{k_x,k_y,k_z} \exp\left(-\beta \hbar k^2 / 2m\right) =
		\int_0^\infty g\left(k\right)
		\exp\left(-\beta \hbar k^2 / 2m\right) \dd{k}
	\end{equation}

	We get the following integral
	\begin{equation}
		Z_1
        =
        \frac{4\pi V}{{\left(2\pi\right)}^3} \int_0^\infty k^2
		e^{-\alpha k^2} \dd{k}
		\hspace{1em} \mathrm{with}
		\hspace{1em} \alpha = \frac{\beta \hbar^2}{2m}
	\end{equation}

	Using a standard Gaussian integral, we get
	\begin{equation}
		\label{eq:partition_function}
		Z_1 = \frac{V}{\lambda^3} \hspace{1em}
		\mathrm{with} \hspace{1em} \lambda =
		\frac{h}{\sqrt{2\pi m k_B T}}
	\end{equation}
	where $\lambda$ is the thermal de Broglie
	wavelength.

	The classic atom thermal energy $E = \frac{3}{2}k_B	T = \frac{p^2}{2m}$,
	$ p = \sqrt{3 m k_B T} $ and thus $ \lambda = h / \sqrt{3m k_B T} $.
	Between the two, there is only a factor of
	$\sqrt{2\pi/3}$ difference.
\end{derivation}

\section{Thermal de Broglie wavelength}%
\label{sec:thermal_de_broglie_wavelength}

This is the characteristic size of system where quantum effects
become important. For example, if we put an electron in a box,
since the de Broglie wavelength depends on the mass and
temperature. If we set the mass as the mass of an electron, then
a box with a side of $\lambda_{\mathrm{th}} =
	\SI{1}{\metre}$ would have
$T = \SI{ 5e-15 }{\kelvin}$. This has not been achieved
experimentally. If, instead we have a box of side
$\lambda_{\mathrm{th}} =
	\SI{3e-4}{\metre}$ would have $T = \SI{ 6e4 }{\kelvin}$. This
is about the distance between 2 atoms in a solid. Thus,
electrons in a solid are always in the quantum regime.

\chapter{Lecture 15 (Week 17)}%
\label{cha:lecture_15}

\section{Partition function of an ideal gas}%
\label{sec:partition_function_of_an_ideal_gas}

The partition function for a single atom is
$Z_1$, but a real gas has got
$N$ atoms. We then 2 cases.

\textbf{Case 1:} The atoms are distinguishable (each one
different colour). Then, we would have a system (each being an
atom), connected to a reservoir (the other atoms). Each atom is
in a different configuration, then we would have to take into
account all the permutations as one atom can take the
configuration of the other. The probability of a particular
macrostate of gas is $\propto\exponential{\left(-\beta E_{i1} + -\beta E_{i2} + -\beta E_{i3} + \hdots\right)}$, and thus
\begin{equation}
	Z_{N,D} = \left(\sum_{i1}e^{-\beta E_{i1}}\right) \time\left(\sum_{i2}e^{-\beta E_{i2}}\right)\times\left(\sum_{i3}e^{-\beta E_{i3}}\right) \hdots
\end{equation}

But each sum is the same and equal to $Z_1$,
thus
\begin{equation}
	Z_{N, D} = Z^N_1
\end{equation}

However, a real gas does not have distinguishable atoms.
Recalling from Section~\ref{cha:lecture_10}, if all the
atoms are indistinguishable, then we have over-counted our
states by $N!$. However, this is only
approximately true because we have neglected the possibility
that two atoms have exactly the same energy levels. This would
be like putting two particles in the same box in the example
below.

\textbf{Case 2:} If the atoms are indistinguishable,
then we have to divide by $N!$
\begin{equation}
	N_{N,I} \simeq \frac{Z_1^N}{N!} \equiv Z_G
\end{equation}

This is approximately true as long as we can neglect any atoms
which, in this approximation, would have the same quantum
numbers. This is OK as long as the number of possible states is
\textit{much larger} than the number of atoms. This
corresponds the regime where the linear size of the system
$\gg \lambda_{\mathrm{th}}$.

So the expression of $Z$ works in the
classical regime, since the size of the system satisfies the
condition above.
\begin{equation}
	Z_G = {\frac{V}{\lambda^3_\mathrm{th}}}^N / N! =
	{V {\left(m k_B T / 2\pi \hbar^2\right)}^{3/2} }^N / N!
\end{equation}

\section{Pressure of an ideal gas}%
\label{sec:pressure_of_an_ideal_gas}

If we recall the definitions for pressure and entropy in terms
of $Z$ and $F$, then we
can derive this for a whole system of an ideal gas. While the
expressions are similar to what we have derived before, we can
get a particular equation for entropy for a monoatomic ideal gas

\begin{definition}
	The Sackur-Tetrode equation for entropy of a monoatomic ideal
	gas is
	\begin{equation}
		S = N k_B \left[ \left(\naturallogarithm{\frac{V}{\lambda^3 N}}\right) + \frac{5}{2} \right]
	\end{equation}
\end{definition}

\section{Gibbs Paradox}%
\label{sec:gibbs_paradox}

We can use the equation for entropy we just quoted to discuss
this paradox. Consider two boxes separated with a partition.
Each box has $N$ particles of either type
$A$ or $B$ and both have
volume $V$. Now, we remove the partition.
What happens?

\textbf{Case 1:} If the atoms $A$
and $B$ are indistinguishable. Then before,
the total entropy is

\begin{equation}
	S = 2 \times N k_B \left[
		\left(\naturallogarithm{\frac{V}{\lambda^3 N}}\right) + \frac{5}{2}\right].
\end{equation}

After we remove the partition, the volume is
$2 V$S, and the total number of particles is
$2 N$. Thus, we have the same entropy as
before. The change in entropy is 0.

\textbf{Case 2:} The atoms $A$ and
$B$ are distinguishable. Before, we have

\begin{equation}
	S = 2 \times N k_B \left[
		\left(\naturallogarithm{\frac{V}{\lambda^3 N}}\right) + \frac{5}{2}\right].
\end{equation}

After we remove the partition, there would be
$N$ atoms of each type in a volume of
$2V$. Since this is an ideal gas, the atoms
of different types don't interact. Each set of atom types move
freely in a volume of $2V$. Thus, the entropy
is
\begin{equation}
	S = 2 \times N k_B \left[
		\left(\naturallogarithm{\frac{2V}{\lambda^3 N}}\right) + \frac{5}{2}\right].
\end{equation}

Therefore, the total change in entropy is
$\Delta S = 2 N k_B \naturallogarithm{2}$.

The paradox is that if the particles have some property
$X$ which we can't measure, but someone else
could? So they appear to be identical but are in fact not. Would
the entropy change when suddenly we are able to measure the
difference?

The answer, from the two cases, is \textit{yes}.
Entropy depends on our knowledge of the system. There is a
connection between entropy and connection. The way to interpret
the extra entropy is to note that after mixing
$A$ and $B$, to restore
the system to its original state, we would need to do work on
the system equal to at least $T\Delta S$.

\chapter{Lecture 16 (Week 17)}%
\label{cha:lecture_16}

\section{Grand canonical ensemble}%
\label{sec:grand_canonical_ensemble}

The grand canonical ensemble is when we can exchange heat
\textit{and} particles with a reservoir.

There are two equivalent ways to describe a system. First, we
consider each atom separately as a subsystem. Each atom has a
set of energy levels. Otherwise, we can consider the whole gas
as the system. Each energy level of the system is the sum of
some particular combination of the energy levels of each atom.
There are multiple particles in the system. In principle each
level could be multiply occupied. This is the grand canonical
ensemble.

When we dealt with ideal gas in the canonical ensemble, we had
to deal with identical configurations and had to ignore the
issues with degeneracy. This is not feasible in quantum regime,
due to \textit{Pauli's exclusion principle}.

\section{Gibb's distribution}%
\label{sec:gibb_s_distribution}

Similar to how we derived the Boltzmann's distribution, we have
a system and a reservoir, except now it can exchange energy and
particles. Both reservoir and system are isolated, which means
there is a total energy and a total number of particles.

Our strategy: We assume the system has only one state to find
the number of states available to the reservoir. Then, we Taylor
expand the state number as a function of energy and number of
particles. The probability is proportional to the number of
states, which we then equate as an exponential.

\begin{derivation}
	\begin{equation}
		\naturallogarithm \Omega \left( U_0 - E_i, N_0 - N_i\right)
		=
		\naturallogarithm\Omega\left( U_0, N_0\right) - E_i
		{\left(\pdv{\naturallogarithm\Omega}{U}\right)}_{N,V} - N_i
		{\left(\pdv{\naturallogarithm\Omega}{N}\right)}_{U,V}
	\end{equation}

	Now, we use the first law with the chemical potential in
	Equation~\ref{eq:first_law_chemical_potential}. We then write its exact
	differential. We get
	\begin{equation}
		\frac{1}{T} = {\left(\pdv{S}{U}\right)}_{V,
		N}\hspace{1em} \frac{\mu}{T} =
		-{\left(\pdv{S}{N}\right)}_{U,V}
	\end{equation}
	Using $S = k_B \naturallogarithm\Omega$, we now have
	\begin{equation}
		\naturallogarithm\Omega =
		\naturallogarithm\Omega\left(U_0, N_0\right) -
		\frac{E_i}{k_B T} + \frac{N_i \mu}{k_B T}
	\end{equation}
	Taking the exponential of both sides yields
	\begin{equation}
        P\left(N_i, E_i\right)\propto e^{\beta
		\left(\mu N_i - E_i\right)}
	\end{equation}
	where the normalisation constant is $1/Z_G$,
	where $Z_G$ is the Grand Partition function
	\begin{equation}
        Z_G = \sum_i e^{\beta\left(\mu N_i - E_i \right)}
	\end{equation}
\end{derivation}

We can now use the grand partition function to work out
$U$, the average energy of the system. These
quantities turn out to be
\begin{equation}
	\left< N \right>
	=
	\sum_i N_i P_i = k_B T {\left(\pdv{\naturallogarithm Z_G}{\mu}\right)}_{\beta}
\end{equation}
and
\begin{equation}
	U = \sum_i E_i P_i = - {\left(\pdv{\naturallogarithm Z_G}{\beta}\right)}_{\mu} + \mu
	\left<N\right>
\end{equation}

\chapter{Lecture 17 (Week 17)}%
\label{cha:lecture_17}

\section{Statistics of quantum particles}%
\label{sec:statistics_of_quantum_particles}

Two types of quantum particles are fermions and bosons.

\subsection{Fermions}%
\label{sub:fermions}

Fermions are any particles with half integer spin,
$S = 1/2, 3/2$ etc.\ in units of
$\hbar$. An example would be an electron. A
composite fermion consist of multiple particles, and the entity
as a whole has a half integer spin. For example, a carbon
nucleus.

The total wavefunction for a fermion system must be
antisymmetric with respect to particle exchange. Mathematically,
this means
\begin{equation}
	\Psi\lefT(a,b\right)
	=
	-\Psi\left(b,a\right)
\end{equation}
if the space and spin coordinates of two identical particles are
interchanged, then the total wave function must changes spin.

This property implies that each fermion must have a unique
quantum number (e.g., $k_x$,
$k_y$, $k_z$, spin). We can
only have one fermion in any single quantum state.

For example, $A\left(x,y\right) = -A\left(y,x\right)$.\ If
$y=x$ then this is only true if
$A = 0$. In general, $A\left(x,x\right) \neq -A\left(x,x\right)$.
This property is called the \textit{Pauli's exclusion principle}.

\subsection{Bosons}%
\label{sub:bosons}

Bosons are particles with integer spin $S = 0, 1, 2$.
An example would be a photon with spin $ S = \pm 1 $.
This corresponds to left or right hand polarisation. Phonon are
quantised lattice vibration in a solid
($S = 0$).

For a system of Bose particles the total wavefunction must be
symmetric with respect to particle exchange.
\begin{equation}
	\Psi\left(a,b\right) =
	\Psi\left(b,a\right)
\end{equation}

So there are no restrictions on the number of bosons in a single
quantum state. Both fermion and boson were terms coined by Paul
Dirac.

\section{Distribution functions of particles}%
\label{sec:distribution_functions_of_particles}

Distribution function describes the mean occupancy of a
particular state. For a fermion system, the state can either be
occupied or not, but on average (ensemble average), there will
be some mean occupancy somewhere between 0 and 1. Similarly for
bosons, each state will have some mean occupancy.

\subsection{Fermi-Dirac distribution function}%
\label{sub:fermi_dirac_distribution_function}

Consider a very simple system which has just one state, with
energy $\epsilon$, connected to a heat and particle
reservoir (a grand canonical ensemble). Here, we use Gibb's
distribution derived in Section~\ref{cha:lecture_16}. There
can either be 0 or 1 particle in the state.
\begin{equation}
	Z_G = \sum_i e^{\beta \left(\mu N_i - E_i\right)}
\end{equation}

The two possibilities are $N_0 = 0\hspace{1em} E_0 = 0$ and
$N_1 = 1 \hspace{1em} E_1 = \epsilon$. Thus
\begin{equation}
	Z_G = 1 + e^{\beta\left( \mu - \epsilon\right)}.
\end{equation}

We can find the mean occupancy from the grand partition function
\begin{equation}
	\left<N\right>
	=
	k_B T {\left(\pdv{Z_G}{\mu}\right)}_{\beta}.
\end{equation}

\begin{definition}
	The Fermi-Dirac distribution function for fermions is
	\begin{equation}
		\label{eq:fermi_dirac_distribution_function}
		\left<N\right>
        = \frac{1}{e^{\beta\left(\epsilon - \mu\right)} + 1}.
	\end{equation}
\end{definition}

\subsection{Bose-Einstein distribution function}%
\label{sub:bose_einstein_distribution_function}

Consider the same system as the fermions, but now there is no
restriction on the number of particles allowed in the state.
Now, the possible states of the systems are
\begin{equation}
	\begin{aligned}
		N_0 = 0 & \hspace{1em} E_0 = 0         \\
		N_i = 1 & \hspace{1em} E_1 = \epsilon  \\
		        & \vdots                       \\
		N_i =n  & \hspace{1em} E_i = n\epsilon
	\end{aligned}
\end{equation}

We find the mean occupancy using the same equation.
\begin{definition}
	The Bose-Einstein distribution function for bosons is
	\begin{equation}
		\label{eq:bose_einstein_distribution_function}
		\left<N\right>
		=
		\frac{1}{e^{\beta\left(\epsilon - \mu\right)} - 1}
	\end{equation}
\end{definition}

These distribution functions tells the mean occupancy of some
state with energy $E$ in terms of the
chemical potential.

\subsection{General distribution function}%
\label{sub:general_distribution_function}

In general, for systems with many different energy states, we
need to generalise these results.

\begin{definition}
	The mean occupancy for a general system with an energy level
	$\epsilon_i$ and particles is
	\begin{equation}
		\left<N_i\right>
		=
        \frac{1}{e^{\beta\left(\epsilon_i - \mu\right)}\pm 1}
	\end{equation}
	where it is $+$ for fermions and
	$-$ for bosons.

	For the Fermi-Dirac relation, $\left<N\right> = f\left(\epsilon\right)$ varies
	between 0 and 1, and it is 0.5 at $\epsilon =\mu$. The
	width in energy of the step is $\sim k_B T$.

	For the Bose-Einstein relation, $\left<N\right> = b\left(\epsilon\right)$ varies
	between 0 and $\infty$. $\epison > \mu$
	because $b\left(\epsilon\right) \rightarrow \infty$ at $\epsilon = \mu$.

	See Figure~\ref{fig:mean_occupancy}.
\end{definition}

\begin{figure}[htpb]
	\centering
	\includesvg[width=0.9\columnwidth]{./mean_occupancy.svg}
	\caption{The graph on the left shows the Fermi-Dirac mean occupancy while
		the one the right shows the Bose-Einstein mean occupancy.}%
	\label{fig:mean_occupancy}
\end{figure}

\section{Classical limit}%
\label{sec:classical_limit}

At low occupancy $\left<N\right> \ll 1$, the Fermi-Dirac,
Bose-Einstein and Boltzmann distribution functions (with a
energy zero shifted to $\mu$) all become
equal. This is because in this limit there is very little chance
of multiple occupancy so the quantum nature of the particles is
irrelevant.

If we plot $\left<N\right>$ against
$\beta\left(\epsilon - \mu\right)$, we see all functions tend
asymptotically towards each other (the Fermi-Dirac and
Bose-Einstein becomes similar to the Boltzmann's distribution
with 0 shifted to $\mu$). See
Figure~\ref{fig:classical_limit}.

Note that if we shift the energy in the Boltzmann distribution,
we are basically multiplying the original distribution with a
constant. It is equivalent to setting the total number of
particles in the system.

\begin{figure}[htpb]
	\centering
	\includesvg[width=0.9\columnwidth]{./classical_limit.svg}
	\caption{Plot to illustrate the classical limit of the Fermi-Dirac and
		Bose-Einstein mean mean occupancy}%
	\label{fig:classical_limit}
\end{figure}

\chapter{Lecture 18 (Week 18)}%
\label{cha:lecture_18}

\section{Fermi-gas}%
\label{sec:fermi_gas}

A Fermi-gas is gas of \textit{non-interacting} fermions confined
to a specific volume. Examples of system that this approximately
describes
\begin{itemize}
	\item Conduction electrons in a simple metal (behaves as if they are
	      neutral particles moving around in metal)
	\item Neutrons in a neutron star (collapsed core of a giant star which
	      consists of mostly just neutrons; these have intense magnetic
	      fields)
	\item Electrons in a white dwarf (a star at the final stage of stellar
	      evolution)
\end{itemize}

As it is non-interacting we can use the energy-momentum
relationship for free particles
\begin{equation}
	\epsilon = \frac{p^2}{2m} = \frac{\hbar^2 k^2}{2m}
\end{equation}

Because the gas is confined to a finite volume its
$k$-states are quantised (as for a classical
ideal gas). The difference is that now we can only put 1 fermion
in a single state. Electrons have 2 spin states,
$m_s = \pm 1/2$, so we can put one of each spin state in
a single $k$-state.

At $T = 0$, all states are filled for
$\epsilon < \mu$. The maximum filled energy at
$T = 0$ is called the \textit{Fermi energy}
$\epsilon_F$.
\begin{equation}
	\mu\left(T = 0\right)\equiv\epsilon_F
\end{equation}
with the maximum momentum $p=\hbar^2 {k_F}^{2}$.

At $T = 0$ we can calculate
$\mu$, the chemical potential, by first
summing up the mean occupancy of the state.
\begin{equation}
	\sum_i \left<N_i\right> = N
\end{equation}
where $N$ is the total number of electrons
in the gas.

As the states are very close together in energy we can convert
the sum to an integral.
\begin{equation}
	\sum_i^\infty\rightarrow\int_0^\infty
	D\left(\epsilon\right)\dd{\epsilon} = \int_0^\infty
	D\left(k\right)\dd{k}
\end{equation}
this quantity $D\left(\epsilon\right)$ is called the energy
density of states. We covered this in
Section~\ref{sec:density_of_states}.

\textbf{Note:} we can perform this integral over the
energy or over $k=\absolutevalue{\vec{k}}$.

\begin{derivation}
	In our axes of $k$-states, each occupancy
	must be quantised with gap $2\pi/L$. See
	Figure~\ref{fig:frequency_space}.

	Per unit volume of $k$-space, the number of
	$k$-states is $V/{\left(2\pi\right)}^3$, with
	2 electrons, $\uparrow$ and $\downarrow$
	in each state.

	The steps we take are exactly like those from
	Section~\ref{sec:density_of_states}, we quote the result that
	\begin{equation}
		g\left(k\right)\dd{k} =
		\frac{Vk^2}{2pi^2}
	\end{equation}

	We can express the DOS in terms of $\epsilon$ and
	$\epsilon + \dd{\epsilon}$ with $2\dd{k} = \dd{\epsilon}$
	\begin{equation}
		2g\left(k\right)\dd{k} =
		g\left(\epsilon\right)\dd{\epsilon}
	\end{equation}

	After some rearranging, this gives
	\begin{equation}
		g\left(\epsilon\right) =
		\frac{V}{2pi^2}{\frac{2m}{\hbar^2}}^{\frac{3}{2}}\sqrt{\epsilon}
	\end{equation}
\end{derivation}

With this, we can work out $N$
\begin{equation}
	N = \int^\infty_0 f\left(\epsilon\right)D\left(\epsilon\right)
\end{equation}
where $f\left(\epsilon\right)$ is the Fermi-Dirac function as
seen in Section~\ref{sub:fermi_dirac_distribution_function}. We know at
$T = 0$, $f\left(\epsilon\right) = 1$ for
$\epsilon < \epsilon_{F}$, otherwise 0. (See
Figure~\ref{fig:mean_occupancy}). Thus, we only need to
integrate $D\left(\epsilon\right)$ between 0 and
$\epsilon_{F}$. $D\left(\epsilon\right) = g\left(\epsilon\right)$ as we derived.

\begin{definition}
	The Fermi energy is the maximum filled energy for a Fermi-gas at
	$T = 0$ and it is given as
	\begin{equation}
		\label{eq:fermi_energy}
		\epsilon_{F}
		=
		\frac{h^2}{2m}{\left(\frac{3\pi^2 N}{V}\right)}^{\frac{2}{3}}
	\end{equation}
\end{definition}

We can also get the same result if we note that at
$T = 0$, all states are confined within a
frequency space sphere of radius $k_f$.

Thus the total number of states $N$ is the
total volume multiplied by the state per unit volume of
frequency space
\begin{equation}
	\label{eq:fermi_gas_number_of_states}
	N = \frac{4}{3\pi}\frac{2V}{{\left(2\pi\right)}^3}k^3_F =
	\frac{Vk^3_F}{3\pi^2}.
\end{equation}

Rearrange $k$ for $\epsilon$
and we get the same result.

\section{Fermi temperature}%
\label{sec:fermi_temperature}

\begin{definition}
	This is a temperature equivalent to the Fermi energy (it is not
	a physical temperature)
	\begin{equation}
		T_F = \frac{\epsilon_F}{k_B}
	\end{equation}
\end{definition}

\chapter{Lecture 19 (Week 18)}%
\label{cha:lecture_19}

\section{Thermodynamic properties of Fermi gas}%
\label{sec:thermodynamic_properties_of_fermi_gas}

At $T = 0$, we found from
Section~\ref{sec:fermi_gas} the expression for
$N$, which is
Equation~\ref{eq:fermi_gas_number_of_states}.

Now, we can also write an expression for
$U$, the internal energy of the Fermi gas.
This is written as
\begin{equation}
	U = \int^\infty_0\epsilon
	D\left(\epsilon\right)f\left(\epsilon\right)\dd{\epsilon}
	= \frac{2^{\frac{3}{2}}V}{5\pi^2\hbar^3}{\epsilon_F}^{\frac{5}{2}},
\end{equation}
where we are effectively taking the energy
$\epsilon$ of each state, multiplied it by the
density of states $D\left(\epsilon\right)$, multiplied by the
Fermi energy $f\left(\epsilon\right)$, and integrate over all
space. This is essentially the energy of each state multiplied
by the probability it is occupied.

To get the pressure $p$, we can eliminate
$\epsilon$ from the expression for
$U$ using
\begin{equation}
	p = -{\left(\pdv{U}{V}\right)}_{S,N}.
\end{equation}

What we do now is express $\epsilon_F$ in terms of
$N$. Then substitute this into
$U$. Then, we can finally do the partial
derivative, keeping $S$ and
$N$ constant. We get
\begin{equation}
	p = \frac{2}{3}\frac{U}{V}.
\end{equation}

\begin{derivation}
	Let
	\begin{equation}
		a =
		\frac{V}{2\pi^2}{\left(\frac{2m}{\hbar^2}\right)}^{\frac{3}{2}}\hspace{1em}\textrm{so that}\hspace{1em}D\left(\epsilon\right)
		= a\epsilon_F^{\frac{1}{2}}.
	\end{equation}

	Then at $T = 0$
	\begin{equation}
		N =
		\frac{2}{3}a\epsilon_F^{\frac{3}{2}}\hspace{1em}\textrm{and}\hspace{1em}
		U = a\epsilon_F^{\frac{5}{2}}
	\end{equation}

	From the first law, we rearrange to get the pressure, which is
	the relationship we quoted above. Replace
	$\epsilon_F$ in $U$ using
	$N$. This gives
	\begin{equation}
		U =
		\frac{2}{5}a{\left(\frac{3}{2}\frac{N}{a}\right)}^{\frac{5}{3}}
		= \frac{2}{5}{\left(\frac{3}{2}N\right)}^{\frac{5}{3}}a^{-\frac{2}{3}}
	\end{equation}
	Expressing $p$ using the chain rule
	\begin{equation}
		p = -\dv{U}{a}\dv{a}{V}
	\end{equation}

	This finally gives us
	\begin{equation}
		p = \frac{2}{3}\frac{U}{V}.
	\end{equation}

	Compare this to a classical gas (monoatomic):
	\begin{equation}
		U =
		\frac{3}{2}nRT\hspace{1em}\textr{and}\hspace{1em}P=\frac{nRT}{V},
	\end{equation}
	which would gives us the same answer.
\end{derivation}

This expression is exactly the same for a classical ideal gas,
but for an ideal gas $p$ and
$U$ depend on $T$. Here,
the pressure is at $T = 0$. This is called
\textit{degeneracy pressure} because it results from the fermionic
properties of the particles (the Pauli's exclusion principle).
If we go back to the example of the white dwarf, it is this
pressure that keeps the star from collapsing further. This
similarly applies to the neutron star.

Solving equation for $N$ at finite
$T$ (not at $T = 0$) gives
an expression for the temperature dependence of the chemical
potential. We find that the chemical potential has a very week
temperature dependence because $T \ll T_F$, where
$T_F$ is the Fermi temperature defined in
Section~\ref{sec:fermi_temperature}.
\begin{definition}
	The equation for the chemical potential as a function of
	temperature is
	\begin{equation}
		\mu\left(T\right)
        =
		\simto
		\epsilon_F\left(1 - \frac{\pi^2}{12}{\left(\frac{T}{T_F}\right)}^2\right).
	\end{equation}

	For gases at room temperature and below, we can assume that the
	chemical potential is equal Fermi energy (it has no temperature
	dependence)
\end{definition}

\section{Specific heat}%
\label{sec:specific_heat}

We are also interested in the heat capacity, which is the
derivative of the total energy with respect to temperature. At
$T = 0$, this is a step function. At finite
temperature, we see some broadening of the step function. This
width in broadening is $k_B T$. See
Figure~\ref{fig:mean_occupancy}. As temperature increases the
occupancy of the states above $\epsilon_F$ increases
and those below $\epsilon_F$ decreases. The end
result is a transfer of electrons from below
$\epsilon_F$ to above $\epsilon_F$. This
results in a net increase in internal energy. Essentially, we
promoted some of the states to go beyond the Fermi energy. We
are interested in the change in this energy, and then take the
derivative of that to get the heat capacity.

Recall that
\begin{equation}
	\label{eq:density_of_states_wrt_number_of_states}
	D\left(\epsilon\right) = \dv{N}{\epsilon}
\end{equation}
and the number of states in the energy range
$\delta\N = D\left(\epsilon\right)\delta\epsilon$. Since $\delta\epsilon = k-B T$, we know
$\delta N$, and since $\delta U = \delta N\delta\epsilon$, we
get
\begin{equation}
	\delta U
	=
	D\left(\epsilon\right){\left(k_B T\right)}^2
\end{equation}

Finally, we can find $C_V$, which is
\begin{equation}
	C_V = \dv{U}{T} = 2D\left(\epsilon\right)k_B^2 T
\end{equation}

\textbf{Note:} we are working with the assumption that
$\epsilon_F \ll k_B T$, therefore, $D\left(\epsilon\right)$ in
this range $k_B T$ is small enough to appear
constant, i.e.\ $D\left(\epsilon\right) = D\left(\epsilon_F\right)$. This is an
approximation.

We could also carry out the calculation more precisely.

\begin{definition}
	The heat capacity of a Fermi-gas is
	\begin{equation}
		C_V = \frac{\pi^2}{3}k_B^2 D\left(\epsilon\right) T.
	\end{equation}
\end{definition}

The different coefficient comes from if we calculate
$U$ through evaluating the integral. The
main point is that $C_V$ is linearly dependent
on the temperature and tends to zero at $T = 0$,
in accordance with the Third Law (See
Section~\ref{sub:consequences_of_the_third_law} and specifically number 2).
What's more, it also depends on the density of state at the
Fermi energy. Thus, a metal with a higher density of state would
have a higher heat capacity.

\textbf{Note:} we can also derive this in terms of the
partition function, but it is more involved and complicated.

\section{Fermi gas in one and two dimensions}%
\label{sec:fermi_gas_in_one_and_two_dimensions}

Sometimes, electrons are trapped in a single layer. For example,
graphene is a very thin and flat sheet where the electrons live
on a 2D sheet. The thickness of the sheet is thin enough it can
be approximated to be sufficiently thin for 2D. Another example
is high temperature superconductors which involve a layer in a
compound in which electrons can move freely in one plane, but
not in the other plane. To some extent, we can view these to be
confined in a 2D plane. Another example is a 2D electron gas,
which are electrons confined to the region between two pieces of
semiconductors. At the interface between these two materials, we
can have a layer of trapped electrons.

The case of 1D also exists, we just have to limit one more
dimension through elimination from approximation.

The key point is that the number of
$k$-states per unit \textit{area}
of $k$-space is now
\begin{equation}
	\frac{L^2}{{\left(2\pi\right)}^2}
	=
	\frac{A}{{\left(2\pi\right)}^2}
\end{equation}

The calculation is carried out as before, but our density of
states will also change into a circle instead of a sphere.
Unlike the case in 3D, the DOS in 2D does
\textit{not} depend on energy, but only
$m$, the mass of the electron. This is
expressed as
\begin{equation}
	D\left(\epsilon\right) = \frac{Am}{\pi\hbar^2}.
\end{equation}

\chapter{Lecture 20 (Week 18)}%
\label{cha:lecture_20}%

Consider a Bose-Einstein gas of massless bosons, with energy
\begin{equation}
	\epsilon
	=
	\hbar\omega
	=
	\hbar ck
\end{equation}
where $c$ is the speed of light.

The Bose-Einstein distribution function differs slightly from
the Fermi-Dirac function. Refer to
equations~\ref{eq:fermi_dirac_distribution_function} and~\ref{eq:bose_einstein_distribution_function}
for the respective equations.

For photons, the chemical potential is $\mu = 0$.
This gives the special Planck distribution.
\begin{equation}
	\label{eq:plancks_distribution_function}%
	p\left(\epsilon\right) = \frac{1}{e^{\beta\epsilon} - 1}
\end{equation}

We might ask, why is the chemical potential zero for a photon?
This is because photons can be created or destroyed (with
arbitrarily low energy) so the number (photon number) in a
closed box of fixed energy is not fixed. So the free energy
cannot depend on $N$. We know that
$\mu$ can be written in as the partial
derivative of the Helmholtz free energy with respect to
$N$, keeping $V,T$
constant, thus, by definition, it is 0.
\begin{equation}
	\mu = {\left(\pdv{F}{N}\right)}_{V,T} = 0
\end{equation}

\subsection{Density of states of Bose gas}%
\label{sub:density_of_states_of_bose_gas}%

We will now derive the density of states of a Bose gas.

\begin{derivation}
	As usual, we start in our frequency space. If the photons are in
	a box of side $L$, with the number of
	$k$-states per unit volume of
	$k$-space is $L^3/{\left(2\pi\right)}^3 = V/{\left(2\pi\right)}^3$.

	For photons, there are 2 polarisation of photons,
	$\pm 1$. Thus, for each
	$k$-state, we would have two photons, making
	the number of photons per unit volume $2V/{\left(2\pi\right)}^3$.

	We also notice that $D\left(\epsilon\right)\dd{\epsilon} =
		D\left(k\right)\dd{k}$.

	We care about the dispersion relation, which is the derivative
	of $k$ with respect to
	$\epsilon$, which is $\hbar c$. For
	bosons, it is linear, while for fermions, it is a square
	relation.

	Thus, the density of state for $\epsilon\rightarrow\dd{\epsilon}$ is
	\begin{equation}
		D\left(\epsilon\right) =
		4\pik^2\dv{k}{\epsilon}\frac{2V}{{2\pi}^3}.
	\end{equation}

	Replace $k$ with $\epsilon$ to
	get
	\begin{equation}
		\label{eq:density_of_states_photon_gas}
		D\left(\epsilon\right) =
		\frac{V}{\pi^2}\frac{\epsilon^2}{{\hbar c}^3}
	\end{equation}
\end{derivation}

\subsection{Internal energy of photon gas}%
\label{sub:internal_energy_of_photon_gas}%

We also know that the internal energy is expressed as
\begin{equation}
	U = \int_0^\infty\epsilon
	p\left(\epsilon\right)D\left(\epsilon\right)\dd{\epsilon}
\end{equation}

Since we already calculated the DOS, we can substitute that and
$p\left(\epsilon\right)$ and evaluate the integral. We will also
need the standard integral
\begin{equation}
	\int^\infty_0\frac{x^3}{e^x-1}\dd{x}
	=
	\frac{\pi^4}{15}.
\end{equation}

With the substitution $x = \beta\epsilon$, we get
\begin{equation}
	\label{eq:photon_gas_intermediate_derivation}
	U = \frac{V}{\pi^2{\left(\hbar c\right)}^3}
	\int^\infty_0\frac{\epsilon^3\dd{\epsilon}}{e^{\beta\epsilon} - 1} =
	\frac{V}{\pi^2{\left(\hbar c\right)}^3\beta}\int_0^\infty\frac{x^3\dd{x}}{e^x-1}
	= \frac{V{\left(k_B T\right)}^4\pi^2}{15{\left(\hbar c\right)}^3}
\end{equation}

\begin{definition}
	The internal energy of a photon gas $U$ is
	\begin{equation}
		U = \frac{V\pi^2}{15{\left(\hbar c\right)}^3}{\left(k_B T\right)}^4.
	\end{equation}
\end{definition}

\subsection{Stefan-Boltzmann Law (not examinable)}%
\label{sub:stefan_boltzmann_law_not_examinable_}

An example of why we would want to find this quantity is the
radiating power, or the \textit{Stefan-Boltzmann Law}. Here, we need
to relate the energy density to the number of photons hitting a
surface per unit time. This is similar to our analysis of the
kinetic theory of an ideal gas.

Consider $n$ photons per unit volume,
travelling with a speed $c$. In a time
$\delta{t}$ the photons hitting area
$A$ will sweep out a volume (that of a
parallelepiped).
\begin{equation}
	V_\mathrm{photon} = c\delta tA\cos{\theta}.
\end{equation}

If we have a unit area and a unit time, $V_\mathrm{photon} = nc\cos{\theta}$,
then the total number of photons hitting per unit time is
$nc\cos{\theta}$.

But not all the photons in this volume is travelling at the
angle $\theta$. Thus, we want to know the
fraction of photons travelling at angle $\theta\rightarrow\theta + \dd{\theta}$.
Think of the photon gas as a unit sphere, then the number of
photons travelling at an angle $\theta\rightarrow\theta + \dd{\theta}$ lies
within a thin ring with width $\dd{\theta}$, this ring
has a radius $\sin{\theta}$ and thus a volume of
\begin{equation} V_\mathrm{ring} =
	2\pi\sin{\theta}\dd{\theta}.
\end{equation}

Thus, the fraction of photons would be
\begin{equation}
	\frac{V_\mathrm{ring}}{V_\mathrm{sphere}}
	=
	\frac{2\pi\sin{\theta}\dd{\theta}}{4\pi}
	=
	\frac{\sin{\theta}\dd{\theta}}{2}.
\end{equation}

Finally, we want the total number of photons hitting the surface
at this angle as we integrate around the whole ring of width
$\dd{\theta}$, this gives us the result
\begin{equation}
	\int^{\frac{\pi}{2}}_0\frac{nc\cos{\theta}\sin{\theta}\dd{\theta}}{2}
	=
	\frac{nc}{4}.
\end{equation}

Each photon is carrying a radiating power per unit area
$ P_{\mathrm{photon}} =
	\frac{c}{4}\times\textrm{energy density}$.

This energy density is $U$, which we
calculated in Section~\ref{sub:internal_energy_of_photon_gas}. Putting this
into our result and we get
\begin{equation}
	P_\mathrm{\photon} = \frac{c\pi^2k_B^4T^4}{60{\left(\hbar c\right)}^3} = \sigma T^4
\end{equation}
where $\sigma$ is the Stefan-Boltzmann constant.

\section{Spectral density}%
\label{sec:spectral_density}

Our integral for the total energy can be written as
$U/V$. Take the integral from
Equation~\ref{eq:photon_gas_intermediate_derivation}, we call this the spectral
energy density
\begin{equation}
	\frac{U}{V} =
	\frac{1}{\pi^2{\left(\hbar c\right)}^3}.
\end{equation}

We can also write this in terms of frequency, where
$\epsilon = \hbar\omega$, giving us
\begin{equation}
	\frac{U}{V}
	=
	\frac{\hbar}{\pi^2 c^3}\int^\infty_0
	\frac{\omega^3\dd{\omega}}{e^{\beta\hbar\omega} - 1}
	=
	\int^\infty_0 u_\omega\dd{\omega}
\end{equation}
where $u_\omega$ is known as the spectral density.

\begin{definition}
	The spectral density of a Bose-gas is
	\begin{equation}
		\label{eq:spectral_density}
		u_{\omega}
		=
		\frac{\hbar}{\pi^2 c^3}\frac{\omega^3}{\left(e^{\beta\hbar\omega} - 1\right)}
	\end{equation}
\end{definition}

If we plot $u_{\omega}$ as a function of
$\omega$, we get a curve known as
\textit{Wien's displacement law}, where the maximum roughly occurs at
$\hbar\omega\simeq 3k_B T$.

This is one of the early triumphs of quantum mechanics, and it
solves what was known as the ultraviolet catastrophe.

To see where this crisis comes from using the classical
equipartition theorem---in a normal oscillator, there is an
equal amount of potential energy and kinetic energy. Because
there are two degrees of freedom, we get an energy of
$k_B T$ per mode (See
Section~\ref{sec:classical_equipartition_theorem}). Therefore, the total energy
$\dd{U}/\dd{\epsilon}$ is proportional to
$\epsilon^2$ (See Equation~\ref{eq:density_of_states_photon_gas}).
If we integrate for $U$ to infinity, we get
unbounded infinite energy. This is fixed by introducing the Bose
factor which reduces the spectral density at high frequency.

We can derive Wien's displacement law below
\begin{derivation}
	We start by differentiating the spectral density from
	Equation~\ref{eq:spectral_density} with respect to
	$\omega$
	\begin{equation}
		\dv{}{\omega}\left[\frac{\omega^3}{e^{\beta\hbar\omega} - 1}\right]
		=
		\left[\frac{3\omega^2}{e^{\beta\hbar\omega} - 1} - \frac{\omega^3\beta\hbar}{{e^{\beta\hbar\omega} - 1}^2}\right].
	\end{equation}
	We want to find where the peak is, therefore we set this equal
	to 0. Some algebraic manipulation, alongside the assumption that
	$e^{\beta\hbar\omega} \gg 1$ gives us
	\begin{equation}
		3 \simeq \omega\beta\hbar
	\end{equation}
	where $\beta = 1/k_B T$. Rearranging this gives us Wien's
	displacement law as quoted above.
\end{derivation}

We can also derive thermodynamic properties of photon gas using
the spectral density. We start with the entropy.
\begin{derivation}
	From the first law, we can write
	\begin{equation}
		{\left(\pdv{U}{T}\right)}_{V}
		=
		T{\left(\pdv{S}{T}\right)}_{V}
		=
		C_V
	\end{equation}
	Thus,
	\begin{equation}
		S = \int^T_0\frac{1}{T}{\left(\pdv{U}{T}\right)}_{V}\dd{T}.
	\end{equation}

	Now, we differentiate $U$ with respect to
	$T$ to get
	\begin{equation}
		\dv{U}{T} = \frac{16}{c}V\sigma T^3
	\end{equation}

	Put this integral in and evaluate to get
	\begin{equation}
		\label{eq:entropy_of_photon_gas}
		S = \frac{16}{3}\frac{V\sigma}{c}T^3 =
		\frac{4}{3}\frac{U}{T}
	\end{equation}
\end{derivation}

We can also derive the pressure
\begin{derivation}
	Start with the first law and rearrange to get
	\begin{equation}
		p = T\pdv{S}{V} - \pdv{U}{V}
	\end{equation}
	So
	\begin{equation}
		\dv[]{U}{V} = \frac{U}{V}
	\end{equation}
	Take the expression for entropy from
	Equation~\ref{eq:entropy_of_photon_gas}, substitute it in and simplify
	to get
	\begin{equation}
		\label{eq:pressure_of_photon_gas}
		p =\frac{1}{3}\frac{U}{V}.
	\end{equation}
	This means the pressure is a third of the spectral density for a
	photon gas. Recall that for an ideal Fermi gas, the pressure is
	$2/3$ of the energy density. The difference
	comes from the fact that these photons are relativistic---they
	travel at the speed $c$.
\end{derivation}

\subsection{Bose-Einstein gas with massive particles (non-examinable)}%
\label{sub:bose_einstein_gas_with_massive_particles}

If we have bosons with mass (so not photons or phonons), then
$\mu\neq 0$. In fact, it will be negative.
Previously, when calculating $N$ or
$U$, we replaced the sum with the integral.
But bosons can have a large number of particles in a single
state. According to the third law, at $T = 0$
all particles will be in the ground state. Thus, if we replace
the sum with the integral, this gives us a very bad
approximation if most of the particles are in a single state.

When occupancy of the lowest state becomes large compare to the
occupancy of all the other states, there is a phase transition.

This is known as the Bose-Einstein condensation.

\end{document}
% vim: fen fdm=syntax
